\documentclass[fontsize=11pt]{scrbook}
\KOMAoptions{twoside=false}
\usepackage{fontspec}
\setlength\parindent{0pt}
%\usepackage[utf8]{inputenc}
%\usepackage[T1]{fontenc}
\usepackage[british]{babel}

%%%%%%%%%%%%%%%%%%%%%%%%%%%%%%%%%%%%%%%%%%%%%%%%%%%%%%%%%%%%%%%%%%%%%%%
% Satzdinge
\usepackage[onehalfspacing]{setspace}


%%%%%%%%%%%%%%%%%%%%%%%%%%%%%%%%%%%%%%%%%%%%%%%%%%%%%%%%%%%%%%%%%%%%%%%
% Farbdinge
\usepackage{xcolor}
    \definecolor{myred}{RGB}{198,24,38} % Rot der Uni Heidelberg
    \definecolor{mygreen}{RGB}{40,114,51} % Smaragdgrün
    
\usepackage{graphicx}
    % diffrent style of subfigures
% \usepackage{subfigure}
\usepackage[labelfont=bf]{caption}
\usepackage{subcaption}


    
%%%%%%%%%%%%%%%%%%%%%%%%%%%%%%%%%%%%%%%%%%%%%%%%%%%%%%%%%%%%%%%%%%%%%%%
% Displaystyledinge
\newcommand{\block}{ \enskip \textcolor{myred}{\rule[-\dp\strutbox]{2pt}{\baselineskip}} \enskip}
	% Chapter
\renewcommand*{\chapterformat}{\thechapter \block}
\addtokomafont{chapter}{\normalfont \LARGE \scshape \bfseries}
	% Section
\renewcommand*{\sectionformat}{\enskip \thesection \block}
\addtokomafont{section}{\raggedright \normalfont \Large \scshape}
	% Subsection
\renewcommand*{\subsectionformat}{\enskip \enskip \thesubsection \block}
\addtokomafont{subsection}{\raggedright \normalfont \large \scshape}
	% Subsubsection
\renewcommand*{\subsubsectionformat}{\thesubsubsection \block}
\addtokomafont{subsubsection}{\enskip \enskip \enskip \raggedright \normalfont \normalsize \scshape}

%%%%%%%%%%%%%%%%%%%%%%%%%%%%%%%%%%%%%%%%%%%%%%%%%%%%%%%%%%%%%%%%%%%%%%%
% Schriftdinge
%\usepackage{bm,stmaryrd}
%\usepackage{fourier, bm}
%\usepackage[adobe-utopia]{mathdesign}
    % normales mathbb behalten-Dinge
    %\DeclareSymbolFont{AMSb}{U}{msb}{m}{n}
    %\makeatletter
    %\DeclareSymbolFontAlphabet{\math@bb}{AMSb}
    %\AtBeginDocument{\protected\def\mathbb{\math@bb}}
    %\makeatother
    % normales mathcal behalten-Dinge
    %\DeclareMathAlphabet{\mathcal}{OMS}{cmsy}{m}{n}

%%%%%%%%%%%%%%%%%%%%%%%%%%%%%%%%%%%%%%%%%%%%%%%%%%%%%%%%%%%%%%%%%%%%%%%
% Nummerierungsdinge
\usepackage[inline]{enumitem}
    \let\olditemize\itemize
    \renewcommand\itemize{\olditemize\addtolength{\itemsep}{-6pt}}

%%%%%%%%%%%%%%%%%%%%%%%%%%%%%%%%%%%%%%%%%%%%%%%%%%%%%%%%%%%%%%%%%%%%%%%
% Mathedinge
\usepackage{amssymb, amsmath, mathtools, nicefrac}

%%%%%%%%%%%%%%%%%%%%%%%%%%%%%%%%%%%%%%%%%%%%%%%%%%%%%%%%%%%%%%%%%%%%%%%
% Chemiedinge
\usepackage{chemscheme}

% Tricksy Hobbitses don't want autoref, so tricksy trickes are needed ...
%\usepackage{aliascnt}
%\newtheorem{dummy}{***}[chapter]
% \newcommand{\mynewtheorem}[2]{
%   \newaliascnt{#1}{dummy}
%   \newtheorem{#1}[#1]{#2}
%   \aliascntresetthe{#1}
%   \expandafter\def\csname #1autorefname\endcsname{#2}
% Y
% \theoremstyle{definition}
%   \mynewtheorem{theo}{Theorem}
%   \mynewtheorem{coro}{\iflanguage{ngerman}{Korollar}{}\iflanguage{british}{Corollary}{}}
%   \mynewtheorem{lemma}{Lemma}
%   \mynewtheorem{defi}{Definition}
% \theoremstyle{remark}
%   \mynewtheorem{expample}{\iflanguage{ngerman}{Beispiel}{}\iflanguage{british}{Example}{}}
%   \mynewtheorem{remark}{\iflanguage{ngerman}{Bemerkung}{}\iflanguage{british}{Remark}{}}
%   \mynewtheorem{nota}{Notation}
%   \mynewtheorem{interpret}{Interpretation}

% vertical arrow
% \newcommand\vertarrowbox[3][6ex]{%
%   \begin{array}[t]{@{}c@{}} #2 \\
%   \left\uparrow\vcenter{\hrule height #1}\right.\kern-\nulldelimiterspace\\
%   \makebox[0pt]{\scriptsize#3}
%   \end{array}%
% }

%%%%%%%%%%%%%%%%%%%%%%%%%%%%%%%%%%%%%%%%%%%%%%%%%%%%%%%%%%%%%%%%%%%%%%%
% tabllendinge
\usepackage{float}
\floatstyle{plaintop}
\restylefloat{table}


%%%%%%%%%%%%%%%%%%%%%%%%%%%%%%%%%%%%%%%%%%%%%%%%%%%%%%%%%%%%%%%%%%%%%%%
% Kurzbefehldinge
\def\RR{\mathbb R}
\def\CC{\mathbb C}

%%%%%%%%%%%%%%%%%%%%%%%%%%%%%%%%%%%%%%%%%%%%%%%%%%%%%%%%%%%%%%%%%%%%%%%
% Mathmode in heading-Dinge 
%\newcommand{\mathtitle}[2]{\texorpdfstring{$#1$}{#2}}
\usepackage{physics}
\usepackage{siunitx}

\sisetup{
  locale = UK ,
  separate-uncertainty = true,
  exponent-product = \cdot,
  output-product = \cdot
}

%%%%%%%%%%%%%%%%%%%%%%%%%%%%%%%%%%%%%%%%%%%%%%%%%%%%%%%%%%%%%%%%%%%%%%%
% Restklassen- o.ä.dinge
% \makeatletter
% \newsavebox\myboxA
% \newsavebox\myboxB
% \newlength\mylenA

% \newcommand*\xoverline[2][0.75]{%
%     \sbox{\myboxA}{$\m@th#2$}%
%     \setbox\myboxB\null% Phantom box
%     \ht\myboxB=\ht\myboxA%
%     \dp\myboxB=\dp\myboxA%
%     \wd\myboxB=#1\wd\myboxA% Scale phantom
%     \sbox\myboxB{$\m@th\overline{\copy\myboxB}$}%  Overlined phantom
%     \setlength\mylenA{\the\wd\myboxA}%   calc width diff
%     \addtolength\mylenA{-\the\wd\myboxB}%
%     \ifdim\wd\myboxB<\wd\myboxA%
%       \rlap{\hskip 0.5\mylenA\usebox\myboxB}{\usebox\myboxA}%
%     \else
%         \hskip -0.5\mylenA\rlap{\usebox\myboxA}{\hskip 0.5\mylenA\usebox\myboxB}%
%     \fi}
% \makeatother
% \renewcommand{\bar}{\xoverline}

\usepackage[version=4]{mhchem}
%%%%%%%%%%%%%%%%%%%%%%%%%%%%%%%%%%%%%%%%%%%%%%%%%%%%%%%%%%%%%%%%%%%%%%%
% Inhaltsverzeichnisdinge
\usepackage{tocloft}
	\renewcommand{\cftchapfont}{\scshape \bfseries}
	\renewcommand{\cftsecfont}{\scshape}
	\renewcommand{\cftsubsecfont}{\scshape}
	\setcounter{tocdepth}{0}
	\renewcommand{\cfttoctitlefont}{\normalfont \scshape \Large \bfseries}
	\renewcommand{\cftloftitlefont}{\normalfont \scshape \large \bfseries}
    \setlength\cftaftertoctitleskip{6pt} % distance toctitle and toc
    \setlength\cftafterloftitleskip{6pt} % distance toftitle and tof
    %\renewcommand{\cftdot}{} % No dots in toc if enabled

%%%%%%%%%%%%%%%%%%%%%%%%%%%%%%%%%%%%%%%%%%%%%%%%%%%%%%%%%%%%%%%%%%%%%%%
% Abstractdinge
\newenvironment{abstract}
	{ \itshape \footnotesize \begin{center}  \begin{minipage}{.8\textwidth} \begin{singlespace} \textsc{\iflanguage{ngerman}{Abstrakt.}{}\iflanguage{british}{Abstract.}{}}}
	{\end{singlespace} \end{minipage} \end{center} \vspace{12pt}}

%%%%%%%%%%%%%%%%%%%%%%%%%%%%%%%%%%%%%%%%%%%%%%%%%%%%%%%%%%%%%%%%%%%%%%%
% Diagrammdinge
% \usepackage[arrow, matrix, curve]{xy}
\usepackage{float}

%%%%%%%%%%%%%%%%%%%%%%%%%%%%%%%%%%%%%%%%%%%%%%%%%%%%%%%%%%%%%%%%%%%%%%%
% Tabellendinge
\usepackage{booktabs, tabularx, longtable}
\usepackage{csvsimple}
\usepackage{lscape}
\usepackage{rotating}

%%%%%%%%%%%%%%%%%%%%%%%%%%%%%%%%%%%%%%%%%%%%%%%%%%%%%%%%%%%%%%%%%%%%%%%
% Kommentardinge
\usepackage{comment}

%%%%%%%%%%%%%%%%%%%%%%%%%%%%%%%%%%%%%%%%%%%%%%%%%%%%%%%%%%%%%%%%%%%%%%%
% Fußnotendinge
\renewcommand{\thefootnote}{$\mathfrak{[\thechapter.\alph{footnote}]}$}


%%%%%%%%%%%%%%%%%%%%%%%%%%%%%%%%%%%%%%%%%%%%%%%%%%%%%%%%%%%%%%%%%%%%%%%
% Zitatdinge
\usepackage{csquotes, epigraph}
    \let\originalepigraph\epigraph 
    \renewcommand\epigraph[2]{\originalepigraph{\textit{#1}}{\textsc{#2}}}
    \setlength{\epigraphwidth}{0.5\textwidth}

%%%%%%%%%%%%%%%%%%%%%%%%%%%%%%%%%%%%%%%%%%%%%%%%%%%%%%%%%%%%%%%%%%%%%%%
% Bibliographiedinge
\usepackage[backend=biber, style=chem-angew, natbib=true]{biblatex}
    \let\cite=\supercite
    
	\defbibheading{secbib}[]{%
		\section*{#1}%
		\markboth{#1}{#1}}
	\addbibresource{bibliography.bib}
	\renewcommand*{\bibfont}{\footnotesize}
    \setlength{\bibitemsep}{0pt plus 0.3ex}






%%%%%%%%%%%%%%%%%%%%%%%%%%%%%%%%%%%%%%%%%%%%%%%%%%%%%%%%%%%%%%%%%%%%%%%
% Underbracedinge, d.h. schöne geschwungene Klammern
%\usepackage{etoolbox}
% \makeatletter
% \patchcmd{\upbracefill}{\m@th}{\scriptstyle\m@th}{}{}
% \patchcmd{\upbracefill}{$\braceld$}{$\scriptstyle\braceld$}{}{}
% \patchcmd{\upbracefill}{\bracelu}{\bracelu\mkern-1mu}{}{}
% \patchcmd{\upbracefill}{\hfill\braceru}{\hfill\mkern-1mu\braceru}{}{}
% \makeatother

%%%%%%%%%%%%%%%%%%%%%%%%%%%%%%%%%%%%%%%%%%%%%%%%%%%%%%%%%%%%%%%%%%%%%%%
% Codedinge
\usepackage{listings}%, scrhack}
\renewcommand*\ttdefault{cmvtt}
\lstset{language=Matlab, numbers=left, backgroundcolor=\color{black!5}, linewidth=\linewidth, breaklines=true, breakatwhitespace=false, tabsize=2, basicstyle=\ttfamily, extendedchars=true, inputencoding=utf8, commentstyle=\color{black!10!blue}, keywordstyle=\color{myred}, literate=%
		{Ö}{{\"O}}1
		{Ä}{{\"A}}1
		{Ü}{{\"U}}1
		{ß}{{\ss}}1
		{ü}{{\"u}}1
		{ä}{{\"a}}1
		{ö}{{\"o}}1
		{~}{{	extasciitilde}}1
}

%%%%%%%%%%%%%%%%%%%%%%%%%%%%%%%%%%%%%%%%%%%%%%%%%%%%%%%%%%%%%%%%%%%%%%%
% TikZ- und Plot-Dinge
% \usepackage{tikz}
%     \usetikzlibrary{
%         arrows.meta,
%         math
%         }
% \usepackage{pgfplots}
%     \pgfplotsset{compat=1.7}
    
%%%%%%%%%%%%%%%%%%%%%%%%%%%%%%%%%%%%%%%%%%%%%%%%%%%%%%%%%%%%%%%%%%%%%%%
% Todo-Dinge
% \usepackage[
%             color=gray,
%             backgroundcolor=myyellow,
%             %disable
%             ]{todonotes}

%%%%%%%%%%%%%%%%%%%%%%%%%%%%%%%%%%%%%%%%%%%%%%%%%%%%%%%%%%%%%%%%%%%%%%%
% Spielerei-Dinge


%%%%%%%%%%%%%%%%%%%%%%%%%%%%%%%%%%%%%%%%%%%%%%%%%%%%%%%%%%%%%%%%%%%%%%%
% Algorithmendinge
% \usepackage[
%             vlined,
%             ruled,
%             linesnumbered
%             ]{algorithm2e}

%%%%%%%%%%%%%%%%%%%%%%%%%%%%%%%%%%%%%%%%%%%%%%%%%%%%%%%%%%%%%%%%%%%%%%%
% Hurenkinder und Schusterjungen verhindern
\clubpenalty10000
\widowpenalty10000
\displaywidowpenalty=10000

%%%%%%%%%%%%%%%%%%%%%%%%%%%%%%%%%%%%%%%%%%%%%%%%%%%%%%%%%%%%%%%%%%%%%%%
% Ref-Dinge
\usepackage[hidelinks,urlcolor=mygreen,citecolor=mygreen]{hyperref}
    % \renewcommand{\algorithmautorefname}{Algorithm}
\usepackage{microtype}
\usepackage{textcomp}
%%%%%%%%%%%%%%%%%%%%%%%%%
\title{B.Sc. Thesis}
\newcommand{\duedate}{October, 2023}
\author{Ron Bellemann}
%%%%%%%%%%%%%%%%%%%%%%%%%

\begin{document}

\thispagestyle{empty}
\pagenumbering{gobble}{}
\begin{titlepage}
\begin{center}

\vspace*{.06\textheight}
{\scshape\LARGE \textcolor{myred}{\iflanguage{ngerman}{Ruprecht-Karls-Universität Heidelberg}{}\iflanguage{british}{University of Heidelberg}{}} \par}\vspace{1.5cm}
\textsc{\Large \iflanguage{ngerman}{Bericht zum organischen Forschungspraktikum}{}\iflanguage{british}{Research Project Report}{}}\\\vspace{1.5cm}

\hrule \vspace{0.2cm}
{\huge \bfseries \scshape
Computational studies on ground state and excited state properties of Azobenzenetolane derivatives.
}\vspace{0.2cm}\hrule\vspace{1.5cm}
 
\vfill

\begin{minipage}[t]{0.4\textwidth}
\begin{flushleft} \large
\emph{\iflanguage{ngerman}{Vorgelegt von}{}\iflanguage{british}{Author}{}:}\\
\textcolor{myred}{Ron Bellemann}
\end{flushleft}
\end{minipage}
\begin{minipage}[t]{0.4\textwidth}
\begin{flushright} \large
\emph{\iflanguage{ngerman}{Betreuer}{}\iflanguage{british}{Supervisor}{}:}\\
\textcolor{myred}{M.Sc. Nils Oberhof}
\end{flushright}
\end{minipage}\\\vspace{2.5cm}


\iflanguage{ngerman}{
\large \textit{Blubb}\\[.3cm]
\textit{an der}\\[.4cm]
\textcolor{myred}{Fakultät für Chemie und Geowissenschaften, Universität Heidelberg}\\[.3cm]
\textit{am}\\[.4cm]
{\textcolor{myred}{\duedate}}
}{}
\iflanguage{british}{
%\large \textit{Blubb}\\[.3cm]
%\textit{in the}\\[.3cm]
%\textcolor{myred}{Faculty of Chemistry and Earth Sciences , Heidelberg}\\[1cm]
\textit{Handed in on}\\[.3cm]
{\textcolor{myred}{\duedate}}
}{}

\end{center}
\end{titlepage}

\cleardoublepage
    \pagenumbering{roman}
%\begin{abstract}
%    \textcolor{red}{textquatsch} 
%\end{abstract}

% \epigraph{
% The reviewers (I know their names) [...] % Ich möchte, dass sie versuchen zu verstehen, dass es Quatsch ist, was sie erzählen. Dr. Juan Marcos Cerviño 
% % Das ist einfach nur simples CSV-to-Bullshit [...] - Sebastian Pauly
% }{
% Boris Plakhutin
% }

\pagestyle{empty}
\tableofcontents

\vspace{24pt}
%\listoffigures
\newpage
% \vspace{18pt}
% {\normalfont \scshape \large \bfseries List of abbreviations}

% \begin{longtable}{>{\centering\arraybackslash}p{{0.2\textwidth}}p{0.7\textwidth}}
%         Abbreviation & \multicolumn{1}{c}{Definition} \\ \toprule \endhead
%         $n$p-$n$h & $n$ particle $n$ hole \\ \cmidrule{1-2}
%         ADC & Algebraic Diagrammatic Construction \\ \cmidrule{1-2}
%         CI & Configuration Interaction \\ \cmidrule{1-2}
%         CIS(D) & Configuration Interaction - singles and doubles \\ \cmidrule{1-2}
%         DODS & Different orbitals for different spins \\ \cmidrule{1-2}
%         EA & Electron attachment \\ \cmidrule{1-2}
%         EOM-CCSD & Equations of motion - coupled cluster singles and doubles \\ \cmidrule{1-2}
%         FCI & Full configuration interaction \\ \cmidrule{1-2}
%         HF & Hartree-Fock \\ \cmidrule{1-2}
%         IP & Ionisation potential  \\ \cmidrule{1-2}
%         MP & Møller-Plesset perturbation theory \\ \cmidrule{1-2}
%         ROHF & Restricted open-shell Hartree-Fock \\ \cmidrule{1-2}
%         SCF & Self-consistent field \\ \cmidrule{1-2}
%         SODS & Same orbitals for different spins \\ \cmidrule{1-2}
%         TBE & Theoretical best estimate \\ \cmidrule{1-2}
%         UHF & Unrestricted Hartree-Fock \\ \cmidrule{1-2}
%         \bottomrule
% \end{longtable}




\clearpage
\pagestyle{headings}
\pagenumbering{arabic}

%%%%%%%%%%%%%%%%%%%%%%%%%

    \addtocontents{toc}{\protect\setcounter{tocdepth}{1}} % No section numbering in the introductory chapter


% !TEX root = Einleitung.tex
%\chapter{Introduction}
%Kommt noch
% !TEX root = Theorie.tex
\chapter{Theoretical Section}
\section{Density Functional Theory} \label{abs:grundlagen}
\begin{comment}
Modern DFT rests on two theorems by Hohenberg and Kohn (1964). The
first theorem states that the ground-state electron density
uniquely determines the electronic wavefunction and hence
all ground-state properties of an electronic system. The
second theorem establishes that the energy of an electron
distribution can be described as a functional of the electron
density, and this functional is a minimum for the groundstate
density. Thus, the problem of solving the many-body
Schro¨dinger equation is bypassed, and now the objective
becomes to minimize a density functional.
\end{comment}

Since most of the calculations in this report have been done using density functional theory (DFT), a brief overview is given in the following section.\\
First let us define a Hamiltonian $\hat{H}$ 
\begin{align}
    \hat{H} \equiv \hat{T} + \hat{V} + \hat{V}_{\mathrm{ee}}
\end{align}
where (in atomic units)
\begin{align}
    \hat{T} = -\frac{1}{2}\sum^{N}_{j}\nabla_{j}^2, \hspace{1cm} \hat{V} = \sum^{N}_{j}v(r_j), \hspace{1cm} \hat{V}_{\mathrm{ee}} = \frac{1}{2} \sum^{N}_{i\neq j}\frac{1}{r_i - r_j},
\end{align} 
which describes $N$ nonrelativistic, interacting electrons in a nonmagnetic state. $\hat{T}$ denotes the kinetic energy operator and $\hat{V}$ the potential energy operator of the electron interaction and $v(r)$ describes an arbitrary external potential e.g. a Coulomb potential, which are summed over all electrons resulting in $\hat{V}$. \cite{DFT_1} One could now formulate the time-independent Schrödinger equation 
\begin{align}
    \hat{H}\Psi = E\Psi,
\end{align}
where $\Psi$ describes a wave function and then attempt to find approximate solutions to it. However, finding those solutions is complicated, even for small numbers of $N$. At larger $N$ the computational cost is prohibitively expensive, adding to the problem.\\
%A solution to this problem was to use one body density
Instead of using many-body wave-functions, one could use the one-body density, which is the idea behind density functional theory. 
Modern density functional theory rests on two theorems formulated in 1964 by Hohenberg and Kohn, known as HKI and HKII. 
HKI proves that the ground state density $\rho(r)$ determines the external potential energy $v(r)$ to within a trivial additive constant. Since in turn $v(r)$ fixes the Hamiltonian, it follows that the full many-particle ground state is a unique functional of $\rho(r)$.\cite{DFT_2} 
Applying these ideas to total energy, HK formulated the energy functional $E_{v}[\rho(r)]$ 
\begin{align} \label{eq:energy_func}
    E_{v}[\rho(r)] = \int v(r) \rho(r) \mathrm{d}^{3}r + F[\rho(r)] 
\end{align}
for a given potential $v(r)$. $F[\rho]$ describes a universal functional, 
\begin{align} \label{eq:gen_func}
    F[\rho] = T[\rho] + V_{\mathrm{ee}}[\rho] = \bra{\Psi[\rho]}T+V_{\mathrm{ee}}\ket{\Psi[\rho]}
\end{align}
which is valid for any number of particles and external potentials. $\Psi[\rho]$ refers to a N-electron wave-function which yields the density $\rho$ and minimizes the expectation value of the kinetic energy functional $T[\rho]$ and the functional referring to the potential energy of electron interaction $V_{\mathrm{ee}}[\rho]$. \\
The second theorem, HKII, establishes via a variational principle that the ground state energy functional $E_{v_0}$ only yields the ground state energy $E_{0}$, if the true ground state density $\rho_{0}(\textbf{r})$ is inserted. For all other densities $\rho(r)$ the calculated energy $E_{v_0}[\rho]$
\begin{align}
    E_{0} = E_{v_0}[\rho_{0}] < E_{v_0}[\rho]
\end{align}
will be larger than the true ground state energy. \\
However the main problem regarding eq.\,\ref{eq:gen_func} is that both the kinetic $T[\rho]$ and the electron-electron functional are unknown. Direct approximations for both of those functionals exist, the oldest one being the Thomas-Fermi approximation.\cite{DFT_4} Since better approximations have been found its use for practical calculations is rather limited.
Kohn and Sham approached this problem from a different direction by introducing $N$ single-particle ``orbitals'' $\phi_{i}$, which are derived from a single determinant wavefunction describing $N$ non-interacting electrons.\cite{DFT_5} Through this, one can describe the electron density $\rho(r)$
\begin{align}\label{eq:HK_orbitals}
    \rho(r) = \sum^{N}_{i} |{\phi_{i}(r)^2}|
\end{align}
and the kinetic energy $T_{\mathrm{s}}$ of this system
\begin{align}
    T_{\mathrm{s}}[\rho] = - \frac{1}{2}\sum^{N}_{i} \bra{\phi_{i}}\nabla^2\ket{\phi_{i}}
\end{align}
exactly. $T_{s}$ describes the kinetic energy of a system of non-interacting electrons, which produce the true ground state density, but not the true kinetic energy. \\
With those equations we can now partition the universal functional $F[\rho]$ the following way
\begin{align} \label{eq:KS_functional}
    F[\rho] = T_{\mathrm{s}}[\rho] + V_{\mathrm{c}}[\rho] + E_{\mathrm{xc}}[\rho],
\end{align}
where 
\begin{align}
    V_{\mathrm{c}}[\rho] = \frac{1}{2}\int\int\frac{\rho(r) \rho(r')}{|r-r'|} \mathrm{d}^{3}r \hspace{0.1 cm} \mathrm{d}^{3}r'
\end{align}
describes the classical Coulomb interaction of the charge distribution $\rho(r)$ and $E_{\mathrm{xc}}[\rho]$ is the so-called exchange-correlation energy which is defined by 
\begin{align}
    E_{xc}[\rho] = T[\rho] - T_{\mathrm{s}}[\rho] + V_{\mathrm{ee}}[\rho] - V_{\mathrm{c}}[\rho].
\end{align}
$E_{\mathrm{xc}}[\rho]$ can be interpreted as the sum of two errors, the first one arises from using a non-interacting kinetic energy and the second one from treating the electron-electron interaction classically. Using the orbitals which were defined earlier one can reformulate equations \ref{eq:KS_functional} and \ref{eq:energy_func}, thus deriving the following equation
\begin{align} \label{eq:KS_SGL}
    \left(-\frac{1}{2}\nabla^{2} + v(r) + \int\frac{\rho(r')}{|r-r'|} \mathrm{d}^{3}r' + v_{\mathrm{xc}}(r)\right)\phi_i(r) = \epsilon_i \phi_i(r),
\end{align}
which has the form of a single-particle Schrödinger equation. The exchange-correlation potential $v_{\mathrm{xc}}(r)$ is defined by 
\begin{align} \label{eq:KS_Ex}
    v_{\mathrm{xc}}(r) = \frac{\delta E_{\mathrm{xc}}[\rho]}{\delta \rho(r)}.
\end{align}
This set of nonlinear equations (eq.\,\ref{eq:HK_orbitals}, eq.\,\ref{eq:KS_SGL} and eq.\,\ref{eq:KS_Ex}) are called the Kohn-Sham equations and have to be solved self-consistently. This is accomplished by starting with an initial guess for $\rho(r)$, calculating the potentials in eq.\,\ref{eq:KS_SGL} and then solving the equation itself for $\phi_i(r)$. The obtained orbitals can now be used to calculate a new density via eq. \ref{eq:HK_orbitals} and the process is repeated until convergence is reached.

% The solution to this nonlinear problem is usually found
% by starting with an initial guess for n(r), calculating the corresponding vs(r) and then
% solving the differential equation 13 for the φi. From these a new density is calculated
% using eqn. (14) and the process is restarted until reasonable convergence is reached.



% This set of nonlinear equations are called the Kohn-Sham equations and have to be solved self-consistently. They describes the behavior of non-interacting “electrons” in an effective local potential.
% Writing the functional (Equation 15) explicitly in terms of the density built from non-
% interacting orbitals (Equation 14) and applying the variational theorem (Equation 13)
% we find that the orbitals, which minimise the energy, satisfy the following set of
% equations;


% Exc is simply the sum of the error made in using a non-interacting kinetic
% energy and the error made in treating the electron-electron interaction classically.


%and searching for better approximations is subject of current research 
%
% - problem now was to determine the kinetic energy T and the electron electron interaction => Kohn sham
%
%
%
%  - Kohn sham now used introduced a fictitious system of N non-
% interacting electrons to be described by a single determinant wavefunction in N
% “orbitals” φi.
% - kinectic energy is known exactly by 



% the ground state energy can be written as a functional
% of the density, Ev0 [n], which gives the ground-state energy E0 if and only if the true
% ground-state density n0(r) is inserted. For all other densities n(r), the inequality


% where Ψ[n] is that N -electron wave-function which yields the density n and minimizes
% the expectation value of ˆT + ˆVee

\begin{comment}
The first theorem
may be summarised by saying that the energy is a functional of the density – E[ρ]

First they showed that
there is a one-to-one relationship between the external
potential VextðrÞ and the (nondegenerate) GS wave function
Ψ, and that there is a one-to-one relationship between Ψ and
the ground state density nðrÞ of an N-electron system, 14
nðrÞ ¼ N
Z
%dr2  drN Ψ ðr; r2; …; rN ÞΨðr; r2; …; rN Þ;

(HK I) establishes a
one-to-one mapping between the exact electron den-
sity, F(r), and the exact external potential, Vext(r), and
since Vext(r) determines the exact ground-state wave
function Ψ(r), the exact ground-state wave function
is a functional of the electron density, Ψ[F](r).

Thus v (r) is (to within a constant) a unique functional
of n(r); since, in turn, v(r) fixes H we see that the full
many-particle ground state is a unique functional of
rs(r).
\end{comment}


\begin{comment}
We limit ourselves here to the simplest
class of systems, N nonrelativistic, interacting electrons in a
nonmagnetic state with Hamiltonian


Before diving straight into the concept of density functional theory 
first hamiltonian
then HKI and HKII
\end{comment}

\begin{comment}
Modern density functional theory rests on two theorems formulated 1964 by Hohenberg and Kohn, known as HKI and HKII. HKI proves, that the ground state density $\rho(r)$ determines the external potential energy $V_{\mathrm{ext}}(r)$ to within a trivial additive constant. That means, that for a given Hamiltonian e.g.
\begin{align}
    H \equiv T + U + V
\end{align}
where (in atomic units)
\begin{align}
    T \equiv -\frac{1}{2}\sum_{j}\nabla_{j}^2, \hspace{1cm} V \equiv \sum_{j}v(r_j), \hspace{1cm} U = \frac{1}{2} \sum_{i\neq j}\frac{1}{r_i - r_j}
\end{align}
all properties can of $H$ are determined by $\rho(r)$. T denotes the kinetic energy and U the potential energy of N electrons. $v(r)$ describes an arbitrary potential which could be a Coulomb potential. Due to this we can now formulate 

The ground state density n(r)n(r) determines the external potential energy v(r)v(r) to within a trivial additive constant.
    
\end{comment}

% Traditional electronic structure methods attempt to find approximate solutions to the
% Schr¨odinger equation of N interacting electrons moving in an external, electrostatic poten-
% tial (typically the Coulomb potential generated by the atomic nuclei). However, there are
% serious limitations of this approach: (i) the problem is highly nontrivial, even for very small
% numbers N and the resulting wave-functions are complicated objects, (ii) the computational
% effort grows very rapidly with increasing N , so the description of larger systems becomes
% prohibitive.

% A different approach is taken in density functional theory where, instead of the many-
% body wave-function, the one-body density is used as fundamental variable. Since the density
% n(r) is a function of only three spatial coordinates (rather than the 3N coordinates of the
% wave-function), density functional theory is computationally feasible even for large systems.

% Density functional theory uses 

\section{Time-dependent Density Functional Theory}
Time-dependent density functional theory (TDDFT) is an extension of DFT which allows for the calculation of excited-state properties of medium to large molecular systems, e.g. excitation energies, oscillator strengths and excited-state geometries.\cite{TDDFT} To account for example for an applied external time-dependent perturbation $δv(r, t)$ one has to use a time-dependent electronic density $\rho(r, t)$
\begin{align}
    \rho(r, t) = \int |\Psi(r_1, r_2, ..., r_N, t)|^2 \mathrm{d}r_2...\mathrm{d}r_N \mathrm{ ,}
\end{align}
a function of only one space variable $r$ with sufficient degree of freedom to fully describe the response of a system to the time-dependent pertubation. That thesis describes the Runge-Gross theorem\cite{Runge-Gross}, which is a extension of the DFT Hohenberg-Kohn theorem to the time-dependent case. The HK theorem is not sufficient for the study of excitation, which is the response of the system to an external pertubation such as the interaction of a molecule with an electromagnetic wave. The reason for this is that the the total external potential is now the sum of the static external potential $v(r)$ due to the nuclei and the time-dependent external perturbation $\delta v(r, t)$ 
\begin{align}
    v(r, t) = v(r) + \delta v(r, t)
\end{align}
and thus also time-dependent. Since the HK-theorem only holds for static potentials and densities, DFT cannot describe those systems correctly. To summarize, the Runge-Gross theorem formulates, similarly to the HK, that a one-to-one correspondence between the external time-dependent potential $v(r, t)$ and the the time-dependent density $\rho(r, t)$ exists. Its formulation leads to the important statement that every observable $O(t)$ is a unique functional of the time-dependent density (and of the initial state $\Psi_0$): 
\begin{align}
    O(t) = O[\rho, \Psi_0](t).
\end{align}
Using the Runge-Gross theorem one can formulate TDDFT, but instead of focussing on the energy like in DFT, TDDFT uses the action integral $A[\rho]$
\begin{align}
    A[\rho] = \int_{t_0}^{t_1}\bra{\Psi[\rho](r, t)}i\frac{\partial}{\partial t} - \hat{H}(r, t) \ket{\Psi[\rho](r, t)}\mathrm{d}t
\end{align}
Similar to the energy functional of DFT, one may derive the exact time-dependent density $\rho(r, t)$ from the stationary points of action, $ \frac{\delta A[\rho]}{\delta \rho(r, t)} = 0$, by varying the action and searching for stationary points.\\
Another similarity to DFT is the ability to use a Kohn-Sham approach to solve TDDFT, which is achieved by introducing a  Kohn-Sham fictious non-interacting system providing the same density $\rho(r, t)$ of the real interacting system. From this one can formulate a Kohn-Sham potential $v^{\mathrm{KS}}(r, t)[\rho]$ which itself is a functional of the density. This results in a time-dependent one-particle Schrödinger-like equation 
\begin{align}
    i\frac{\partial}{\partial t}\phi_i^{\mathrm{KS}}(r, t) = \left(-\frac{1}{2}\nabla_{i}^{2}+v^{\mathrm{KS}}(r, t)\right)\phi_i^{\mathrm{KS}}(r, t),
\end{align}
where $\phi_i^{KS}(r, t)$ denote the Kohn-Sham wavefunctions. The density can then be derived via
\begin{align}
    \rho(r, t) = \sum_{i}^{\mathrm{occ}} |\phi_i^{\mathrm{KS}}(r, t)|^{2}.
\end{align}
One can now formulate the time-dependent single-particle Kohn-Sham potential $v^{\mathrm{KS}}$
\begin{align}
    v^{\mathrm{KS}}[\rho](r, t) = v(r, t) + v_{\mathrm{H}}[\rho](r, t) + v_{\mathrm{xc}}[\rho](r, t)\mathrm{,}
\end{align}
where $v$ denotes the external potential, $v_{\mathrm{H}}$ a Hartree potential
\begin{align}
    v_{\mathrm{H}}[\rho](r, t) = \int \frac{\rho(r', t)}{|r-r'|}\mathrm{d}^3 r'
\end{align}
and $v_{\mathrm{xc}}$ an exchange-correlation potential
\begin{align}
    v_{\mathrm{xc}}[\rho](r, t) = \frac{\delta A_{\mathrm{xc}}[\rho]}{\delta \rho (r, t)}\mathrm{,}
\end{align}
which is, similar to DFT, unknown and has to be approximated.\cite{TDDFT} One can now follow the Kohn-Sham scheme in a similar fashion to DFT.\\
Despite being a useful theory to calculate excitation energies, spectra and dynamics, there are certain areas in which TDDFT performs rather poorly. One of them is the description Rydberg states and extended $\pi$-systems resulting in errors as large as a few electron volts,\cite{DFT_pi_error_1, DFT_pi_error_2} which is caused by using exchange-correlation functionals that decay faster than $\frac{1}{r}$ where $r$ is the electron-nucleus distance. There are several ways to mitigate this issue, one for example is using asymptotically corrected functionals e.g. LB94.\cite{DFT_pi_corr_1} \\
Additionally TDDFT yields substantial errors for charge-transfer (CT) excited states, resulting in excitation energies which are far lower than the true value and potential energy curves of CT states that do not exhibit the correct $\frac{1}{R}$ asymptote, where R is a distance coordinate between positive and negative charges of the CT state.\cite{DFT_CT_error_1, DFT_CT_error_2}


% by underestimating the excitation energies and 


% one serious drawback of DFT is represented by the non correct representation of conical intersections (CIs) \cite{TDDFT_CI} between potential energy surfaces of the ground and excited states, which play an important role in photochemistry.

% Notes Downfalls of TDDFT:
% - 96,97 



% states a one-to-one correspondence
% between the external time-dependent potential v(r, t) and the time-dependent den-
% sity ρ(r, t)

% In simplified terms the Runge-Gross theorem states he Runge–Gross theorem states a one-to-one correspondence
% between the external time-dependent potential v(r, t) and the time-dependent den-
% sity ρ(r, t)

% The time-dependent density alones a sufficient degree of freedom to fully describe the response of a system to the
% time-dependent perturbation. This is the thesis of the Runge–Gross theorem [3] Sec-
% tion 2), an extension of the DFT Hohenberg–Kohn theorem to the time-dependent
% case



% Time-dependent density-functional theory (TDDFT) is an extension of
% DFT to address excited-state properties, dynamics, and spectroscopy. 

%  it has become one of the most prominentand most widely used approaches for the calculationof excited-state properties of medium to large molec-ular systems, for example, excitation energies, oscil-lator strengths, excited-state geometries, 


\section{Azobenzene}
Azobenzene (AB) is a diazene derivative for which both hydrogen atoms where substituted with phenyl groups. \cite{AB_1} There are two conformers, the  \textit{cis} isomer (cAB) and the thermodynamically more stable \textit{trans} isomer (tAB). The \textit{trans} $\rightarrow$ \textit{cis} isomerization can be induced by a variety of different means, e.g. irradiation with UV-visible light\cite{AB_2}, mechanical stress\cite{AB_3} or electrostatic stimulation\cite{AB_4}, whereas cAB, in addition to the photoinduced isomerization, can undergo also thermal isomerization.\cite{AB_2} The exact mechanism of this isomerization will be discussed later.
%
\begin{figure}[h]
\centering
\begin{subfigure}{.5\textwidth}
  \centering
  \includegraphics[width=.95\linewidth]{Figures/Theory/UV_cisAB.png}
  \caption{cAB}
  \label{fig:UV_exp_cAB}
\end{subfigure}%
\medskip
\begin{subfigure}{.5\textwidth}
  \centering
  \includegraphics[width=1\linewidth]{Figures/Theory/UV_transAB.png}
  \caption{tAB}
  \label{fig:UV_exp_tAB}
\end{subfigure}
\caption{The experimentally obtained molar absorption coefficients $\epsilon$ of \textit{cis}-azobenzene (cAB) \textbf{(a)} and \textit{trans}-azobenzene (tAB) \textbf{(b)} in methanol at different temperatures.\cite{AB_5}}
\label{fig:UV_exp_both_AB}
\end{figure} \\
%
%
In the absorption spectrum of tAB (figure \ref{fig:UV_exp_both_AB} \textbf{(b)}) one can distinguish two different bands in the UV-VIS region. 
The strong absorption band ($\lambda_{max} \sim \SI{320}{\nano\meter}$) can be assigned to the symmetry allowed $\pi\pi^{*}$ (S2 state) transition showing some vibrational structure \cite{AB_6}, while the weaker band ($\lambda_{max} \sim \SI{450}{\nano\meter}$) (S1 state) arises due to a symmetry forbidden n$\pi^{*}$ transition. 
The corresponding n$\pi^{*}$ transition ($\lambda_{max} \sim \SI{270}{\nano\meter}$) (S1 state) of cAB (figure \ref{fig:UV_exp_both_AB} \textbf{(a)} absorbs more strongly, whereas the $\pi\pi^{*}$ band ($\lambda_{max} \sim \SI{440}{\nano\meter}$) (S2 state) is weaker than the transition observed in tAB. \\

Isomerization occurs in both directions upon S1 $\leftarrow$ S0 excitation (QY$_{trans\rightarrow cis} \sim 0.56$, QY$_{cis\rightarrow trans} \sim 0.25$) and S2 $\leftarrow$ S0 excitation (QY$_{trans\rightarrow cis} \sim 0.27$, QY$_{cis\rightarrow trans} \sim 0.11$), where QY denotes the respective quantum yields.\cite{UV-cis-trans-azo-2} One can see quite clearly that S1 excitation QYs are nearly double those measured after S2 excitation, which is clearly a violation of Kasha–Vavilov's rule stating that the quantum yield of luminescence, or in this case isomerization, is generally independent regarding the excitation wavelength.\cite{AB_7} The violation of Kasha–Vavilov's rule together with the fact that the isomerization quantum yields (QY$_{cis\rightarrow trans}$ and QY$_{trans\rightarrow cis}$) do not equal unity implies the existence of several pathways for isomerization. \\
\begin{scheme}[h]
\centering
  \includegraphics[width=1\linewidth]{Figures/Theory/tAB-isomerization.png}
  \caption{${trans\rightarrow cis}$ isomerization-mechanism of inversion-assisted torsion.\cite{AB_8} The red arrows indicate the direction of rotation and translation.}
  \label{fig:inversion-torsion-tAB}
\end{scheme}%
%
To further understand the reason behind those rule violations, one has to investigate the isomerization mechanism of azobenzene. In the case of tAB it is generally accepted that both torsion and bending (``inversion assisted torsion'', see scheme \ref{fig:inversion-torsion-tAB}) lead to an extended conical intersection seam through which the n$\pi^*$ excited state can decay into the ground state. \cite{AB_8} Further research suggested that there is a pathway both common to n$\pi^*$ and $\pi\pi^*$ excitation.  \\
Research by Nenov et al.\cite{AB_8} suggests that the reason for the wavelength dependency of the quantum yield lies whithin an ultrafast decay channel which is only accessable after excitation in the $\pi\pi^*$ state. Using sub-\SI{20}{fs} transient absorption spectroscopy supplemented with computations they were able to observe that the decay of the $\pi\pi^*$ state coincides with the buildup of n$\pi^*$ population which is reached via CNN in-plane bendings, according to their non-linear spectroscopy simulations. After relaxing into a n$\pi^*$ state via a conical intersection those CNN in-plane bending modes allow the system to reach a high-energy planar n$\pi^{*}$/ground state conical intersection which then allows for an ultrafast (450 fs) nonproductive decay of the n$\pi^*$ into the ground state. Contrary to $\pi\pi^*$ excitation the n$\pi^{*}$/ground state conical intersection may not be reached by the system upon excitation into the n$\pi^*$ state which explains the Kasha–Vavilov's rule violation.\cite{AB_8} A schematic view of the PES regarding the photodynamics of tAB is shown in figure \ref{fig:tAB_PES}.\\
\\
%
%
%
Applications for azobenzene compounds range from non-linear optics\cite{AB_nonlinear, AB_nonlinear2}, to optical data storage \cite{AB_memory1} or even bio-medical usage like the release of a payload drug\cite{AB_payload1, AB_payload2} making them an interesting subject of research.
%
%
%
% a high-energy planar n$\pi^*$/ground state via a conical intersection which then further relaxes via an ultrafast (450 fs) 
% Radiative to kinetic energy transfer into these modes drives the system to a high-energy planar nπ*/ground state conical intersection, inaccessible upon directexcitation of the nπ*state, that triggers an ultrafast (0.45 ps) nonproductive decay ofthe nπ*state and is thus responsible for the observed Kasha rule violation in UV excitedtrans-AB. On the other hand, cis-AB is built only after intramolecular vibrational energyredistribution and population of the NN torsional mode.
% A schematic view of the PES regarding the photdynamics of tAB is shown in figure \ref{fig:tAB_PES}.
%
%
%
% which makes use of transient absorption spectroscopy sublemented with suggests that the reason for the wavelength dependence of the quantum yield 
\begin{figure}[h]
\centering
\includegraphics[width=1\linewidth]{Figures/Theory/tAB_PES_scheme.png}
\caption{Schematic PES of tAB in the space of the CNN in-plane bending and CNNC torsion modes after S1 (n$\pi^*$, red) and S2 ($\pi\pi^*$, blue) excitation from the ground state. The $\pi\pi^*$ excited state can decay via $\pi\pi^*$ $\rightarrow \mathrm{CI}$($\pi\pi^*$/n$\pi^*$) $\rightarrow$ n$\pi^*$ $\rightarrow$ $\mathrm{CI}$(n$\pi^*$/GS$^*$) in a sub-picosecond timescale. Another pathway leads to \textit{trans}-\textit{cis} isomerization via intramolecular vibrational energy redistribution into the torsional mode, which happens at a much slower timescale ($2$-$4$ ps) and mirrors the pathway that occurs after S1 excitation. Energy values in eV.\cite{AB_8}}
\label{fig:tAB_PES}
\end{figure}
%\textit{trans} \rightarrow \mathrm{\textit{cis}} isomerization via intramolecular vibrational energy redistribution into the torsional mode, which happens at a much slower timescale ($2$-$4$ ps) and mirrors the pathway that occurs after S1 excitation. Energy values in eV.\cite{AB_8}
%
%
%Nowadays it is generallyaccepted that the decay to the ground state (GS) upon nπ*excitation oftrans-AB involves an extended conical intersection(CI) seam reached through both torsion and bending(“inversion-assisted torsion”,Scheme 1b).
% \textit{Trans}$\rightarrow$\textit{cis} as well as \textit{cis}$\rightarrow$\textit{trans} isomerization occurs upon S1 $\leftarrow$ S0 excitation with a quantum yield of QY$_{trans\rightarrow cis} \sim 0.13$, QY$_{cis\rightarrow trans} \sim 0.25$ and S2 $\leftarrow$ S0 excitation, QY$_{trans\rightarrow cis} \sim 0.5$, QY$_{cis\rightarrow trans} \sim 0.23$ excitation.\cite{UV-cis-trans-azo-2}
\begin{comment}
Looking at quantum yields (QY) for the respective isomerization reactions one finds that excitation in the S2 state leads to 




 Excitation in the S2 state leads to
isomerization quantum yields (QYE-Z B 0.13, QYZ-E B 0.23)
that are nearly half of the yields measured after S1 excitation
(QYE-Z B 0.25, QYZ-E B 0.5), which is a violation of Kasha’s
rule21,22




% \begin{figure}
%     \begin{subfigure}
%         \includegraphics{Figures/Theory/UV_cisAB.PNG}
%     \end{subfigure}
%     \caption{Caption}
%     \label{fig:enter-label}
% \end{figure}
%
Azobenzene (AB) is the structurally simplest representative of aryl azo molecules which are by far the most popular molecular photoswitches. Their application ranges from fields like 
Their photochromism
is based on the E 2 Z isomerization of the NQN bond

NOTES
- AB photochromism based on E->Z isomerisation of the N=N bond (length change 3.4 A)
- E-isomer thermodynamically more stable 
- Z-isomer can undergo photoinduced isomerisation and thermal isomerization
- UV-VIS absorption spectrum of 
    -   E-AB shows weak n -> pi* transition band at 450 nm (S1 state) and strong pi->pi*        transiton band at 315 nm (S2 state)
    -   Z-AB at 440 nm and 260 nm (same states)
- S2 excitation quantum yield (QY (E-Z) 0.13, QY (Z-E) 0.23) nearly half of the yields after S1 excitation (QY (E-Z) 0.25, QY (Z-E) 0.5) 
    => violation of Kashas rule 
        => implies the presence of different relaxation pathways
The lower QYs after
p - p* excitation is explained by the presence of an ultrafast
internal conversion channel from the S2 state cite32,33 (potentially
involving a crossing with a higher singlet state38–40 

Isomerisation mechanisms
- mainly two version
    1) in-plane inversion (one N=N-C angle increases to 180\si{\deg} C-N=N=C dihedral angle remains at 0\si{\deg} 
    2) rotation (torsion of the N=N bond resulting in change of the C-N=N-C dihedral angle to 120 \si{\deg} cite31
    

% Azobenzene (AB) is the structurally simplest representative of aryl azo compounds composed of two phenyl rings, which are interconnected by a N=N bond. Their 
% and are widely employed as 


Together with its more complex derivatives it is widely used 

Azobenzene (AB) and its derivatives are by far the most popular
molecular photoswitches with a wide range of application in the
elds of photobiology, photochemistry and functional organic
materials.1–7 Their attractive light-responsive properties are
based on the ultra-fast photoisomerization between two long-
lived E- and Z-congurations.1–13 In addition, their photo-
chromic and isomerization properties and, in particular, their
activation wavelength can be modulated on demand by suitable
chemical modications, e.g. by introduction of substituents to
the benzene rings. Owing to their adjustable photochemical
properties, AB-based compounds provide a versatile basis for
numerous applications in data storage devices,3 optoelectronic
devices,4 and molecular switches.
\end{comment}
\newpage
\section{Tolane}
Tolane (diphenylacetylene) is a molecule composed of two phenyl groups, which are attached to a acetylene group. Data obtained by x-ray diffraction\cite{tol_xray}, infrared\cite{tol_infra} and Raman\cite{tol_raman} spectra suggests that tolane has $D_{2\mathrm{h}}$ symmetry in its ground-state. Tolane possesses a seemingly non-fluorescent ``dark'' S1 state with a decay time of \SI{200}{\pico\second}
and a fluorescent S2 state with a decay time of \SI{8}{\pico\second}.\cite{tol_states} Excitation with a \SI{277}{\nano\meter} laser resulted in a quantum yield of 0.00336, which increased to 0.5 at a temperature of \SI{77}{K}.\cite{tol_1}
%
%
% quantumyield of $\phi_{\mathrm{F}} = 0.50$ at low temperatures \cite{tol_1}
\begin{figure}[h]
\centering
\includegraphics[width=0.3\linewidth]{Figures/Theory/DiphenylacetyleneSVG.svg.png}
\caption{Tolane (diphenylacetylene).}
\label{fig:tolane}
\end{figure}\\
%
In the realm of anorganic chemistry, tolane is widely used as a ligand for metal complexes, e.g. a $pi$-accepting ligand to modulate the high electron density at d$^{10}$ metal center of Ni(0).\cite{tol_ligand} \\
%
\begin{figure}[h]
\centering
\includegraphics[width=0.7\linewidth]{Figures/Theory/tolan_UV_2.png}
\caption{Absorption spectrum of $\SI{2e-4}{}$ M tolane in 2-methyl pentane.\cite{tol_uv_spec}}
\label{fig:tolane_UV}
\end{figure}\\
%
Another interesting application arises from long chains of tolane molecules which branch off into dendrimers. 
Such structures can act as "photoantennas" and can trap photoexcited electron-hole pairs at its core, which initially formed at the outer edges of those molecules.\cite{tol_dendrimer}\\
In its polymeric form, poly(phenyleneethynylenes) (PEs) can be used as "molecular wires" in which excited states created by absorption of a photon at the polymer backbone.\cite{swager1998molecular}
Those excited states then migrate rapidly to terminal groups, e.g. anthracene groups where they are trapped.
Combined with a terminal group that quences the excitation upon binding to a molecule, PEs can amplify the difference in intensity between the quenched and non-quenched state.\cite{swager1998molecular}
PEs can be synthesized through a Pd-catalyzed Heck-Cassar-Sonogashira-Hagihara reaction or alkyne metathesis.\cite{Bunz2001}


% In its polymeric form, poly(diphenylacetylene) (PDPA) possess a variety of interesting properties. 
% They are quite emissive even in solid state,\cite{tol_emission} which is very unusual since conventional $\pi$-conjugated polymers quench their fluorescence in their solid state due to the strong intermolecular $\pi$-$\pi$ interactions based on the planar geometry.\cite{tol_pi_polymer} 
% The fluorescence emission of PDPAs is highly dependent on external stimuli such as liquid solvents, making them excellent sensor for e.g. the determination of viscosity of various solvents.\cite{tol_sensor}




% Due to the fact that tolane is a conjugated molecule and its $\pi$-system extends itself throughout the whole molecule we used it as the backbone of the molecules which we investigated in this report.\cite{tol_backbone}
%!TEX root = ExperimentalSection.tex
\chapter{Computational methods}
In this work Avogadro 1.2.0 was used to design the various molecular geometries and its universal force-field tool was used to roughly optimize the structure. Further computations were performed by using the Q-chem 5.2 software.\cite{qchem} Most geometry optimizations were performed using the CAM-B3LYP\cite{CAMB3LYP} exchange-correlation functional in combination with the D3(BJ)\cite{DJBJ} dispersion correction and the 6-311G*\cite{6311G} basis set. The benchmark calculation in the beginning made use of additional exchange-correlation functionals (B97,\cite{B97} B97M-rV,\cite{B97MV} $\omega$B97M-V\cite{w97MV}) and basis sets (3-21G,\cite{321G} 3-21G*,\cite{321G} 6-31G,\cite{631G} 6-31G*,\cite{631G} def2-SVPD,\cite{def2} def2-TZVP\cite{def2}). \\
Based on those optimized ground state geometries the non-adiabatic excitation energies of the 20 lowest lying singlet excited states were calculated using TDA-TDDFT\cite{TDA_TDDFT} using the previously mentioned basis sets and functionals. Since manually finding all possible conformers for the ring molecules was too labourous we employed the CREST\cite{crest} (Conformer-Rotamer Ensemble Sampling Tool) software for this task. The 10 energetically lowest structures were then geometrically optimized with the previously mentioned methods. We performed frequency calculations for each molecule to confirm that those are indeed the energetically lowest energies. To visualize the molecules as well as their molecular orbitals the IQmol 2.14.0 software package was used in combination with an isovalue of 0.2 if not otherwise stated.
%
% All computations in this work were performed using the Q-chem 6.1\cite{qchem} software. The molecular geometry was designed with Avogadro 1.2.0 and its universal force-field tool was used to roughly optimize the structure.
%
%\section{Benchmark Sets}
%Test

\chapter{Results} \label{abs:ergebnis}
\section{Benchmark computations of Azobenzene and Tolane}
%
The first objective of this project was to select a functional and basis set combination which worked well across \textit{cis}-AB, \textit{trans}-AB and tolane which are the components of the molecules we are interested in. 
To accomplish this task we calculated the first 20 excited states of each molecule with four different functionals (B97, B97M-rV, CAM-B3LYP and $\omega$B97M-V)with a D3 correction and 8 different basis sets (3-21G, 3-21G*, 6-31G, 6-31G*, 6-311G, 6-311G*, def2-SVPD and def2-TZVP).\\
The resulting states were fitted with a Lorentzian function resulting in figure \ref{fig:uv_fig_cAB}, figure \ref{fig:uv_fig_tAB} and figure \ref{fig:uv_fig_tolane} which were calculated at the CAM-B3LYP/6-311G* level of theory with the D3(BJ) dispersion correction.
%
%After calculating the natural transition orbitals (ntos) for each molecule at the CAM-B3LYP/6-311G* level of theory with the D3(BJ) dispersion correction we assigned 
%\\
%To accomplish this task we calculated the first 20 excited states of each molecule at the CAM-B3LYP/6-311G* level of theory with the D3(BJ) dispersion correction. The resulting states were fitted with a Lorentzian function resulting in UV-spectra \ref{fig:uv_fig_cAB}, \ref{fig:uv_fig_tAB} and \ref{fig:uv_fig_tolane}. After calculating the natural transition orbitals (ntos) of those three molecules we assigned the energetically lowest electronic states to their specific transition which are shown in tables \ref{tab:excited_states_cAB}, \ref{tab:excited_states_tAB} and \ref{tab:excited_states_tolane}.
%
%
\begin{figure}[H]
    \centering
    \includegraphics[scale=0.75]{Figures/Fragments_uv_ink/cAB-uv-ink.eps}
    \caption{UV spectrum of \textit{cis}-azobenzene, calculated at the CAM-B3LYP/6-311G* level of theory with the D3(BJ) dispersion correction. 
    The bars signify the strength and wavelength of the individual transitions. 
    The numbers above certain bands correspond to the respective transition. }
    \label{fig:uv_fig_cAB}
\end{figure}

%
%
%
%
To assign transitions to the respective electronic states natural transition orbitals (ntos) were calculated at the CAM-B3LYP/6-311G* level of theory with the D3(BJ) dispersion correction for cAB, tAB and tolane.\\
The energetically lowest electronic state (S1) of \textit{cis}-azobenzene shown in figure \ref{fig:uv_fig_cAB} can be assigned to the n$\pi^*$ transition, while S2-S5 are $\pi\pi^*$ transitions. 
%
%
\begin{table}[H]
\caption{Type of transition, excitation wavelength $\lambda$ in nm, vertical excitation energies $E$ in eV and their respective oscillator strength $f$ of the five energetically lowest excited states of cAB. Calculated at the CAM-B3LYP/6-311G* level of theory with D3(BJ) correction.}
\label{tab:excited_states_cAB}
\vspace{0.1 cm}
\centering
\begin{tabular}{ccccc}
\toprule
State & Trans. & $\lambda$ (nm) & $E$ $\left(\si{\eV}\right)$ & $f$    \\
\midrule
S1    & n$\pi^*$		& 431			& 2.877				& 0.0286676397          \\
S2    & $\pi\pi^*$ 		& \textbf{256}	& \textbf{4.837}	& \textbf{0.1367460701} \\
S3    & $\pi\pi^*$ 		& 249			& 4.973             & 0.0210386691          \\
S4    & $\pi\pi^*$		& 243			& 5.109             & 0.0797947997          \\
S5    & $\pi\pi^*$		& 241			& 5.152             & 0.0000158155         \\
\bottomrule
\end{tabular}
\end{table}
%
%
%Figures\Fragments_ntos\cab
\begin{figure}[ht]
\centering
\begin{subfigure}{.19\textwidth}
  \centering
  \includegraphics[width=1\linewidth]{Figures/Fragments_ntos/cAB/cAB-s1-att.png}
  %\caption{S1}
%  \label{fig:nto_cab_}
\end{subfigure}%
% \medskip
\begin{subfigure}{.19\textwidth}
  \centering
  \includegraphics[width=1\linewidth]{Figures/Fragments_ntos/cAB/cAB-s2-att.png}
  %\caption{S2}
%  \label{fig:}
\end{subfigure}%
%
\begin{subfigure}{.19\textwidth}
  \centering
  \includegraphics[width=1\linewidth]{Figures/Fragments_ntos/cAB/cAB-s3-att.png}
  %\caption{S3}
%  \label{fig:}
\end{subfigure}
%
\begin{subfigure}{.19\textwidth}
  \centering
  \includegraphics[width=1\linewidth]{Figures/Fragments_ntos/cAB/cAB-s4-att.png}
  %\caption{S4}
%  \label{fig:}
\end{subfigure}
%
\begin{subfigure}{.19\textwidth}
  \centering
  \includegraphics[width=1\linewidth]{Figures/Fragments_ntos/cAB/cAB-s5-att.png}
  %\caption{S5}
%  \label{fig:}
\end{subfigure}
%
\vskip\baselineskip
%
\begin{subfigure}{.19\textwidth}
  \centering
  \includegraphics[width=1\linewidth]{Figures/Fragments_ntos/cAB/cAB-s1-det.png}
  \caption{S1}
%  \label{fig:}
\end{subfigure}%
% \medskip
\begin{subfigure}{.19\textwidth}
  \centering
  \includegraphics[width=1\linewidth]{Figures/Fragments_ntos/cAB/cAB-s2-det.png}
  \caption{S2}
%  \label{fig:}
\end{subfigure}%
%
\begin{subfigure}{.19\textwidth}
  \centering
  \includegraphics[width=1\linewidth]{Figures/Fragments_ntos/cAB/cAB-s3-det.png}
  \caption{S3}
%  \label{fig:}
\end{subfigure}
%
\begin{subfigure}{.19\textwidth}
  \centering
  \includegraphics[width=1\linewidth]{Figures/Fragments_ntos/cAB/cAB-s4-det.png}
  \caption{S4}
  %\label{fig:}
\end{subfigure}
%
\begin{subfigure}{.19\textwidth}
  \centering
  \includegraphics[width=1\linewidth]{Figures/Fragments_ntos/cAB/cAB-s5-det.png}
  \caption{S5}
  %\label{fig:}
\end{subfigure}
%
\caption{Detachment (red) and attachment (blue) densities for the energetically lowest five excited singlet states S1 to S5 of cAB at the CAM-B3LYP/6-311G* with level of theory with an isovalue of 0.02.}
\label{fig:nto_cab}
\end{figure}
\vspace{1 cm}
%
%
%
\begin{figure}[H]
    \centering
    \includegraphics[scale=0.75]{Figures/Fragments_uv_ink/tAB_uv_ink.eps}
    \caption{UV spectrum of \textit{trans}-azobenzene, calculated at the CAM-B3LYP/6-311G* level of theory with D3(BJ) correction.. The bars signify the strength and wavelength of the individual transitions. The numbers above certain bands correspond to the respective transition.}
    \label{fig:uv_fig_tAB}
\end{figure}
%
%
%
Similarly to cAB, the first electronic state can be assigned to the n$\pi^*$ transition. The small intensity can be explained by the symmetry-forbidden nature of the n$\pi^*$ transition. States S2-S5 can be assigned to $\pi\pi^*$ transitions (figure \ref{fig:nto_tab}).
%
%
%
\begin{table}[H]
\caption{Type of transition, excitation wavelength $\lambda$ in nm, vertical excitation energies $E$ in eV and their respective oscillator strength $f$ of the five energetically lowest excited states of tAB. Calculated at the CAM-B3LYP/6-311G* level of theory with D3 correction.}
\label{tab:excited_states_tAB}
\vspace{0.1 cm}
\centering
\begin{tabular}{ccccc}
\toprule
State & Trans. & $\lambda$ (nm)     & $E$ $\left(\si{\eV}\right)$          & $f$    \\
\midrule
S1    & n$\pi^*$	& 439             & 2.824                  & 0.0000541015          \\
S2    & $\pi\pi^*$ 	&\textbf{284}	  & \textbf{4.372}         & \textbf{1.0411258424} \\
S3    & $\pi\pi^*$	& 256             & 4.847                  & 0.0308033979          \\
S4    & $\pi\pi^*$	& 255             & 4.857                  & 0.0000179591          \\
S5    & $\pi\pi^*$	& 219             & 5.664                  & 0.0000219419         \\
\bottomrule
\end{tabular}
\end{table}
%
%
%
%Figures\Fragments_ntos\tab
\begin{figure}[ht]
\centering
\begin{subfigure}{.19\textwidth}
  \centering
  \includegraphics[width=1\linewidth]{Figures/Fragments_ntos/tAB/tAB-s1-att.png}
  %\caption{S1}
%  \label{fig:nto_cab_}
\end{subfigure}%
% \medskip
\begin{subfigure}{.19\textwidth}
  \centering
  \includegraphics[width=1\linewidth]{Figures/Fragments_ntos/tAB/tAB-s2-att.png}
  %\caption{S2}
%  \label{fig:}
\end{subfigure}%
%
\begin{subfigure}{.19\textwidth}
  \centering
  \includegraphics[width=1\linewidth]{Figures/Fragments_ntos/tAB/tAB-s3-att.png}
  %\caption{S3}
%  \label{fig:}
\end{subfigure}
%
\begin{subfigure}{.19\textwidth}
  \centering
  \includegraphics[width=1\linewidth]{Figures/Fragments_ntos/tAB/tAB-s4-att.png}
  %\caption{S4}
%  \label{fig:}
\end{subfigure}
%
\begin{subfigure}{.19\textwidth}
  \centering
  \includegraphics[width=1\linewidth]{Figures/Fragments_ntos/tAB/tAB-s5-att.png}
  %\caption{S5}
%  \label{fig:}
\end{subfigure}
%
\vskip\baselineskip
%
\begin{subfigure}{.19\textwidth}
  \centering
  \includegraphics[width=1\linewidth]{Figures/Fragments_ntos/tAB/tAB-s1-det.png}
  \caption{S1}
%  \label{fig:}
\end{subfigure}%
% \medskip
\begin{subfigure}{.19\textwidth}
  \centering
  \includegraphics[width=1\linewidth]{Figures/Fragments_ntos/tAB/tAB-s2-det.png}
  \caption{S2}
%  \label{fig:}
\end{subfigure}%
%
\begin{subfigure}{.19\textwidth}
  \centering
  \includegraphics[width=1\linewidth]{Figures/Fragments_ntos/tAB/tAB-s3-det.png}
  \caption{S3}
%  \label{fig:}
\end{subfigure}
%
\begin{subfigure}{.19\textwidth}
  \centering
  \includegraphics[width=1\linewidth]{Figures/Fragments_ntos/tAB/tAB-s4-det.png}
  \caption{S4}
  %\label{fig:}
\end{subfigure}
%
\begin{subfigure}{.19\textwidth}
  \centering
  \includegraphics[width=1\linewidth]{Figures/Fragments_ntos/tAB/tAB-s5-det.png}
  \caption{S5}
  %\label{fig:}
\end{subfigure}
%
\caption{Detachment (red) and attachment (blue) densities for the energetically lowest five excited singlet states S1 to S5 of tAB at the {CAM-B3LYP/6-311G*} level of theory with an isovalue of 0.02.}
\label{fig:nto_tab}
\end{figure}
\vspace{1 cm}
%
%
%
\begin{figure}[H]
    \centering
    \includegraphics[scale=0.75]{Figures/Fragments_uv_ink/tolane_uv_ink.eps}
    \caption{UV spectrum of tolane, calculated at the CAM-B3LYP/6-311G* level of theory with a D3(BJ) dispersion correction. 
    The bars signify the strength and wavelength of the individual transitions. 
    The numbers above certain bands correspond to the respective transition. }
    \label{fig:uv_fig_tolane}
\end{figure}
%
%
%
Figure \ref{fig:uv_fig_tolane} shows the UV-spectrum of tolane, while the energies of the electronic states of tolane are shown in table \ref{tab:excited_states_tolane} and their ntos are shown in figure \ref{fig:nto_tolane}.\\
All five electronic states can be assigned to $\pi\pi^*$ transitions.
%
%
%
\begin{table}[H]
\caption{Type of transition, excitation wavelength $\lambda$ in nm, vertical excitation energies $E$ in eV and their respective oscillator strength $f$ of the five energetically lowest excited states of tolane. 
Calculated at the CAM-B3LYP/6-311G* level of theory with a D3(BJ) correction.}
\label{tab:excited_states_tolane}
\vspace{0.1 cm}
\centering
\begin{tabular}{ccccc}
\toprule
State & Trans. & $\lambda$ (nm)     & $E$ $\left(\si{\eV}\right)$          & $f$    \\
\midrule
S1    & $\pi\pi^*$	& \textbf{263}    & \textbf{4.7061}        & \textbf{1.1157256742} \\
S2    & $\pi\pi^*$	& 238             & 5.2116                 & 0.0000370134          \\
S3    & $\pi\pi^*$	& 238             & 5.2184                 & 0.0000015931          \\
S4    & $\pi\pi^*$	& 233             & 5.3175                 & 0.0000001004          \\
S5    & $\pi\pi^*$	& 199             & 6.2348                 & 0.0000046560         \\
\bottomrule
\end{tabular}
\end{table}
%
%
%
%Figures\Fragments_ntos\tolan
\begin{figure}[ht]
\centering
\begin{subfigure}{.19\textwidth}
  \centering
  \includegraphics[width=1\linewidth]{Figures/Fragments_ntos/tolan/tolane-s1-att.png}
  %\caption{S1}
%  \label{fig:nto_cab_}
\end{subfigure}%
% \medskip
\begin{subfigure}{.19\textwidth}
  \centering
  \includegraphics[width=1\linewidth]{Figures/Fragments_ntos/tolan/tolane-s2-att.png}
  %\caption{S2}
%  \label{fig:}
\end{subfigure}%
%
\begin{subfigure}{.19\textwidth}
  \centering
  \includegraphics[width=1\linewidth]{Figures/Fragments_ntos/tolan/tolane-s3-att.png}
  %\caption{S3}
%  \label{fig:}
\end{subfigure}
%
\begin{subfigure}{.19\textwidth}
  \centering
  \includegraphics[width=1\linewidth]{Figures/Fragments_ntos/tolan/tolane-s4-att.png}
  %\caption{S4}
%  \label{fig:}
\end{subfigure}
%
\begin{subfigure}{.19\textwidth}
  \centering
  \includegraphics[width=1\linewidth]{Figures/Fragments_ntos/tolan/tolane-s5-att.png}
  %\caption{S5}
%  \label{fig:}
\end{subfigure}
%
\vskip\baselineskip
%
\begin{subfigure}{.19\textwidth}
  \centering
  \includegraphics[width=1\linewidth]{Figures/Fragments_ntos/tolan/tolane-s1-det.png}
  \caption{S1}
%  \label{fig:}
\end{subfigure}%
% \medskip
\begin{subfigure}{.19\textwidth}
  \centering
  \includegraphics[width=1\linewidth]{Figures/Fragments_ntos/tolan/tolane-s2-det.png}
  \caption{S2}
%  \label{fig:}
\end{subfigure}%
%
\begin{subfigure}{.19\textwidth}
  \centering
  \includegraphics[width=1\linewidth]{Figures/Fragments_ntos/tolan/tolane-s3-det.png}
  \caption{S3}
%  \label{fig:}
\end{subfigure}
%
\begin{subfigure}{.19\textwidth}
  \centering
  \includegraphics[width=1\linewidth]{Figures/Fragments_ntos/tolan/tolane-s4-det.png}
  \caption{S4}
  %\label{fig:}
\end{subfigure}
%
\begin{subfigure}{.19\textwidth}
  \centering
  \includegraphics[width=1\linewidth]{Figures/Fragments_ntos/tolan/tolane-s5-det.png}
  \caption{S4}
  %\label{fig:}
\end{subfigure}
%
\caption{Detachment (red) and attachment (blue) densities for the energetically lowest five excited singlet states S1 to S5 of tolane at the CAM-B3LYP/6-311G* + D3(BJ) level of theory with an isovalue of 0.02.}
\label{fig:nto_tolane}
\end{figure}
%\vspace{1 cm}
%
%
%
To obtain the functional and basis set combination that performs the best across cAB, tAB and tolane, we compared the calculated energy values of the first and second excited state of each molecule to the energetically lowest and second lowest absorption maximum of their respective UV-spectra from literature.\cite{UV-cis-trans-azo-1, UV-cis-trans-azo-2, UV-tolane-ref}\\
%After obtaining the energy values of the 20 energetically lowest electronic states for each functional and basis set, we wanted %to determine the best combination which worked well across \textit{cis}-AB, \textit{trans}-AB and tolane. To accomplish this %task we compared the energy value of the first and second excited state were compared
To compare the deviation in energy of both excited states the relative error of each excited state $\Delta E\left(\text{ES1}\right)$ and $\Delta E\left(\text{ES2}\right)$ was calculated (equation \ref{eq:relative_error_ES1} and \ref{eq:relative_error_ES2}). The total deviation in energy of both excited states $\Delta E\left(\text{ES1, ES2}\right)$ was then derived by averaging over the relative error of the individual excited states (equation \ref{eq:relative_error_ES1_ES2}).
%
\begin{equation}
\label{eq:relative_error_ES1}
    \Delta E\left(\text{ES1}\right) = \text{abs}\left(\frac{E_{\text{Comp}}(\text{ES1})-E_{\text{Exp}}(\text{ES1})}{E_{\text{Exp}}(\text{ES1})}\right)
\end{equation}

\begin{equation}
\label{eq:relative_error_ES2}
    \Delta E\left(\text{ES2}\right) = \text{abs}\left(\frac{E_{\text{Comp}}(\text{ES2})-E_{\text{Exp}}(\text{ES2})}{E_{\text{Exp}}(\text{ES2})}\right)
\end{equation}

$E_{\text{Comp}}(\text{ES1})$ and $E_{\text{Comp}}(\text{ES2})$ denote the computed excitation energy values and $E_{\text{Exp}}(\text{ES1})$ and $E_{\text{Exp}}(\text{ES2})$ the experimentally derived energy values for the first and second excitation energies.
\begin{equation}
\label{eq:relative_error_ES1_ES2}
    \Delta E\left(\text{ES1, ES2}\right) = \frac{1}{2} \left(\Delta E\left(\text{ES1}\right) + \Delta E\left(\text{ES2}\right)\right)
\end{equation}


% \begin{equation}
% \begin{split}
% \Delta E(\text{relative error}) = 
% \frac{1}{2} \text{abs}\left( \frac{E_{\text{Comp}}(\text{ES1})-E_{\text{Exp}}(\text{ES1})}{E_{\text{Exp}}(\text{ES1})} \right) \\
% + \frac{1}{2} \text{abs}\left( \frac{E_{\text{Comp}}(\text{ES2})-E_{\text{Exp}}(\text{ES2})}{E_{\text{Exp}}(\text{ES2})} \right)
% \end{split}
% \end{equation}

% To select the ideal functional and basis set for this research project the first two excited states were computed and compared with their respective experimental values. 

\begin{figure}[H]
    \centering
    \includegraphics[scale=0.75]{Figures/cAB_Benchmark_Plot.eps}
    \caption{\textit{cis}-AB: Relative errors of the excitation energies (eq. \ref{eq:relative_error_ES1_ES2}) calculated by the functionals mentioned in the label and plotted against the respective basis sets.}
    \label{fig:cAB_Benchmark}
\end{figure}
%
Focusing on \textit{cis}-AB, the first excited state represents a $\text{n}\pi^{*}$ transition whereas the second excited state corresponds to a $\pi\pi^{*}$ transition. 
The CAM-B3LYP functional in combination with the 6-311G$^{*}$ basis set yielded a satisfying result in relation to the minimal relative error and their respective computational cost. \newline
The calculated transition energies which were obtained from the CAM-B3LYP/6-311G* + D3(BJ) calculation as well as the experimentally derived values are shown in table \ref{tab:cAB-Exp-calc}.\\
% The calculated energies for the excited states transitions derived by it tended to be a bit higher than the experimental values:
% \begin{equation*}
% \begin{split}
%     E_{\text{Comp}}(\text{ES1}) &= \SI{2.877}{\eV} \\
%     E_{\text{Exp}}(\text{ES1}) &= \SI{2.82}{\eV}
% \end{split}
% \end{equation*}
% for the first excited state and
% \begin{equation*}
% \begin{split}
%     E_{\text{Comp}}(\text{ES1}) &= \SI{4.836}{\eV} \\
%     E_{\text{Exp}}(\text{ES1}) &= \SI{4.77}{\eV}
% \end{split}
% \end{equation*}
% for the second excited state.
%
% $\left(E_{\text{Comp}}(\text{ES1}) = \SI{2.877}{\eV}, E_{\text{Exp}}(\text{ES1}) = \SI{2.82}{\eV} \text{  and  } E_{\text{Comp}}(\text{ES2}) = \SI{2.877}{\eV}, E_{\text{Exp}}(\text{ES2}) = \SI{2.82}{\eV}\right)$
%
% Camb3lyp
% calculated: 2.877	4.836
% experimental: 2.820	4.770
%
\begin{table}[H]
    \caption{Type of transition and vertical excitation energies of \textit{cis}-AB, calculated at the CAM-B3LYP/6-311G* + D3(BJ) level of theory $(E_{\text{calc}})$ and experimentally obtained ($E_{\text{exp}}$).}
    \label{tab:cAB-Exp-calc}
    \vspace{0.1 cm}
    \centering
    \begin{tabular}{ccSS}
        \toprule
        \multicolumn{1}{c}{State} & \multicolumn{1}{c}{Transition} & \multicolumn{1}{c}{$E_{\text{calc}}$ (\si{\eV})}  & \multicolumn{1}{c}{$E_{\text{exp}}$ (\si{\eV})\cite{UV-cis-trans-azo-2}} \\
        \midrule
        S1    & n$\pi^{*}$      &   2.877           & 2.82\\
        S2    & $\pi\pi^{*}$    &   4.836           & 4.77\\
        \bottomrule
    \end{tabular}
\end{table}%
%
It appears that the calculated excitation energies tend to be marginally higher than the experimentally obtained ones.
%
%\multicolumn{1}{c}{State} & \multicolumn{1}{c}{Transition} & \multicolumn{1}{S}{$E_{\text{calc}}$ (\si{\eV})}  & \multicolumn{1}{S}{$E_{\text{exp}}$ (\si{\eV})} \\ 
\begin{comment}
\begin{table}[H]
\caption{cAB}
\centering
\begin{tabular}{llll}
\toprule
State & Excitation energy (eV) & Oscillator strength  & Exp. Value (eV) \\ 
\midrule
S1    & 2.877                  & 0.03                & \\
S2    & \textbf{4.836}         & \textbf{0.14}       & \\
S3    & 4.973                  & 0.02                & \\
S4    & 5.109                  & 0.08                & \\
S5    & 5.152                  & \SI{1.58e-5}{}      & \\
S6    & \textbf{5.638}         & \textbf{0.16}       & \\ 
\bottomrule
\end{tabular}
\end{table}
\end{comment}
%
%
\begin{figure}[H]
    \centering
    \includegraphics[scale=0.75]{Figures/tAB_Benchmark_Plot.eps}
    \caption{\textit{trans}-AB: Relative errors of the excitation energies (eq. \ref{eq:relative_error_ES1_ES2}) calculated by the functionals mentioned in the label and plotted against the respective basis sets.}
    \label{fig:tAB_Benchmark}
\end{figure}
%
In the case of \textit{trans}-AB, the first excited state is a $\text{n}\pi^{*}$ transition and the second one represents $\pi\pi^{*}$ transition, mirroring the first two transitions in \textit{cis}-AB.
CAM-B3LYP performed quite well in combination with the 6-311G* basis set and is only slightly outperformed by B97 and B97M-rV as seen in Figure \ref{fig:tAB_Benchmark}. \\
Similarly to \textit{cis}-AB, using the CAM-B3LYP functional lead to a slight overestimation of the first vertical excitation energy. 
The biggest difference appeared in relation to the second vertical excitation energy in which the experimental value is over \SI{0.4}{\eV} lower than the calculated energy as shown in table \ref{tab:tAB-Exp-calc}. \\ 
Since this deviation is greater in tAB than in cAB, one could assume that CAM-B3LYP overestimates conjugation effects which affect tAB in a greater way than cAB, due to its planar geometry. 
This could lead to a lower MO energy of $\pi$ and a higher MO energy of $\pi^{*}$, thus leading to a increased transition energy.
% \begin{equation*}
% \begin{split}
%     E_{\text{Comp}}(\text{ES1}) &= \SI{2.877}{\eV} \\
%     E_{\text{Exp}}(\text{ES1}) &= \SI{2.82}{\eV}
% \end{split}
% \end{equation*}
% for the first excited state and
% \begin{equation*}
% \begin{split}
%     E_{\text{Comp}}(\text{ES1}) &= \SI{4.836}{\eV} \\
%     E_{\text{Exp}}(\text{ES1}) &= \SI{4.77}{\eV}
% \end{split}
% \end{equation*}
% for the second excited state.
%
\begin{table}[H]
\caption{Type of transition and vertical excitation energies of \textit{trans}-AB, calculated at the CAM-B3LYP/6-311G* + D3(BJ) level of theory $(E_{\text{calc}})$ and experimentally obtained ($E_{\text{exp}}$).}
\label{tab:tAB-Exp-calc}
\vspace{0.1 cm}
\centering
\begin{tabular}{ccSS}
\toprule
\multicolumn{1}{c}{State} & \multicolumn{1}{c}{Transition} & \multicolumn{1}{c}{$E_{\text{calc}}$ (\si{\eV})}  & \multicolumn{1}{c}{$E_{\text{exp}}$ (\si{\eV})\cite{UV-cis-trans-azo-2}} \\
\midrule
S1    & n$\pi^{*}$      &   2.824           & 2.79\\
S2    & $\pi\pi^{*}$    &   4.372           & 3.95\\
\bottomrule
\end{tabular}
\end{table}
%
\begin{comment}
\begin{table}[H]
\caption{tAB}
\centering
\begin{tabular}{llll}
\toprule
State & Excitation energy (eV) & Oscillator strength  & Exp. Value (eV) \\ \midrule
S1    & 2.824                & $5.4 \cdot 10^{-5}$             & 2.79            \\
S2    & \textbf{4.372}       & \textbf{1}                      & 3.95            \\
S3    & 4.847                & 0.03                            & -               \\
S4    & 4.867                & $1.8 \cdot 10^{-5}$             & -               \\ \bottomrule
\end{tabular}
\end{table}
\end{comment}
%
\begin{figure}[H]
    \centering
    \includegraphics[scale=0.75]{Figures/tolan_Benchmark_Plot.eps}
    \caption{Tolane: Relative errors of the excitation energies (eq. \ref{eq:relative_error_ES1_ES2}) calculated by the functionals mentioned in the label and plotted against the respective basis sets.}
    \label{fig:tolan_Benchmark}
\end{figure}
%
%
The first and second excited states correspond both to $\pi\pi^{*}$ transitions (see figure \ref{fig:NTOS-tolane}) and their respective transition energies are shown in table \ref{tab:tolane-Exp-calc}. 
The benchmark yielded an interesting result. It appears that the higher rung functionals, CAM-B3LYP and $\omega$B97M-V, perform worse than their lower rung counterparts especially in bigger basis sets (def2-SVPD, def2-TZVP). 
Similar to the case of tAB, one explanation could be the overestimation of conjugation effects which affect a flat molecule with an extensive $\pi$ MO greatly. 
This would explain why in the case of tolane both transition energies deviate greatly from the experimental value while in the case of tAB and cAB only the $\pi\pi^{*}$ transition was majorly affected. 
%
\begin{table}[H]
\caption{Type of transition and vertical excitation energies of tolane, calculated at the CAM-B3LYP/6-311G* + D3(BJ) level of theory $(E_{\text{calc}})$ $(E_{\text{calc}})$ and experimentally obtained ($E_{\text{exp}}$).}
\label{tab:tolane-Exp-calc}
\vspace{0.1 cm}
\centering
\begin{tabular}{ccSS}
\toprule
\multicolumn{1}{c}{State} & \multicolumn{1}{c}{Transition} & \multicolumn{1}{c}{$E_{\text{calc}}$ (\si{\eV})}  & \multicolumn{1}{c}{$E_{\text{exp}}$ (\si{\eV})\cite{UV-tolane-ref}} \\
\midrule
S1    & $\pi\pi^{*}$    &   4.706           & 4.18 \\
S2    & $\pi\pi^{*}$    &   5.212           & 4.32 \\
\bottomrule
\end{tabular}
\end{table}
%
%
Due to historical data in which CAM-B3LYP performed well for azobenzene molecules and the results of our benchmark, we chose CAM-B3LYP/6-311G* + D3(BJ) as the method of choice for further analysis of the molecules of interest.\\
An alternative functional and basis set combination one could use would be the B97M-rV/6-311G* since it performed especially well for tolane with some deficits concerning tAB.
\begin{comment}
The calculated energies for the excited states transitions derived by it tended to be a bit higher than the experimental values:
\begin{equation*}
\begin{split}
    E_{\text{Comp}}(\text{ES1}) &= \SI{2.877}{\eV} \\
    E_{\text{Exp}}(\text{ES1}) &= \SI{2.82}{\eV}
\end{split}
\end{equation*}
for the first excited state and
\begin{equation*}
\begin{split}
    E_{\text{Comp}}(\text{ES1}) &= \SI{4.836}{\eV} \\
    E_{\text{Exp}}(\text{ES1}) &= \SI{4.77}{\eV}
\end{split}
\end{equation*}
for the second excited state.
\end{comment}
%
% \begin{figure}[H]
%     \centering
%     \begin{subfigure}[h]{1.0\textwidth}
%         \centering
%         \includegraphics[scale=0.8]{Figures/tAB_Benchmark_Plot.eps}    
%         \subcaption[Internal Name]{Benchmark tAB}
%         \label{fig:tAB_Benchmark}
%     \end{subfigure}
%     \begin{subfigure}[h]{1.0\textwidth}
%         \centering
%         \includegraphics[scale=0.8]{Figures/tolan_Benchmark_Plot.eps}    
%         \subcaption[Internal Name]{ADC(2)}
%         \label{fig:tolan_Benchmark}
%     \end{subfigure}
%     \caption{Signed Error Distributions of energies in comparison to the TBE, computed with exp. QChem 5.2 and ROHF-DODS-ADC(n)/aug-cc-pVTZ.}
%     \label{fig:Benchmark_all}
% \end{figure}
%
\begin{comment}
\begin{table}[H]
\caption{tolane}
\centering
\begin{tabular}{llll}
\toprule
State & Excitation energy (eV) & Oscillator strength & Exp. Value (eV) \\ \midrule
S1    & \textbf{4.706}          & \textbf{1.1}        &                 \\
S2    & 5.212                   & $3.7\cdot 10^{-5}$           &                 \\
S3    & 5.218                   & $1.6\cdot 10^{-6}$           &                 \\
S4    & 5.318                   & $1.0\cdot 10^{-7}$           &                 \\ \bottomrule
\end{tabular}
\end{table}
\end{comment}
%
%
%
%
%
%
\section{Conformer analysis}
%
After selecting the functional and basis set best suited for azobenzene and tolane, we wanted to investigate larger molecules consisting of azobenzene and tolane which where linked with methylene molecules. 
The resulting four ring molecules were classified by the configuration of the AB (\textit{cis}-AB or \textit{trans}-AB) and the number of methylene linkers (C1, C3). \\
To determine the conformer with the lowest ground state energy, we had to to sample through a large number of different conformers. 
At first this was done by manually creating raw geometries which were then further refined using DFT. 
Due to the sheer size of possible conformers we had to automate this procedure by using the sampling program CREST which is based on GFNn-xTB methods.\cite{crest} 
The resulting geometries of the 10 energetically lowest rotamers of the corresponding conformers of tAB-1c-tolane and tAB-3c-tolane were then optimized at the CAM-B3LYP/6-311G* + D3(BJ) level of theory.
Since we wanted to model the \textit{trans}-\textit{cis} isomerisation, we needed to obtain the geometries of the corresponding \textit{cis}-isomers cAB-1c-tolane and cAB-3c-tolane.
This was accomplished by decreasing the absolute value of the the dihedral angle of torsion (∢~C-N=N-C) of tAB-1c-tolane-rot9 and tAB-3c-tolane-rot3, such that the required amount of torsion to reach 0\si{\degree} is minimized,
e.g. the minimal amount of steps are needed to hypothetically reach 0\si[]{\degree}.
The angle of torsion was varied in steps of 20\si{\degree} and at each step a geometry optimisation was performed at the CAM-B3LYP/6-311G* + D3(BJ) level of theory.
Since the calculations didn't succeed after a dihedral angle of 100\si[]{\degree} was reached, 
we had to manually decrease the dihedral angle to about 80\si[]{\degree} at which point we performed an unconstrained geometry optimisation at the CAM-B3LYP/6-311G* + D3(BJ) level of theory which yielded the isomers seen in figure~\ref{fig:Conformers}.
%
%
% This was accomplished by decreasing the absolute value of the the dihedral angle of torsion (∢ C-N=N-C) of tAB-1c-tolane-rot9 and tAB-3c-tolane-rot3, such that the required amount of torsion to reach 0\si{\degree} is minimized.
% The angle of torsion was varied in steps of 20\si{\degree} and a geometry optimisation was performed at the CAM-B3LYP/6-311G* + D3(BJ) level of theory
%∢
% The geometries of the corresponding cAB-1c-tolane and cAB-3c-tolane molecules were then obtained by 
%
The final geometries with the lowest energy for each molecule are depicted in figure \ref{fig:Conformers} and their respective energies are shown in table \ref{tab:GS-energy-conformers}.
%
% After selecting the functional and basis set best suited for azobenzene and tolane, we wanted to investigate larger molecules consisting of azobenzene and tolane which where linked with methylene molecules. 
% The resulting four ring molecules were classified by the configuration of the AB (\textit{cis}-AB or \textit{trans}-AB) and the number of methylene linkers (C1, C3). \\
% To determine the conformer with the lowest ground state energy, we had to to sample through a large number of different conformers. 
% At first this was done by manually creating raw geometries which were then further refined using DFT. 
% Due to the sheer size of possible conformers we had to automate this procedure by using the sampling program CREST which is based on GFNn-xTB methods.\cite{crest} 
% The resulting geometries of the 10 energetically lowest rotamers of the corresponding conformers were then optimized with the CAM-B3LYP functional with a D3(BJ) dispersion correction. 
% The final geometries with the lowest energy for each molecule are depicted in figure \ref{fig:Conformers} and their respective energies are shown in table \ref{tab:GS-energy-conformers}.
%
\begin{table}[ht]
\caption{Ground state energies of the \textit{trans} and \textit{cis} AB-tolane molecules.}
\label{tab:GS-energy-conformers}
\centering
\vspace{0.1 cm}
\begin{tabular}{cc} 
\toprule
Molecule           & \textit{E} (\si{\eV}) \\
\midrule
tAB-1c-tolane-rot9 & -32329.113  \\
%cAB-1c-tolane-rot7 & -32328.584 \\
cAB-1c-tolane-tor  & -32328.568 \\
\midrule
tAB-3c-tolane-rot3 & -36607.152 \\
%cAB-3c-tolane-rot9 & -36606.514 \\
cAB-3c-tolane-tor  & -36606.124 \\
\bottomrule
\end{tabular}
\end{table}
%
%
\begin{comment}
To determine the conformer with the lowest ground state energy, we had to to sample through a large number of different conformers. To automate this procedure the program CREST was used. The resulting ten lowest rotamers of the corresponding conformers were then optimized with DFT CAM-B3LYP D3\_BJ and the resulting GS energy is listed in Table (TABLE REF).

Since we were interested in conformers with the lowest ground state energy, we initially modeled those manually and then optimised the raw structures using DFT. Due to the vast possibilities in geometries this task was automated using 
\end{comment}
%
%
\begin{figure}[H]
\centering
\begin{subfigure}{.33\textwidth}
  \centering
  \includegraphics[width=.8\linewidth]{Figures/Conformer/tAB-1c-tolane-crest-rot9.png}
  \caption{tAB-1c-tolane-rot9}
  \label{fig:tAB-1c-tolane-crest-rot9}
\end{subfigure}%
%\medskip
% \begin{subfigure}{.33\textwidth}
%   \centering
%   \includegraphics[width=.9\linewidth]{Figures/Conformer/cAB-1c-tolane-crest-rot7.png}
%   \caption{cAB-1c-tolane-rot7}
%   \label{fig:cAB-1c-tolane-crest-rot7}
% \end{subfigure}
\begin{subfigure}{.33\textwidth}
  \centering
  \includegraphics[width=.8\linewidth]{Figures/Conformer/cAB-for-tAB-1c-tolane-crest-rot9.png}
  \caption{cAB-1c-tolane-tor}
  \label{fig:cAB-1c-tolane-tor}
\end{subfigure}
\vskip\baselineskip
\begin{subfigure}{.33\textwidth}
  \centering
  \includegraphics[width=.9\linewidth]{Figures/Conformer/tAB-3c-tolane-crest-rot3.png}
  \caption{tAB-3c-tolane-rot3}
  \label{fig:tAB-3c-tolane-crest-rot3}
\end{subfigure}%
% \begin{subfigure}{.33\textwidth}
%   \centering
%   \includegraphics[width=.9\linewidth]{Figures/Conformer/cAB-3c-tolane-crest-rot9.png}
%   \caption{cAB-3c-tolane-rot9}
%   \label{fig:cAB-3c-tolane-crest-rot9}
% \end{subfigure}
\begin{subfigure}{.33\textwidth}
  \centering
  \includegraphics[width=.85\linewidth]{Figures/Conformer/cAB-for-tAB-3c-tolane-crest-rot3.png}
  \caption{cAB-3c-tolane-tor}
  \label{fig:cAB-3c-tolane-tor}
\end{subfigure}
\caption{Optimized geometries of different AB-tolane molecules.}
\label{fig:Conformers}
\end{figure}
%
%
% 
% In addition to the 4 geometries obtained via CREST, we used the energetically lowest \textit{trans} conformer as a template to construct the \textit{cis} isomer which would be the likely endpoint for a \textit{trans-cis} isomerisation. 
% This was accomplished by reducing the CNNC dihedral angle in a certain direction which would result in a \textit{cis} conformer resembling one of the energetically lowest \textit{cis} conformers found via sampling with CREST. \\
% As shown in table \ref{tab:GS-energy-conformers} the energetically lowest conformers were those in \textit{trans} configuration. 
% This is likely due to lower Pauli repulsion caused by larger distances between atoms especially at the azobenzene fragment. 
% Another interesting observation was the staggered stacked conformation of tAB-3c-tolane-rot3 (figure \ref{fig:Conformers} \textbf{(d)}), which will be discussed further in section \ref{sec:thermodyn-analysis}.
%
% maybe write something about the difference between this cis without staggered stacked and the trans with staggered stacked
%
\begin{comment}
%This was accomplished by reducing the CNNC angle in steps of \SI{20}{\degree} and optimising the geometry until a cis conformer was reached which are geometrically similar 
Since the trans configuration is in both cases, tAB-1c-tolane and tAB-3c-tolane the lowest in energy due to Pauli repulsion
those were taken as starting points for the modelling of geometries at the isomerisation end point
- tolan twist with CNNC dihedral angle almost 0 favored to non twisted tolan and larger CNNC angle
could be due to the backbown too (1c toalne)
- trans lower in energy than cis, mainly due to Pauli repulsion (wikipedia)?
- 
\end{comment}
\begin{comment}
To determine the conformer with the lowest ground state energy, we had to to sample through a large number of different conformers. To automate this procedure the program CREST was used. The resulting ten lowest rotamers of the corresponding conformers were then optimized with DFT CAM-B3LYP D3\_BJ and the resulting GS energy is listed in Table (TABLE REF).
\end{comment}
%
\begin{comment}
\begin{table}[H]
\caption{Total and relative energies of the the 10 conformers of cAB-1c-tolan. They are sorted by their respective population at room temperature.}
\centering
\begin{tabular}{@{}lllll@{}}
\toprule
molecule name            & Total energy (eV) & E\_rel (eV) & Pop   & Rank \\ \midrule
cAB-1c-tolan-crest-rot7  & -32328.584        & 0.000       & 0.142 & 1    \\
cAB-1c-tolan-crest-rot5  & -32328.582        & 0.002       & 0.132 & 2    \\
cAB-1c-tolan-crest-rot6  & -32328.582        & 0.002       & 0.130 & 3    \\
cAB-1c-tolan-crest-rot8  & -32328.579        & 0.005       & 0.117 & 4    \\
cAB-1c-tolan-crest-rot2  & -32328.572        & 0.012       & 0.087 & 5    \\
cAB-1c-tolan-crest-rot1  & -32328.571        & 0.014       & 0.083 & 6    \\
cAB-1c-tolan-crest-rot4  & -32328.570        & 0.015       & 0.080 & 7    \\
cAB-1c-tolan-crest-rot10 & -32328.569        & 0.015       & 0.078 & 8    \\
cAB-1c-tolan-crest-rot3  & -32328.569        & 0.015       & 0.078 & 9    \\
cAB-1c-tolan-crest-rot9  & -32328.567        & 0.017       & 0.072 & 10   \\ \bottomrule
\end{tabular}
\end{table}
\end{comment}
%
%
\section{Excited state analysis of optimized geometries}
To further characterize the molecules we calculated their lowest lying singlet excited states and visualized the molecular orbitals.
% Table \ref{tab:excited_states_tAB_1c_tolan} shows the six energetically lowest excited states of tAB-1c-tolan-rot9, those states which contribute significantly to the resulting UV-Spectrum were marked in bold.
% To gain further insight in the differences between the linked molecules compared t
%
\subsection{tAB-1c-tolan-rot9}
%
Table \ref{tab:excited_states_tAB_1c_tolan} shows the excitation wavelength, excitation energy and the oscillator strength of the first six transitions of tAB-1c-tolane-rot9.
To investigate the individuality of the AB and the tolane substructures which compose the linked molecules, the spectra of tAB and tolane (red) and the linked molecule tAB-1c-tolane-rot9 (black) were plotted (figure \ref{fig:UV_spec_tAB-1c-tolan}).
%
%
\begin{figure}[H]
    \centering
    \includegraphics[scale=0.75]{Figures/Conformer_UV_spectra_ink/tAB-1c-tolan-crest-rot9-corrected-abs-mod.eps}
    \caption{UV spectra of tAB-1c-tolan-rot9 (black) and a combination of tAB and tolane (red). The bars signify the strength and wavelength of the individual transitions. The numbers above certain bands correspond to the respective transition. 'a' indicates tAB transitions and 't' transitions of the tolane molecule.}
    \label{fig:UV_spec_tAB-1c-tolan}
\end{figure}
%
\begin{table}[ht]
  \caption{Type of transition, excitation Wavelength $\lambda$ in nm, vertical excitation energies $E$ in eV and their respective oscillator strength $f$ of the six energetically lowest excited states of tAB-1c-tolan-rot9. $\mathrm{AB_{ph}}$ refers to orbitals centered at the phenyl groups of AB.}
  \label{tab:excited_states_tAB_1c_tolan}
  \vspace{0.1 cm}
  \centering
  \begin{tabular}{ccccc}
  \toprule
  State & Trans. & $\lambda$ (nm)     & $E$ $\left(\si{\eV}\right)$          & $f$                 \\ \midrule
  S1 & n(AB)$\pi^{*}$(AB)  & 440   & 2.819 & 0.02                \\
  \textbf{S2}    & $\pi\mathrm{(AB)}\hspace{1 pt} \pi^{*}\mathrm{(AB)}$   & \textbf{286} & \textbf{4.341} & \textbf{0.38} \\
  S3    & $\pi\mathrm{(AB_{ph})}\hspace{1 pt} \pi^{*}\mathrm{(AB)}$& 267 & 4.647 & 0.02                \\
  S4    & $\pi\mathrm{(AB_{ph})} \hspace{1 pt} \pi^{*}\mathrm{(AB)}$& 263 & 4.708 & 0.11                \\
  S5    & $\pi\mathrm{(AB_{ph})} \hspace{1 pt} \pi^{*}\mathrm{(AB)}$& 260 & 4.776 & 0.02                \\
  \textbf{S6} & $\pi\mathrm{(tol)} \hspace{1 pt} \pi^{*}\mathrm{(tol)}$& \textbf{257} & \textbf{4.829} & \textbf{0.98}       \\ \bottomrule
  \end{tabular}
  \end{table}
%
The first singlet excited state (S1) of tAB-1c-tolan-crest-rot9 can be assigned to the n$\pi^{*}$-transition as shown by the respective attachment and detachment densities shown in figure \ref{fig:att-det:tAB-1c-tolane-crest-rot9}. 
% The canonical orbitals contributing to this transition are the HOMO-1, HOMO-2 and LUMO, all of them mainly localized at the AB unit.
In comparison with the S1 transition of tAB there is almost no change in the excitation energies ($\SI{2.819}{\eV}$ vs $\SI{2.824}{\eV}$ for tAB), but a major increase in oscillator strength ($0.02$ vs $\SI{5.4E-5}{}$ for tAB) which remains low due to the symmetry forbidden nature of the transition. 
Overall it appears the influence of the linked tolane is rather minor and the transition is very similar to the S1 transition seen in tAB. \\
\\
Similarly to tAB the second singlet excited state (S2) is a $\pi \pi^{*}$-transition which is slightly red shifted in comparison to tAB ($\SI{4.341}{\eV}$, $\SI{4.372}{\eV}$ for tAB) but possesses only a minor difference in their oscillator strength. \\
%The main canonical orbital contributing to this transition is the HOMO-1 which shape represents $\pi$-orbital centered on AB. \\
\\
The singlet excited states S3 to S5 seem to be $\pi\pi^*$-transitions originating from one of the $\pi$-orbitals of the AB phenyl groups with minor contributions from $\pi$-orbitals of the tolane fragment, 
while the $\pi^*$-orbitals are located on the AB unit. \\
% originate from one of the $\pi$-orbitals of the AB phenyl groups and some minor contributions from the $\pi$-orbital located at the tolane unit and 
% All three of them originate from one of the $\pi$-orbitals of the AB centered phenyl groups with some contributions from $\pi$-orbitals of the tolane-fragments and terminate at the $\pi^{*}$-orbitals of AB and \\
% maybe argue that in the NTO picture they look like charge transfer transitions.
\\
The sixth singlet excited state (S6) is a $\pi\pi^{*}$-transition entirely located on the tolane unit. Besides a minor blue shift (\SI{4.829}{\eV},  \SI{4,706}{\eV} for tolane) this transition mirrors the S1 transition of tolane (figure \ref{fig:nto_tolane}) almost identically.\\
% The next transition that contributes majorly to the UV-spectrum of tAB-1c-tolan-crest-rot9 is the sixth excited state (6) which is a $\pi \pi^{*}$-transition from the HOMO into the LUMO+1 which are both localized entirely on the tolane subunit of the molecule (figure \ref{fig:MO_tAB-1c-tolane-crest-rot9.}). 
% Besides a minor blue shift (\SI{4.829}{\eV},  \SI{4,706}{\eV} for tolane) this transition mirrors the S2 transition of tolane almost identically. 
% The shift is caused by an occupied MO lower in energy and a virtual MO higher in energy of tAB-1c-tolane-rot9 compared to tolane. 
% This is quite suprising since one would assume that larger dihedral angle of tAB-1c-tolane-rot9 would lead to a higher energy of the occupied MO and not a lower one. 
%
% The reason for this deviation in MO energy could be the increase
%
%which could be due to the larger diheral angle of tolane-substructure in tAB-1c-tolane-rot9.\\
\\
Overall it appears that the excited states of tAB-1c-tolan-crest-rot9 are very similar to those found in their fragments (tAB and tolane) and the linkage didn't lead to new states which contribute in a major way to the resulting UV-spectrum.
%
\begin{figure}[H]
  \centering
  \begin{subfigure}{.25\textwidth}
    \centering
    \includegraphics[width=.9\linewidth]{Figures/Appendix/tAB-1c-tolane-crest-rot9/tAB-1c-tolane-s1-att.png}
    %\caption{S1}
    % \label{fig:}
  \end{subfigure}%
  % \medskip
  \begin{subfigure}{.25\textwidth}
    \centering
    \includegraphics[width=.9\linewidth]{Figures/Appendix/tAB-1c-tolane-crest-rot9/tAB-1c-tolane-s2-att.png}
    %\caption{S2}
    % \label{fig:}
  \end{subfigure}%
  %
  \begin{subfigure}{.25\textwidth}
    \centering
    \includegraphics[width=.9\linewidth]{Figures/Appendix/tAB-1c-tolane-crest-rot9/tAB-1c-tolane-s3-att.png}
    %\caption{S3}
    % \label{fig:}
  \end{subfigure}
  %
  \vskip\baselineskip
  %
  \begin{subfigure}{.25\textwidth}
    \centering
    \includegraphics[width=.9\linewidth]{Figures/Appendix/tAB-1c-tolane-crest-rot9/tAB-1c-tolane-s1-det.png}
    \caption{S1}
    % \label{fig:}
  \end{subfigure}%
  % \medskip
  \begin{subfigure}{.25\textwidth}
    \centering
    \includegraphics[width=.9\linewidth]{Figures/Appendix/tAB-1c-tolane-crest-rot9/tAB-1c-tolane-s2-det.png}
    \caption{S2}
    % \label{fig:}
  \end{subfigure}%
  %
  \begin{subfigure}{.25\textwidth}
    \centering
    \includegraphics[width=.9\linewidth]{Figures/Appendix/tAB-1c-tolane-crest-rot9/tAB-1c-tolane-s3-det.png}
    \caption{S3}
    % \label{fig:}
  \end{subfigure}
  %
  %
  \vskip\baselineskip
  \vspace{1 cm}
  %
  %
  \begin{subfigure}{.25\textwidth}
    \centering
    \includegraphics[width=.9\linewidth]{Figures/Appendix/tAB-1c-tolane-crest-rot9/tAB-1c-tolane-s4-att.png}
    %\caption{S1}
    % \label{fig:}
  \end{subfigure}%
  % \medskip
  \begin{subfigure}{.25\textwidth}
    \centering
    \includegraphics[width=.9\linewidth]{Figures/Appendix/tAB-1c-tolane-crest-rot9/tAB-1c-tolane-s5-att.png}
    %\caption{S2}
    % \label{fig:}
  \end{subfigure}%
  %
  \begin{subfigure}{.25\textwidth}
    \centering
    \includegraphics[width=.9\linewidth]{Figures/Appendix/tAB-1c-tolane-crest-rot9/tAB-1c-tolane-s6-att.png}
    %\caption{S3}
    % \label{fig:}
  \end{subfigure}
  %
  \vskip\baselineskip
  %
  \begin{subfigure}{.25\textwidth}
    \centering
    \includegraphics[width=.9\linewidth]{Figures/Appendix/tAB-1c-tolane-crest-rot9/tAB-1c-tolane-s4-det.png}
    \caption{S4}
    % \label{fig:}
  \end{subfigure}%
  % \medskip
  \begin{subfigure}{.25\textwidth}
    \centering
    \includegraphics[width=.9\linewidth]{Figures/Appendix/tAB-1c-tolane-crest-rot9/tAB-1c-tolane-s5-det.png}
    \caption{S5}
    % \label{fig:}
  \end{subfigure}%
  %
  \begin{subfigure}{.25\textwidth}
    \centering
    \includegraphics[width=.9\linewidth]{Figures/Appendix/tAB-1c-tolane-crest-rot9/tAB-1c-tolane-s6-det.png}
    \caption{S6}
    % \label{fig:}
  \end{subfigure}
  %
  \caption{Detachment (red) and attachment (blue) densities at an isovalue of 0.02 for the energetically lowest four excited singlet states S1 to S4 of tAB-1c-tolane-crest-rot9 at the CAM-B3LYP/6-311G* level of theory}
  \label{fig:att-det:tAB-1c-tolane-crest-rot9}
  \end{figure}
%
%
% Canonical MOs of tAB-1c-tolane-crest-rot9
% \begin{figure}[h]
% \centering
% \begin{subfigure}{.25\textwidth}
%   \centering
%   \includegraphics[width=.9\linewidth]{Figures/Conformer_MO_diagrams/tAB-1c-tolane/tAB-1c-tolane-95.png}
%   \caption{HOMO-6}
%   % \label{fig:sub1}
% \end{subfigure}%
% % \medskip
% \begin{subfigure}{.25\textwidth}
%   \centering
%   \includegraphics[width=.9\linewidth]{Figures/Conformer_MO_diagrams/tAB-1c-tolane/tAB-1c-tolane-98.png}
%   \caption{HOMO-3}
%   % \label{fig:sub1}
% \end{subfigure}%
% %
% \begin{subfigure}{.25\textwidth}
%   \centering
%   \includegraphics[width=.9\linewidth]{Figures/Conformer_MO_diagrams/tAB-1c-tolane/tAB-1c-tolane-99.png}
%   \caption{HOMO-2}
%   % \label{fig:sub1}
% \end{subfigure}%
% %
% \begin{subfigure}{.25\textwidth}
%   \centering
%   \includegraphics[width=1\linewidth]{Figures/Conformer_MO_diagrams/tAB-1c-tolane/tAB-1c-tolane-100.png}
%   \caption{HOMO-1}
%   % \label{fig:sub1}
% \end{subfigure}%
% %
% %\caption{A figure with two subfigures}
% %\label{fig:test}
% \vskip\baselineskip
% \begin{subfigure}{.25\textwidth}
%   \centering
%   \includegraphics[width=.95\linewidth]{Figures/Conformer_MO_diagrams/tAB-1c-tolane/tAB-1c-tolane-101.png}
%   \caption{HOMO}
%   % \label{fig:sub1}
% \end{subfigure}%
% %
% \begin{subfigure}{.25\textwidth}
%   \centering
%   \includegraphics[width=.95\linewidth]{Figures/Conformer_MO_diagrams/tAB-1c-tolane/tAB-1c-tolane-102.png}
%   \caption{LUMO}
%   % \label{fig:sub1}
% \end{subfigure}%
% %
% \begin{subfigure}{.25\textwidth}
%   \centering
%   \includegraphics[width=.95\linewidth]{Figures/Conformer_MO_diagrams/tAB-1c-tolane/tAB-1c-tolane-103.png}
%   \caption{LUMO+1}
%   % \label{fig:sub1}
% \end{subfigure}%
% %
% \caption{Canonical orbitals of tAB-1c-tolane-crest-rot9, isovalue = 0.1.}
% \label{fig:MO_tAB-1c-tolane-crest-rot9.}
% \end{figure}

\subsection{tAB-3c-tolane-rot3}
%
%
\begin{figure}[H]
    \centering
    \includegraphics[scale=0.75]{Figures/Conformer_UV_spectra_ink/tAB-3c-tolan-crest-rot3-corrected-abs-mod.eps}
    \caption{UV spectra of tAB-3c-tolan-crest-rot3 (black) and a combination of tAB and tolane (red). 
    The bars signify the strength energy of the individual transitions. 
    The numbers above certain bands correspond to the respective transition.
     ``a'' indicates tAB transitions and ``t'' transitions of the tolane molecule.}
    \label{fig:UV_spec_tAB-3c-tolan}
\end{figure}
%
%
%
\begin{table}[b]
  \caption{Type of transition, excitation wavelength $\lambda$ in nm, vertical excitation energies $E$ in eV and their respective oscillator strength $f$ of the six energetically lowest excited states of tAB-3c-tolan-rot3. 
  The subscript ``ph'' refers to orbitals centered at phenyl group.}
  \label{tab:excited_states_tAB_3c_tolan}
  \vspace{0.1 cm}
  \centering
  \begin{tabular}{ccccc}
  \toprule
  State  & Trans. & $\lambda$ (nm)  & $E$ (eV)               & $f$                \\ \midrule
  S1     & n$\mathrm{(AB)}\hspace{1 pt} \pi^{*}\mathrm{(AB)}$ & 437             & 2.839                  & \SI{3.57e-3}{}     \\
  S2     & $\pi\mathrm{(tol)}\hspace{1 pt} \pi^{*}\mathrm{(AB)}$ & 300             & 4.120                  & \SI{8.13e-3}{}     \\
  \textbf{S3}     & $\pi\mathrm{(AB)}\hspace{1 pt} \pi^{*}\mathrm{(AB)}$ & \textbf{295}    & \textbf{4.208}         & \textbf{0.20}      \\
  S4     & $\pi\mathrm{(AB_{ph}/tol_{ph})}\hspace{1 pt} \pi^{*}\mathrm{(AB)}$ & 274    & 4.519                  & 0.07               \\
  S5     & $\pi\mathrm{(AB_{ph}/tol_{ph})}\hspace{1 pt} \pi^{*}\mathrm{(AB)}$ & 271    & 4.568                  & 0.02               \\
  \textbf{S6}     & $\pi\mathrm{(tol)}\hspace{1 pt} \pi^{*}\mathrm{(tol)}$ & \textbf{263}    & \textbf{4.709}         & \textbf{1.48}      \\ \bottomrule
  \end{tabular}
  \end{table}
%
%
% In contrast to tAB-1c-tolane, tAB-3c-tolane shows major deviations between the 
%Despite possessing a completely different geometry than tAB-1c-tolane-rot9 due to $\pi$-$\pi$ interactions leading to staggered stacked orientation of the benzene rings,
%the first excited state doesn't differ much between tAB, tAB-1c-tolane and tAB-3c-tolane.
The first singlet excited state (S1) of tAB-3c-tolane-rot3 (table \ref{tab:excited_states_tAB_3c_tolan}) can be assigned to  a n$\pi^*$ transition (figure \ref{fig:att-det:tAB-3c-tolane}) which is slightly higher in energy compared to the S1 transition in tAB ($\SI{2.839}{\eV}$ vs $\SI{2.824}{\eV}$ for tAB) but much lower in intensity ($\SI{3.57E-3}{}$ vs $\SI{5.4E-5}{}$ for tAB).
The small difference between S1 of tAB-3c-tolane-rot3, tAB-1c-tolane-rot9 and tAB, which possess quite different geometries, is likely due to the local nature of the n$\pi^*$ transition.\\
% S1 is a n$\pi^{*}$ transition mainly between the HOMO-4 and LUMO, both MOs which are centered around the AB substructure. 
% The transition is slightly energetically higher compared to the S1 transition in tAB ($\SI{2.839}{\eV}$ vs $\SI{2.824}{\eV}$ for tAB) but much lower in intensity ($\SI{3.57E-3}{}$ vs $\SI{5.4E-5}{}$ for tAB). \\ 
\\
The second singlet excited state (S2) resembles a $\pi\pi^*$ charge-transfer transition, since its detachment densities are located on the tolane substructure,
while its attachment densities are located on the AB unit. 
This excited state seems to be facilitated by the small intramolecular distance between the two $\pi$-systems.\\ 
% The intensity of S2 is quite small and is only caused by the transition between the HOMO and LUMO. Considering that the HOMO is a $\pi$-orbital localized on the tolane substructure while the LUMO is a $\pi^{*}$-orbital of tAB one could argue that this resembles a charge-transfer transition. Due to the absence of this transition on the tAB and tAB-1c-rot9 spectrum, one could conclude that this transition is due to the staggered stacked conformation, resulting in a smaller distance between the two $\pi$-systems. \\
\\
The third singlet excited state (S3) can be assigned to a $\pi\pi^*$ transition with detachment densities at one of ABs phenyl rings and the acetylene unit and attachment densities at the azo group.
% It resembles the second singlet excited state of tAB-1c-tolane-rot9, while being slightly red shifted (\SI[]{4.21}[]{\electron\volt} \SI[]{4.34}[]{\electron\volt} for tAB-1c-tolane-rot9)
Compared to S2 of tAB, S3 is slightly red-shifted (\SI[]{4.21}{\electronvolt}, \SI[]{4.37}{\electronvolt} for tAB) and lacks in intensity (0.20, 1.04 for tAB). 
One possible cause could be the the reduced planarity which could lead to a reduced overlap of the orbitals and therefor a reduced transition probability.\\
\\
The sixth singlet excited state (S6), a $\pi\pi^*$ transition, is similar to the first transition of tolane, albeit with a higher intensity (1.48, 1.12 for tolane).
The attachment and detachment densities are both on the tolane substructure of the molecule. 
%
%
% The next transition that contributes in a major way to the resulting UV spectrum is S3. 
% Looking at the participating MOs (HOMO-1, HOMO-2 and LUMO) one can see that it resembles the S2 transition in tAB-1c-tolane which is also a $\pi\pi^{*}$ transition localized on the tAB molecule.
% Compared to the S2 transition in tAB it lacks its intensity and is slightly red-shifted. 
% One reason for this difference could be the increased dihededral angle between to the two benzene rings for the AB unit (\SI{25}{\degree} vs \SI{2}{\degree} for tAB) which affects the AB centered MOs much more than those localized on the tolane fragment. 
% Another reason could be $\pi$-$\pi$ interactions due to the staggered stacked orientation of the benzene rings between the AB and tolane substructures.  \\
%
%
% The $\pi\pi^{*}$ transition corresponding to S6 shows only a negligible deviation in energy but a stark increase in intensity compared to the S1 transition of tolane.
% The contributing orbitals to this transition are mainly the HOMO and LUMO+1 which are both centered on the tolane substructure of the molecule. 
% Despite possessing a similar dihedral distortion between both benzene units of tolane as the AB substructure, this transition seems largely unaffected by it. 
% Both the occupied and unoccupied MOs of this transition are similarly shifted to higher energies than the respective MOs of the tolane molecule. 
% Since the change in the dihedral angle affects both MOs the same way there is no change in transition-energy observed.
% and leading to to 
%
%
%
\begin{figure}[h]
  \centering
  \begin{subfigure}{.25\textwidth}
    \centering
    \includegraphics[width=.9\linewidth]{Figures/Appendix/tAB-3c-tolane/tAB-3c-tolane-s1-att.png}
    %\caption{S1}
    % \label{fig:}
  \end{subfigure}%
  % \medskip
  \begin{subfigure}{.25\textwidth}
    \centering
    \includegraphics[width=.9\linewidth]{Figures/Appendix/tAB-3c-tolane/tAB-3c-tolane-s2-att.png}
    %\caption{S2}
    % \label{fig:}
  \end{subfigure}%
  %
  \begin{subfigure}{.25\textwidth}
    \centering
    \includegraphics[width=.9\linewidth]{Figures/Appendix/tAB-3c-tolane/tAB-3c-tolane-s3-att.png}
    %\caption{S3}
    % \label{fig:}
  \end{subfigure}
  %
  \vskip\baselineskip
  %
  \begin{subfigure}{.25\textwidth}
    \centering
    \includegraphics[width=.9\linewidth]{Figures/Appendix/tAB-3c-tolane/tAB-3c-tolane-s1-det.png}
    \caption{S1}
    % \label{fig:}
  \end{subfigure}%
  % \medskip
  \begin{subfigure}{.25\textwidth}
    \centering
    \includegraphics[width=.9\linewidth]{Figures/Appendix/tAB-3c-tolane/tAB-3c-tolane-s2-det.png}
    \caption{S2}
    % \label{fig:}
  \end{subfigure}%
  %
  \begin{subfigure}{.25\textwidth}
    \centering
    \includegraphics[width=.9\linewidth]{Figures/Appendix/tAB-3c-tolane/tAB-3c-tolane-s3-det.png}
    \caption{S3}
    % \label{fig:}
  \end{subfigure}
  %
  %
  \vskip\baselineskip
  \vspace{1 cm}
  %
  %
  \begin{subfigure}{.25\textwidth}
    \centering
    \includegraphics[width=.9\linewidth]{Figures/Appendix/tAB-3c-tolane/tAB-3c-tolane-s4-att.png}
    %\caption{S1}
    % \label{fig:}
  \end{subfigure}%
  % \medskip
  \begin{subfigure}{.25\textwidth}
    \centering
    \includegraphics[width=.9\linewidth]{Figures/Appendix/tAB-3c-tolane/tAB-3c-tolane-s5-att.png}
    %\caption{S2}
    % \label{fig:}
  \end{subfigure}%
  %
  \begin{subfigure}{.25\textwidth}
    \centering
    \includegraphics[width=.9\linewidth]{Figures/Appendix/tAB-3c-tolane/tAB-3c-tolane-s6-att.png}
    %\caption{S3}
    % \label{fig:}
  \end{subfigure}
  %
  \vskip\baselineskip
  %
  \begin{subfigure}{.25\textwidth}
    \centering
    \includegraphics[width=.9\linewidth]{Figures/Appendix/tAB-3c-tolane/tAB-3c-tolane-s4-det.png}
    \caption{S4}
    % \label{fig:}
  \end{subfigure}%
  % \medskip
  \begin{subfigure}{.25\textwidth}
    \centering
    \includegraphics[width=.9\linewidth]{Figures/Appendix/tAB-3c-tolane/tAB-3c-tolane-s5-det.png}
    \caption{S5}
    % \label{fig:}
  \end{subfigure}%
  %
  \begin{subfigure}{.25\textwidth}
    \centering
    \includegraphics[width=.9\linewidth]{Figures/Appendix/tAB-3c-tolane/tAB-3c-tolane-s6-det.png}
    \caption{S6}
    % \label{fig:}
  \end{subfigure}
  %
  \caption{Detachment (red) and attachment (blue) densities with an isovalue of 0.02 for the energetically lowest six excited singlet states S1 to S6 of tAB-3c-tolane-crest-rot3 at the CAM-B3LYP/6-311G* level of theory.}
  \label{fig:att-det:tAB-3c-tolane}
  \end{figure}



% As seen in figure \ref{fig:UV_spec_tAB-3c-tolan} the S1 n$\pi^{*}$ transition of tAB-3c-tolan-crest-rot3 matches the one seen in tAB quite well, albeit with a higher energy ($\SI{2.839}{\eV}$ vs $\SI{2.824}{\eV}$ for tAB) and intensity ($\SI{3.57E-3}{}$ vs $0.02$ for tAB). The can 




% Canonical MOs of tAB-3c-tolane-crest-rot9
% \begin{figure}[H]
% \centering
% \begin{subfigure}{0.5\textwidth}
%   \centering
%   \includegraphics[width=1\linewidth]{Figures/Conformer_MO_diagrams/tAB-3c-tolane/tAB-3c-tolane-113.png}
%   \caption{HOMO-4}
%   % \label{fig:sub1}
% \end{subfigure}%
% % \medskip
% \vskip\baselineskip
% %
% \begin{subfigure}{.5\textwidth}
%   \centering
%   \includegraphics[width=1\linewidth]{Figures/Conformer_MO_diagrams/tAB-3c-tolane/tAB-3c-tolane-114.png}
%   \caption{HOMO-3}
%   % \label{fig:sub1}
% \end{subfigure}%
% %
% \begin{subfigure}{.5\textwidth}
%   \centering
%   \includegraphics[width=1\linewidth]{Figures/Conformer_MO_diagrams/tAB-3c-tolane/tAB-3c-tolane-115.png}
%   \caption{HOMO-2}
%   % \label{fig:sub1}
% \end{subfigure}%
% %
% \vskip\baselineskip
% %
% \begin{subfigure}{.5\textwidth}
%   \centering
%   \includegraphics[width=1\linewidth]{Figures/Conformer_MO_diagrams/tAB-3c-tolane/tAB-3c-tolane-116.png}
%   \caption{HOMO-1}
%   % \label{fig:sub1}
% \end{subfigure}%
% %
% %\caption{A figure with two subfigures}
% %\label{fig:test}
% \begin{subfigure}{.5\textwidth}
%   \centering
%   \includegraphics[width=1\linewidth]{Figures/Conformer_MO_diagrams/tAB-3c-tolane/tAB-3c-tolane-117.png}
%   \caption{HOMO}
%   % \label{fig:sub1}
% \end{subfigure}%
% %
% \vskip\baselineskip
% %
% \begin{subfigure}{.5\textwidth}
%   \centering
%   \includegraphics[width=1\linewidth]{Figures/Conformer_MO_diagrams/tAB-3c-tolane/tAB-3c-tolane-118.png}
%   \caption{LUMO}
%   % \label{fig:sub1}
% \end{subfigure}%
% %
% \begin{subfigure}{.5\textwidth}
%   \centering
%   \includegraphics[width=1\linewidth]{Figures/Conformer_MO_diagrams/tAB-3c-tolane/tAB-3c-tolane-119.png}
%   \caption{LUMO+1}
%   % \label{fig:sub1}
% \end{subfigure}%
% %
% \caption{Canonical orbitals of tAB-3c-tolane-crest-rot3, iso-value = 0.1.}
% \label{fig:MO_tAB-3c-tolane-crest-rot3.}
% \end{figure}


%%%%%%%%%%%%%%%%%%%%% Idee %%%%%%%%%%%%%%%%%%%%%%%%%
% Alle Tabellen in Anang und nur die GS energies der finalen 4 + cAB for tAB in results
% Jetzt nur noch die Spektren und dann kurz was dazu schreiben

% \begin{figure}[h]
% \centering
% \begin{subfigure}{.5\textwidth}
%   \centering
%   \includegraphics[width=1\linewidth]{Figures/Conformer_UV_spectra/tAB-1c-tolan-crest-rot9.eps}
%   \caption{A subfigure}
%   \label{fig:sub1}
% \end{subfigure}%
% \medskip
% \begin{subfigure}{.5\textwidth}
%   \centering
%   \includegraphics[width=1\linewidth]{Figures/Conformer_UV_spectra/cAB-for-tAB-1c-tolan-crest-rot9.eps}
%   \caption{A subfigure}
%   \label{fig:sub2}
% \end{subfigure}

\subsection{cAB-1c-tolane-tor}
%
%
%
%
%
%
\begin{figure}[H]
    \centering
    \includegraphics[scale=0.75]{Figures/Conformer_UV_spectra_ink/cAB-1c-tolan-abs_mod-tor.eps}
    \caption{UV spectra of cAB-1c-tolan-tor (black) and a combination of cAB and tolane (red). The bars signify the strength and wavelength of the individual transitions.}
    \label{fig:UV_spec_cAB-for-tAB-1c-tolan}
\end{figure}
%
%
\begin{table}[b]
  \caption{Type of transition, excitation wavelength $\lambda$ in nm, vertical excitation energies $E$ in eV and their respective oscillator strength $f$ of the six energetically lowest excited states of cAB-1c-tolan-tor. The subscript ``ph'' refers to orbitals centered at the phenyl groups.}
  \label{tab:excited_states_cAB_1c_tolane}
  \vspace{0.1 cm}
  \centering
  \begin{tabular}{ccccc}
  \toprule
  State & Trans. & $\lambda$ (nm)  & $E$ (eV)               & $f$                 \\ 
  \midrule
  S1    & n$\mathrm{(AB)}\hspace{1 pt} \pi^{*}\mathrm{(AB)}$& 427             & 2.901                  & 0.01                \\
  \textbf{S2}    & $\pi\mathrm{(tol)}\hspace{1 pt} \pi^{*}\mathrm{(tol)}$& \textbf{271}    & \textbf{4.577}         & \textbf{0.37}              \\
  \textbf{S3}    & $\pi\mathrm{(AB)}\hspace{1 pt} \pi^{*}\mathrm{(AB)}$& \textbf{263}             & \textbf{4.708}                  & \textbf{0.32}                \\
  S4    & $\pi\mathrm{(AB)}\hspace{1 pt} \pi^{*}\mathrm{(AB)}$& 249             & 4.980                  & 0.10                \\
  S5    & $\pi\mathrm{(tol)}\hspace{1 pt} \pi^{*}\mathrm{(tol)}$& 243             & 5.093                  & \SI{4.50e-3}{}      \\
  S6    & $\pi\mathrm{(tol)}\hspace{1 pt} \pi^{*}\mathrm{(tol)}$& 242    & 5.121         & \SI{4.02e-3}{}      \\
  \bottomrule
  \end{tabular}
  \end{table}
%
%
\begin{comment}
Table \ref{tab:excited_states_cAB_1c_tolane} contains the wavelength, energy and oscillator strength of the six energetically lowest excited states of cAB-1c-tolane which are plotted in figure \ref{fig:UV_spec_cAB-for-tAB-1c-tolan} in relation to a combined UV spectrum of cAB and tolane. \\
\\  
\end{comment}
%
The first singlet excited state (S1) of cAB-1c-tolane-tor can be assigned to tbe n$\pi^*$ transition, located on the AB substructure.
It is quite similar in terms of energy ($\SI{2.901}{\eV}$ vs $\SI{2.877}{\eV}$ for tAB) and intensity ($\SI{0.01}{}$ vs $0.03$ for cAB) to the first excited singlet state of cAB.
Both, attachment and detachment densities, are located on the azo group.\\
%
%
%
% Looking at the S1 state of cAB-1c-tolane one can hardly see a difference compared to the S1 of cAB (1a) in energy ($\SI{2.901}{\eV}$ vs $\SI{2.877}{\eV}$ for tAB) or intensity ($\SI{0.01}{}$ vs $0.03$ for cAB). 
% The MOs which are mainly involved in this transition are the HOMO-1 and LUMO showing the n$\pi^{*}$ transition quite well. \\
\\
The $\pi\pi^{*}$ S2 state can be compared to the first excited state of tolane, since the attachment and detachment densities are located on the tolane subunit.
It appears the linkage to cAB leads to a smaller intensity and stabilizes this specific state. \\
\\
The attachment and detachment densities of the third excited singlet state (S3) of cAB-1c-tolane, a $\pi\pi^{*}$ transition, are located on the AB subunit of the molecule.
This exited state is quite similar to the S3 of cAB, but possesses a greater intensity (0.32, 0.02 for cAB) and is shifted to a longer wavelength (\SI[]{4.71}{\electronvolt}, \SI[]{4.97}{\electronvolt} for cAB).\\
\\
% The third excited singlet state (S3) of cAB-1c-tolane, also a $\pi\pi^{*}$ transition, 
% is composed of transitions between lower lying HOMOs mainly HOMO-2 and HOMO-4 and the LUMO which are for the most part localized on the AB fragment consisting of the the $\pi$ orbitals of the benzene rings with some admixture of the n Orbitals of N2. 
% Especially after comparing the specific NTOs one can see that it closely resembles S3 (3a) of cAB albeit with an increase in intensity and a shift to a longer wavelength. \\
The fourth excited singlet state (S4) can be assigned to a $\pi\pi^*$ transition, which is located on the azo group and the adjacent phenyl group. 
The attachment and detachment densities of the excited singlet states S5 and S6 are both found on the tolane substructure of the molecule.\\
Compared to tAB-1c-tolane-rot9, one observes that the transitions of cAB-1c-tolane-tor centered around the AB substructure are higher in energy.
One possible cause could be the that the \textit{cis}-conformation leads to a less delocalized $\pi$ system.
% which is likely due to the cis-conformation leading to a less delocalized $\pi$ system. 
% 
% 
% The $\pi\pi^{*}$ transition leading to S4 includes HOMOs -6 till -1 which are contributed evenly around the aromatic system of the molecule and terminates mainly at the LUMO. 
% The NTO picture describes this AB centered transition more clearly. S5 and S6 are mainly tolane centered $\pi\pi^{*}$ transitions. \\
% Compared to tAB-1c-tolane-rot9, one observes that the transitions of cAB-1c-tolane-tor centered around the AB substructure are higher in energy which is likely due to the cis-conformation leading to a less delocalized $\pi$ system. 
% On the other hand it requires a lot less energy compared to tAB-1c-tolane-rot9.
%
%
%
%
%
\begin{comment}
\begin{figure}[H]
\centering
\begin{subfigure}{.33\textwidth}
  \centering
  \includegraphics[width=.9\linewidth]{Figures/Conformer_MO_diagrams/cAB-1c-tolane/cAB-1c-tolane-97.png}
  \caption{HOMO-4}
  % \label{fig:sub1}
\end{subfigure}%
% \medskip
\begin{subfigure}{.33\textwidth}
  \centering
  \includegraphics[width=.9\linewidth]{Figures/Conformer_MO_diagrams/cAB-1c-tolane/cAB-1c-tolane-98.png}
  \caption{HOMO-3}
  % \label{fig:sub1}
\end{subfigure}%
%
\begin{subfigure}{.33\textwidth}
  \centering
  \includegraphics[width=.9\linewidth]{Figures/Conformer_MO_diagrams/cAB-1c-tolane/cAB-1c-tolane-99.png}
  \caption{HOMO-2}
  % \label{fig:sub1}
\end{subfigure}%
%
\vskip\baselineskip
%
\begin{subfigure}{.3\textwidth}
  \centering
  \includegraphics[width=1\linewidth]{Figures/Conformer_MO_diagrams/cAB-1c-tolane/cAB-1c-tolane-100.png}
  \caption{HOMO-1}
  % \label{fig:sub1}
\end{subfigure}%
%
%\caption{A figure with two subfigures}
%\label{fig:test}
%
\begin{subfigure}{.32\textwidth}
  \centering
  \includegraphics[width=.95\linewidth]{Figures/Conformer_MO_diagrams/cAB-1c-tolane/cAB-1c-tolane-101.png}
  \caption{HOMO}
  % \label{fig:sub1}
\end{subfigure}%
%
\vskip\baselineskip
%
\begin{subfigure}{.32\textwidth}
  \centering
  \includegraphics[width=.95\linewidth]{Figures/Conformer_MO_diagrams/cAB-1c-tolane/cAB-1c-tolane-102.png}
  \caption{LUMO}
  % \label{fig:sub1}
\end{subfigure}%
%
\begin{subfigure}{.32\textwidth}
  \centering
  \includegraphics[width=.95\linewidth]{Figures/Conformer_MO_diagrams/cAB-1c-tolane/cAB-1c-tolane-103.png}
  \caption{LUMO+1}
  % \label{fig:sub1}
\end{subfigure}%
%
\caption{Canonical orbitals of cAB-1c-tolane, isovalue = 0.1.}
\label{fig:MO_cAB-1c-tolane.}
\end{figure}
\end{comment}
%
\begin{figure}[H]
  \centering
  \begin{subfigure}{.25\textwidth}
    \centering
    \includegraphics[width=.9\linewidth]{Figures/Appendix/cAB-1c-tolane/cAB-1c-tolane-s1-att.png}
    %\caption{S1}
    % \label{fig:}
  \end{subfigure}%
  % \medskip
  \begin{subfigure}{.25\textwidth}
    \centering
    \includegraphics[width=.9\linewidth]{Figures/Appendix/cAB-1c-tolane/cAB-1c-tolane-s2-att.png}
    %\caption{S2}
    % \label{fig:}
  \end{subfigure}%
  %
  \begin{subfigure}{.25\textwidth}
    \centering
    \includegraphics[width=.9\linewidth]{Figures/Appendix/cAB-1c-tolane/cAB-1c-tolane-s3-att.png}
    %\caption{S3}
    % \label{fig:}
  \end{subfigure}
  %
  \vskip\baselineskip
  %
  \begin{subfigure}{.25\textwidth}
    \centering
    \includegraphics[width=.9\linewidth]{Figures/Appendix/cAB-1c-tolane/cAB-1c-tolane-s1-det.png}
    \caption{S1}
    % \label{fig:}
  \end{subfigure}%
  % \medskip
  \begin{subfigure}{.25\textwidth}
    \centering
    \includegraphics[width=.9\linewidth]{Figures/Appendix/cAB-1c-tolane/cAB-1c-tolane-s2-det.png}
    \caption{S2}
    % \label{fig:}
  \end{subfigure}%
  %
  \begin{subfigure}{.25\textwidth}
    \centering
    \includegraphics[width=.9\linewidth]{Figures/Appendix/cAB-1c-tolane/cAB-1c-tolane-s3-det.png}
    \caption{S3}
    % \label{fig:}
  \end{subfigure}
  %
  %
  \vskip\baselineskip
  \vspace{1 cm}
  %
  %
  \begin{subfigure}{.25\textwidth}
    \centering
    \includegraphics[width=.9\linewidth]{Figures/Appendix/cAB-1c-tolane/cAB-1c-tolane-s4-att.png}
    %\caption{S1}
    % \label{fig:}
  \end{subfigure}%
  % \medskip
  \begin{subfigure}{.25\textwidth}
    \centering
    \includegraphics[width=.9\linewidth]{Figures/Appendix/cAB-1c-tolane/cAB-1c-tolane-s5-att.png}
    %\caption{S2}
    % \label{fig:}
  \end{subfigure}%
  %
  \begin{subfigure}{.25\textwidth}
    \centering
    \includegraphics[width=.9\linewidth]{Figures/Appendix/cAB-1c-tolane/cAB-1c-tolane-s6-att.png}
    % \caption{S3}
    % \label{fig:}
  \end{subfigure}
  %
  \vskip\baselineskip
  %
  \begin{subfigure}{.25\textwidth}
    \centering
    \includegraphics[width=.9\linewidth]{Figures/Appendix/cAB-1c-tolane/cAB-1c-tolane-s4-det.png}
    \caption{S4}
    % \label{fig:}
  \end{subfigure}%
  % \medskip
  \begin{subfigure}{.25\textwidth}
    \centering
    \includegraphics[width=.9\linewidth]{Figures/Appendix/cAB-1c-tolane/cAB-1c-tolane-s5-det.png}
    \caption{S5}
    % \label{fig:}
  \end{subfigure}%
  %
  \begin{subfigure}{.25\textwidth}
    \centering
    \includegraphics[width=.9\linewidth]{Figures/Appendix/cAB-1c-tolane/cAB-1c-tolane-s6-det.png}
    \caption{S6}
    % \label{fig:}
  \end{subfigure}
  %
  \caption{Detachment (red) and attachment (blue) densities at an isovalue of 0.02 for the energetically lowest four excited singlet states S1 to S4 of cAB-1c-tolane at the CAM-B3LYP/6-311G* level of theory.}
  \label{fig:att-det:cAB-1c-tolane-crest-rot9}
  \end{figure}

% \subsection{tAB-3c-tolane}

% \begin{table}[H]
% \caption{Vertical excitation energies in eV and their respective oscillator strength of the six energetically lowest excited states of tAB-3c-tolan-crest-rot3.}
% \label{tab:excited_states_tAB_1c_tolan}
% \vspace{0.3 cm}
% \centering
% \begin{tabular}{lll}
% \toprule
% State & Excitation energy (eV) & Oscillator strength \\ \midrule
% S1    & 2.839                  & \SI{3.57e-3}{}      \\
% S2    & 4.120                  & \SI{8.13e-3}{}     \\
% S3    & \textbf{4.208}         & \textbf{0.20}      \\
% S4    & 4.519                  & 0.07               \\
% S5    & 4.568                  & 0.02               \\
% S6    & \textbf{4.709}         & \textbf{1.48}      \\ \bottomrule
% \end{tabular}
% \end{table}

% \begin{figure}[H]
%     \centering
%     \includegraphics[scale=0.75]{Figures/Conformer_UV_spectra/tAB-3c-tolan-crest-rot3.eps}
%     \caption{UV spectra of tAB-3c-tolan-crest-rot3 (black) and a combination of tAB and tolane (red). The bars signify the strength energy of the individual transitions.}
%     \label{fig:UV_spec_tAB-3c-tolan}
% \end{figure}
% %
% As seen in figure \ref{fig:UV_spec_tAB-3c-tolan} the S1 n$\pi^{*}$ transition of tAB-3c-tolan-crest-rot3 matches the one seen in tAB quite well, albeit with a higher energy ($\SI{2.839}{\eV}$ vs $\SI{2.824}{\eV}$ for tAB) and intensity ($\SI{3.57E-5}{}$ vs $0.02$ for tAB). The can




% % Canonical MOs of tAB-3c-tolane-crest-rot9
% \begin{figure}[H]
% \centering
% \begin{subfigure}{0.5\textwidth}
%   \centering
%   \includegraphics[width=1\linewidth]{Figures/Conformer_MO_diagrams/tAB-3c-tolane/tAB-3c-tolane-113.png}
%   \caption{HOMO-4}
%   % \label{fig:sub1}
% \end{subfigure}%
% % \medskip
% \vskip\baselineskip
% %
% \begin{subfigure}{.5\textwidth}
%   \centering
%   \includegraphics[width=1\linewidth]{Figures/Conformer_MO_diagrams/tAB-3c-tolane/tAB-3c-tolane-114.png}
%   \caption{HOMO-3}
%   % \label{fig:sub1}
% \end{subfigure}%
% %
% \begin{subfigure}{.5\textwidth}
%   \centering
%   \includegraphics[width=1\linewidth]{Figures/Conformer_MO_diagrams/tAB-3c-tolane/tAB-3c-tolane-115.png}
%   \caption{HOMO-2}
%   % \label{fig:sub1}
% \end{subfigure}%
% %
% \vskip\baselineskip
% %
% \begin{subfigure}{.5\textwidth}
%   \centering
%   \includegraphics[width=1\linewidth]{Figures/Conformer_MO_diagrams/tAB-3c-tolane/tAB-3c-tolane-116.png}
%   \caption{HOMO-1}
%   % \label{fig:sub1}
% \end{subfigure}%
% %
% %\caption{A figure with two subfigures}
% %\label{fig:test}
% \begin{subfigure}{.5\textwidth}
%   \centering
%   \includegraphics[width=1\linewidth]{Figures/Conformer_MO_diagrams/tAB-3c-tolane/tAB-3c-tolane-117.png}
%   \caption{HOMO}
%   % \label{fig:sub1}
% \end{subfigure}%
% %
% \vskip\baselineskip
% %
% \begin{subfigure}{.5\textwidth}
%   \centering
%   \includegraphics[width=1\linewidth]{Figures/Conformer_MO_diagrams/tAB-3c-tolane/tAB-3c-tolane-118.png}
%   \caption{LUMO}
%   % \label{fig:sub1}
% \end{subfigure}%
% %
% \begin{subfigure}{.5\textwidth}
%   \centering
%   \includegraphics[width=1\linewidth]{Figures/Conformer_MO_diagrams/tAB-3c-tolane/tAB-3c-tolane-119.png}
%   \caption{LUMO+1}
%   % \label{fig:sub1}
% \end{subfigure}%
% %
% \caption{Canonical orbitals of tAB-3c-tolane-crest-rot3, iso-value = 0.1.}
% \label{fig:MO_tAB-3c-tolane-crest-rot3.}
% \end{figure}


\subsection{cAB-3c-tolane-tor}
%
%
%
%
\begin{figure}[H]
    \centering
    \includegraphics[scale=0.75]{Figures/Conformer_UV_spectra_ink/cAB-3c-tolan-abs_mod.eps}
    \caption{UV spectra of cAB-3c-tolane-tor (black) and a combination of cAB and tolane (red). The bars signify the strength energy of the individual transitions.}
    \label{fig:UV_spec_cAB-for-tAB-3c-tolan}
\end{figure}
%
%
%
\begin{table}[b]
  \caption{Type of transition, excitation wavelength $\lambda$ in nm, vertical excitation energies $E$ in eV and their respective oscillator strength $f$ of the six energetically lowest excited states of cAB-3c-tolane-tor. The subscript ``ph'' refers to orbitals centered at the phenyl group.}
  \label{tab:excited_states_cAB_3c_tolane}
  \vspace{0.1 cm}
  \centering
  \begin{tabular}{ccccc}
  \toprule
  State & Trans.&$\lambda$ (nm)  & $E$ (eV)               & $f$                 \\ 
  \midrule
  S1    & n$\mathrm{(AB)}\hspace{1 pt} \pi^{*}\mathrm{(AB)}$ & 409             & 3.032                  & 0.012               \\
  \textbf{S2}    &$\pi\mathrm{(tol)}\hspace{1 pt} \pi^{*}\mathrm{(tol)}$ & \textbf{271}    & \textbf{4.583}         & \textbf{0.720}      \\
  S3    &$\pi\mathrm{(AB)}\hspace{1 pt} \pi^{*}\mathrm{(AB)}$ & 257             & 4.822                  & 0.019               \\
  \textbf{S4}    &$\pi\mathrm{(AB)}\hspace{1 pt} \pi^{*}\mathrm{(AB)}$ & \textbf{250}    & \textbf{4.968}         & \textbf{0.151}      \\
  S5    &$\pi\mathrm{(AB)}\hspace{1 pt} \pi^{*}\mathrm{(AB)}$ & 244             & 5.077                  & 0.049               \\
  S6    &$\pi\mathrm{(tol)}\hspace{1 pt} \pi^{*}\mathrm{(tol)}$ & 244             & 5.083                  & 0.002               \\
  \bottomrule
  \end{tabular}
  \end{table}
%
%
%  
The overall shape between the UV spectra of cAB-3c-tolane-tor (figure \ref{fig:UV_spec_cAB-for-tAB-3c-tolan}) and cAB-1c-tolane (figure \ref{fig:UV_spec_cAB-for-tAB-1c-tolan}) is very similar except some minor differences. \\
One of those differences is the huge blue shift of S1 (\SI{3.03}{\eV}) compared to cAB (\SI{2.88}{\eV}) and even cAB-1c-tolane-tor (\SI{2.90}{\eV}). 
The attachment and detachment densities of S1 are on the AB unit of the molecule.
S1 can be assigned to the n$\pi\pi^*$ transition. \\
% 
% 
% The transition now occurs between the HOMO which contains the n orbitals and the LUMO+1 containing the $\pi^{*}$ orbitals of the cAB fragment. 
% In contrast the same transition in cAB-1c-tolane-tor contained the HOMO-1 and LUMO, which indicates an appreciation in energy of those cAB centered MOs.\\
% This could be due to the reduced distance between the benzene rings on the cAB fragment of cAB-3c-tolane-tor in comparison to cAB-1c-tolane-tor. 
% The resulting increased interactions between those benzene rings %%%% Write what kind of interactions, Pauli interaction? %%%
% could increase the energy of the MOs located on the cAB fragment.\\
% Compared to the tolane centered HOMO in cAB-1c-tolane, the tolane cnetered HOMO-1 in cAB-3c-tolane decreased in energy, hinting that the less twisted geometry of the tolane leads to a MO lower in energy.
\\
In contrast to S1, the S2 of cAB-3c-tolan is almost identical in energy to the S2 in cAB-1c-tolane-tor, albeit with a lower oscillator strength. 
It can be assigned to a $\pi\pi^*$ transition and the respective attachment and detachment densities are both located on the tolane substructure of the molecule.\\
% One reason for the decreased oscillator strength could be the reduced angle $\left(\measuredangle(\ce{C-C#C}\right)$ = \SI{2.60}{\degree} vs \SI{6.57}{\degree} for cAB-1c-tolane-tor) between the benzene and the acetelyne fragment 
% which could affect conjugation and lower the energy of the particular MOs. 
%
% It too can be assigned to the $\pi\pi^{*}$ transition that is mostly occuring between the HOMO-1 and LUMO, both which are tolane centered. 
% Comparing these MOs to the tolane centered HOMO and LUMO+1 of cAB-1c-tolane one notices that the MOs of cAB-3c-tolane-tor are both lower in energy than those located on cAB-1c-tolane-tor. \\
% One reason could be the reduced angle $\left(\measuredangle(\ce{C-C#C}\right)$ = \SI{2.60}{\degree} vs \SI{6.57}{\degree} for cAB-1c-tolane-tor) between the benzene and the acetelyne fragment which could affect conjugation and lower the energy of the particular MOs. 
% Another contributing factor could be the reduced dihedral angle between the benzene fragments themselves ($\measuredangle(\ce{ph-C#C-ph})$ = \SI{4.72}{\degree} vs \SI{12.83}{\degree} for cAB-1c-tolane-tor). 
% The increased flexibility of cAB-3c-tolane-tor, granted due to the longer linkers, seem to favor a decrease in energy of the MOs centered around the tolane fragment rather than those centered around the cAB substructure. \\
\\
The excited singlet states S3, S4 and S5 are all $\pi\pi^{*}$ transitions centered around the azobenzene substructure, quite similar to those observed in cAB-1c-tolane with no significant deviation in energy. 
The attachment density of S3 is mostly located at the n-orbitlas of the azo group while the attachment densities of S4 and S5 include one of the adjacent phenyl groups.\\
\\
The sixth excited singlet state (S6) can be assigned to a $\pi\pi^*$ transition with attachment and detachment densities mainly located at one of the phenyl groups of the tolane unit.
% 
% The main MOs involved are the HOMO and the LUMO+1 with the exception of S5 which terminates at LUMO+2, LUMO+3 and LUMO+4 which are more evenly distributed among the aromatic system of the molecule. 
% The last transition S6 involves tolane centered MOs mainly the HOMO-1 and terminates similarly to S5 at LUMO+2 and LUMO+3.\\ 
%
% The reason for this is the appreciation in energy of the MOs containing the n and $\pi^{*}$ orbitals of cAB-3c-tolane in respect to cAB-1c-tolane. The HOMO-1 in cAB-1c-tolane centered around the n orbitals of the cAB fragment becomes the 
% Looking at the difference in energy of the MOs one notices that the MO centered around the n-orbitals of the cAB fragment 
%
%
\begin{figure}[H]
\centering
\begin{subfigure}{.33\textwidth}
  \centering
  \includegraphics[width=.9\linewidth]{Figures/Appendix/cAB-3c-tolane/cAB-3c-tolane-s1-att.png}
  %\caption{S1}
  % \label{fig:}
\end{subfigure}%
% \medskip
\begin{subfigure}{.33\textwidth}
  \centering
  \includegraphics[width=.9\linewidth]{Figures/Appendix/cAB-3c-tolane/cAB-3c-tolane-s2-att.png}
  %\caption{S2}
  % \label{fig:}
\end{subfigure}%
%
\begin{subfigure}{.33\textwidth}
  \centering
  \includegraphics[width=.9\linewidth]{Figures/Appendix/cAB-3c-tolane/cAB-3c-tolane-s3-att.png}
  %\caption{S3}
  % \label{fig:}
\end{subfigure}
%
\vskip\baselineskip
%
\begin{subfigure}{.33\textwidth}
  \centering
  \includegraphics[width=.9\linewidth]{Figures/Appendix/cAB-3c-tolane/cAB-3c-tolane-s1-det.png}
  \caption{S1}
  % \label{fig:}
\end{subfigure}%
% \medskip
\begin{subfigure}{.33\textwidth}
  \centering
  \includegraphics[width=.9\linewidth]{Figures/Appendix/cAB-3c-tolane/cAB-3c-tolane-s2-det.png}
  \caption{S2}
  % \label{fig:}
\end{subfigure}%
%
\begin{subfigure}{.33\textwidth}
  \centering
  \includegraphics[width=.9\linewidth]{Figures/Appendix/cAB-3c-tolane/cAB-3c-tolane-s3-det.png}
  \caption{S3}
  % \label{fig:}
\end{subfigure}
%
%
\vskip\baselineskip
\vspace{1 cm}
%
%
\begin{subfigure}{.33\textwidth}
  \centering
  \includegraphics[width=.9\linewidth]{Figures/Appendix/cAB-3c-tolane/cAB-3c-tolane-s4-att.png}
  %\caption{S1}
  % \label{fig:}
\end{subfigure}%
% \medskip
\begin{subfigure}{.33\textwidth}
  \centering
  \includegraphics[width=.9\linewidth]{Figures/Appendix/cAB-3c-tolane/cAB-3c-tolane-s5-att.png}
  %\caption{S2}
  % \label{fig:}
\end{subfigure}%
%
\begin{subfigure}{.33\textwidth}
  \centering
  \includegraphics[width=.9\linewidth]{Figures/Appendix/cAB-3c-tolane/cAB-3c-tolane-s6-att.png}
  %\caption{S3}
  % \label{fig:}
\end{subfigure}
%
\vskip\baselineskip
%
\begin{subfigure}{.33\textwidth}
  \centering
  \includegraphics[width=.9\linewidth]{Figures/Appendix/cAB-3c-tolane/cAB-3c-tolane-s4-det.png}
  \caption{S4}
  % \label{fig:}
\end{subfigure}%
% \medskip
\begin{subfigure}{.33\textwidth}
  \centering
  \includegraphics[width=.9\linewidth]{Figures/Appendix/cAB-3c-tolane/cAB-3c-tolane-s5-det.png}
  \caption{S5}
  % \label{fig:}
\end{subfigure}%
%
\begin{subfigure}{.33\textwidth}
  \centering
  \includegraphics[width=.9\linewidth]{Figures/Appendix/cAB-3c-tolane/cAB-3c-tolane-s6-det.png}
  \caption{S6}
  % \label{fig:}
\end{subfigure}
%
\caption{Detachment (red) and attachment (blue) densities at an isovalue of 0.02 for the energetically lowest four excited singlet states S1 to S6 of cAB-3c-tolane-tor at the CAM-B3LYP/6-311G* level of theory.}
\label{fig:att-det:cAB-3c-tolane-crest-rot9}
\end{figure}
%
%
% It appears due to the distorted nature of the cAB fragment inside of cAB-3c-tolane, that the energy of the AB-centered $\pi^{*}$-orbital increased and therefore the tolane-centered $\pi^{*}$-orbital became the new LUMO.
%
% \begin{figure}[H]
% \centering
% \begin{subfigure}{0.4\textwidth}
%   \centering
%   \includegraphics[width=1\linewidth]{Figures/Conformer_MO_diagrams/cAB-3c-tolane/cAB-3c-tolane-113.png}
%   \caption{HOMO-4}
%   % \label{fig:sub1}
% \end{subfigure}%
% % \medskip
% \vskip\baselineskip
% %
% \begin{subfigure}{.4\textwidth}
%   \centering
%   \includegraphics[width=1\linewidth]{Figures/Conformer_MO_diagrams/cAB-3c-tolane/cAB-3c-tolane-114.png}
%   \caption{HOMO-3}
%   % \label{fig:sub1}
% \end{subfigure}%
% %
% \begin{subfigure}{.4\textwidth}
%   \centering
%   \includegraphics[width=1\linewidth]{Figures/Conformer_MO_diagrams/cAB-3c-tolane/cAB-3c-tolane-115.png}
%   \caption{HOMO-2}
%   % \label{fig:sub1}
% \end{subfigure}%
% %
% \vskip\baselineskip
% %
% \begin{subfigure}{.4\textwidth}
%   \centering
%   \includegraphics[width=1\linewidth]{Figures/Conformer_MO_diagrams/cAB-3c-tolane/cAB-3c-tolane-116.png}
%   \caption{HOMO-1}
%   % \label{fig:sub1}
% \end{subfigure}%
% %
% %\caption{A figure with two subfigures}
% %\label{fig:test}
% \begin{subfigure}{.4\textwidth}
%   \centering
%   \includegraphics[width=1\linewidth]{Figures/Conformer_MO_diagrams/cAB-3c-tolane/cAB-3c-tolane-117.png}
%   \caption{HOMO}
%   % \label{fig:sub1}
% \end{subfigure}%
% %
% \vskip\baselineskip
% %
% \begin{subfigure}{.4\textwidth}
%   \centering
%   \includegraphics[width=1\linewidth]{Figures/Conformer_MO_diagrams/cAB-3c-tolane/cAB-3c-tolane-118.png}
%   \caption{LUMO}
%   % \label{fig:sub1}
% \end{subfigure}%
% %
% \begin{subfigure}{.4\textwidth}
%   \centering
%   \includegraphics[width=1\linewidth]{Figures/Conformer_MO_diagrams/cAB-3c-tolane/cAB-3c-tolane-119.png}
%   \caption{LUMO+1}
%   % \label{fig:sub1}
% \end{subfigure}%
% %
% \caption{Canonical orbitals of cAB-3c-tolane, plotted with an iso-value = 0.1.}
% \label{fig:MO_cAB-3c-tolane}
% \end{figure}


%\caption{A figure with two subfigures}
%\label{fig:test}
% \vskip\baselineskip
% \begin{subfigure}{.5\textwidth}
%   \centering
%   \includegraphics[width=.6\linewidth]{Figures/Conformer/cAB_3c_tolan.png}
%   \caption{A subfigure}
%   \label{fig:sub1}
% \end{subfigure}%
% \begin{subfigure}{.5\textwidth}
%   \centering
%   \includegraphics[width=.6\linewidth]{Figures/Conformer/tAB_3c_tolan.png}
%   \caption{A subfigure}
%   \label{fig:sub2}
% \end{subfigure}
% \caption{A figure with 4 subfigures}
% \label{fig:test}
% \end{figure}


% \begin{figure}[H]
%     \centering
%     \subfigure[cAB$_$1c$_$tolan]{
%         \includegraphics[scale=0.1]{Figures/Conformer/cAB_1c_tolan.png}}
%         %\subcaption{cAB_1c_tolan}
%         %\label{fig:cAB_1c_tolan_raw}
%     \hspace{\fill}
%     \subfigure[tAB$_$1c$_$tolan.png]{
%         \includegraphics[scale=0.1]{Figures/Conformer/tAB_1c_tolan.png}}  
%         %\subcaption{tAB_1c_tolan.png}
%         %\label{fig:tAB_1c_tolan_raw}
%     \caption{test}
%     \label{fig:Benchmark_all}
% \end{figure}
\section{Isomerisation process}
To gain further insight on the trans-cis isomerisation of the linked molecules, we investigated the reactive coordinate of the isomerisation (the CNNC dihedral).
% the effect of torsion on the electronic states. 
The CNNC angle was varied in steps of \SI{20}{\degree} (figure \ref{fig:tors_tAB-cAB-1c-pathway} and \ref{fig:tors_tAB-cAB-3c-pathway}) and a respective constrained optimisation of the ground state was performed. 
The energetically lowest 20 excited singlet states were then calculated and plotted against the torsional angle. 
%
%
% To gain further insight on the trans-cis isomerisation of the linked molecules, we investigated the reactive coordinate of the isomerisation (the CNNC dihedral).
% the effect of torsion on the electronic states. 
% The CNNC angle was varied in steps of \SI{20}{\degree} (figure \ref{fig:tors_tAB-cAB-1c-pathway} and \ref{fig:tors_tAB-cAB-3c-pathway}) and the resulting constrained geometry was optimized. 
% The excited states were then calculated and assigned to specific transitions, which where then plotted against the torsional angle. 
%
%
%
\begin{figure}[H]
    \centering
    \includegraphics[scale=0.21]{Figures/Torsion_pathway/Torsion_pictures_tAB_1c.png}
    \caption{Suggested photoisomerisation pathway between tAB-1c-tolane-rot9 and cAB-1c-tolane-tor.}
    \label{fig:tors_tAB-cAB-1c-pathway}
\end{figure}
%
%
\begin{figure}[H]
    \centering
    \includegraphics[scale=0.75]{Figures/Torsion_s0_s1_s2/Pictures_AB-1c-tors_corr.eps}
    \caption{Transition energies in relation to the CNNC torsional angle of tAB-1c-tolane-rot9 and cAB-1c-tolane-tor.}
    \label{fig:tors_tAB-cAB-1c}
\end{figure}
%
In figure \ref{fig:tors_tAB-cAB-1c}, the relative energies of the electronic states of tAB-1c-tolane-rot9 and cAB-1c-tolane-tor are plotted against the CNNC angle. 
Both transitions take place at the AB substructure of the linked molecules. \\
While the ground state energy (blue) reaches its energetically highest point at \SI{100}{\degree} the n$\pi^{*}$ state reaches its lowest point of energy at the same angle. 
The resulting energy gap of \SI{0.4}{\eV} is still quite large which is mainly due to the inability to obtain constrained geometries between \SI{80}{\degree} and \SI{100}{\degree}. 
The slope of the ground state indicates that its maximum energy would be near a \SI{90}{\degree} CNNC angle, which coincides with the dihedral angle at which the conical intersection of AB is located.\cite{maximum-con-inter-AB} 
Thus one can assume that the conical intersection of tAB-1c-tolane-rot9 and cAB-1c-tolane-tor is still present in AB-1c-tolane, facilitating a possible switching event of the AB subsystem.
%
% a good indication for a conical intersection between \SI{80}{\degree} and \SI{100}{\degree}. Another interesting observation is the quite flat 
% In figure \ref{fig:tors_tAB-cAB-1c}, the energy of the groundstate, the n$\pi*$ and $\pi\pi*$ transitions of tAB-1c-tolane-rot9 and cAB-1c-tolane-tor is plotted against the CNNC angle. 
%
%
%
\begin{figure}[H]
    \centering
    \includegraphics[scale=0.18]{Figures/Torsion_pathway/Torsion_pictures_tAB_3c.png}
    \caption{Suggested photoisomerisation pathway between tAB-3c-tolane-rot9 and cAB-3c-tolane-tor.}
    \label{fig:tors_tAB-cAB-3c-pathway}
\end{figure}
%
%
\begin{figure}[H]
    \centering
    \includegraphics[scale=0.75]{Figures/Torsion_s0_s1_s2/Pictures_AB-3c-tors_corr.eps}
    \caption{Relative energies of the GS and the singlet excited states in in relation to the CNNC torsional angle of tAB-3c-tolane-rot3 and cAB-3c-tolane-tor.}
    \label{fig:tors-tAB-cAB-3c}
\end{figure}
%
In the same vein the vertical excitation energies of tAB-3c-tolane-rot3 and cAB-3c-tolane-tor were plotted against the CNNC dihedral angle (figure \ref{fig:tors-tAB-cAB-3c}). 
The excited states n$\pi^{*}$ AB and $\pi\pi^{*}$ AB could be assigned to transitions at the AB substructure whereas the transition n(tol) $\pi^{*}$(AB) starts at a tolane centered MO and terminates at an AB centered MO, quite similar to a charge-transfer transition. \\
The smallest energy gap (\SI{0.6}{\eV}) between GS and the first excited state can be found at \SI{80}{\degree}, but similarly to the methylene-linked molecules the slope of the ground state (blue) suggests that a suspected conical intersection could likely be found at a CNNC angle of \SI{90}{\degree}. 
The second and third excited seem to decrease in energy respective to a increasing CNNC dihedral angle and don't vary much from each other in terms of energy.


\subsection{CNNC torsion and MO energies}
Since frontier orbitals are a useful tool to investigate the chemical reactivity of certain molecules, we were interested how the CNNC-angle influences their respective energies. \\
To reach this goal, the CNNC-angle of both the tAB-molecule and the cAB-molecule was varied in steps of \SI{20}{\degree} and the resulting geometry was optimized until the calculation no longer succeeded. The frontier orbitals were calculated at each constrained geometry and their energy was plotted in relation to their CNNC-angle.\\
%
\begin{figure}[H]
    \centering
    \includegraphics[scale=0.75]{Figures/Torsion_mo_energy/cAB-tAB-1c-tors_1.eps}
    \caption{The energies of four different molecular orbitals of tAB-1c-tolane-rot9 and cAB-1c-tolane-tor in relation to their CNNC torsional angle.}
    \label{fig:MO_tors_1c_AB}
\end{figure}
%
Figure \ref{fig:MO_tors_1c_AB} shows the LUMO+1, the LUMO, the HOMO and the HOMO-1 of tAB-1c-tolane-rot9 and cAB-1c-tolane-tor as well as their respective energies in relation to the CNNC torsional angle. \\ 
Looking at the energy values of both frontier orbitals which are located on tolane, $\pi^*$ tol and $\pi$ tol, one observes, that the torsional deformation of the molecule has no effect on their energies. This is quite remarkable, because the dihedral angle of tolane while in tAB-1c-tolane-rot9 configuration is \SI{51}{\degree} whereas in cAB-1c-tolane-tor configuration this angle amounts only to \SI{15}{\degree}, implying that the dihedral angle has only a negligible influence on the energies of both tolane-localized frontier orbitals, at least in this torsion interval.\\
% Another explanation could be that the increase of disadvantageous pi stacking increases energy at the cAB configuration thus negating the decrease in energy.
For the AB centered frontier orbitals, the opposite is true. The energy value for $\pi^*$ AB seems to decrease and reaches its minimum energy of \SI{2.24}{\eV} at a torsional angle of \SI{100}{\degree} while the energy of n AB increases to a maximum of \SI{6.19}{\eV} at the same angle. The gap between both MOs decreases from \SI{6.69}{\eV} for the cAB-1c-tolane-tor to \SI{3.95}{\eV} at at a torsional angle of \SI{100}{\degree}. 
%
%
%
%
%
\begin{figure}[H]
    \centering
    \includegraphics[scale=0.85]{Figures/Torsion_mo_energy/cAB-tAB-3c-tors_1.eps}
    \caption{The energies of ten different molecular orbitals of tAB-3c-tolane-rot3 and cAB-3c-tolane-tor in relation to their CNNC torsional angle.}
    \label{fig:tAB-cAB-3c-tors_1}
\end{figure}
%
Figure \ref{fig:tAB-cAB-3c-tors_1} shows LUMO+1, LUMO and the five energetically highest occupied MOs of tAB-3c-tolane-rot3 and their change in energy at varying degrees of torsion. \\
The trend in energy of both LUMOs seems quite similar to the on observed in figure \ref{fig:MO_tors_1c_AB}. $\pi^*\_$tol possesses a higher energy in the trans configuration than cis configuration which can be explained by the increase of the dihedral angle of tolane from \SI{3}{\degree} in cis configuration to \SI{21}{\degree} in trans configuration. $\pi^*\_$AB decreases in energy with a decreasing angle of torsion until it reaches its minimum at \SI{80}{\degree}. \\
Interestingly enough the HOMO-1 of tAB-3c-tolane seems to include the pi orbitals at the azo bridge, but as soon as the torsional angle is decreased to \SI{160}{\degree} the HOMO-1 deforms to include n orbitals at the azo bridge. The curve of n\_AB increases until it reaches its maximum at \SI{80}{\degree}. \\
$\pi$\_tol and $pi$\_(AB\_ph+tol\_ph)\_1 and 2 both show a similar trend and decrease in energy from higher torsional angles to lower ones. An interesting case is n\_AB\_2 which shape includes n orbitals at higher degrees of torsion but loses those at lower torsional angles until only the pi orbitals located on the phenyl rings adjacent to the azo bridge remain. \\
\\
To summarize one can see that the torsion mostly influences AB centered MOs, especially the ones higher in energy like n\_AB\_1 and $pi^*$\_AB. Since LUMO and LUMO+1 differ in their shape and location (AB or tolane) between \textit{cis} and \textit{trans} configuration, one could selectively target the azobenzene or the tolane fragment with certain chemical reactions depending on the specific isomer. 
% Another important aspect is the change in energy of the LUMO and LUMO+1 at the cis configuration which could be used to selectively target the tolane or azobenzene with certain electrophilic reactions. Similarly could the change in shape of the HOMO and HOMO-1 between the trans and cis configuration be used to selectively target specific fragments with nucleophilic reactions.
%
%
%
\section{Thermodynamic analysis of tAB-3c-tolane} \label{sec:thermodyn-analysis}
%
%
\begin{figure}[h]
\centering
\begin{subfigure}{.5\textwidth}
  \centering
  \includegraphics[width=.65\linewidth]{Figures/Thermo_figures/tAB-3c-tolane-flat.png}
  \caption{$E = \SI{-36606.944}{\eV}$}
  \label{fig:thermo_tAB-3c-tolane-flat}
\end{subfigure}%
\medskip
\begin{subfigure}{.5\textwidth}
  \centering
  \includegraphics[width=.8\linewidth]{Figures/Thermo_figures/tAB-3c-tolane.png}
  \caption{$E = \SI{-36607.165}{\eV}$}
  \label{fig:thermo_tAB-3c-tolane}
\end{subfigure}
\caption{The geometries of tAB-3c-tolane-flat \textbf{(a)} and tAB-3c-tolane \textbf{(b)} and their respective calculated energies in the gas phase with D3 correction.}
\label{fig:thermo_AB-3c-tolane}
\end{figure}
%
% \ref{fig:thermo_tAB-3c-tolane}
%
%
%While using CREST lead to the conformation with the lowest energy as seen in figure \ref{fig:thermo_tAB-1c-tolane}, $
%
Before using CREST to sample a much larger pool of possible conformers, we sampled and optimized the possible geometries manually which resulted in tAB-3c-tolane-flat, which is shown in figure \ref{fig:thermo_AB-3c-tolane} \textbf{(a)}.
% figure \ref{fig:thermo_AB-3c-tolane} \textbf{(a)} shows the geometry obtained manually which minimizes the GS energy, after optimisation.
Compared to tAB-3c-tolane-rot3 obtained via CREST, figure \ref{fig:thermo_AB-3c-tolane} \textbf{(b)}, the geometry of tAB-3c-tolane-flat is flatter and lacks the overlapping aromatic ring system of tAB-3c-tolane.
\\
The reason for this staggered stacked conformation of tAB-3c-tolane can be explained with different models. In the Hunter-Sanders picture each aromatic ring possesses a quadrupole moment which leads to negative partial charges above and below the center of the aromatic rings and positive partial charges around their edges.\cite{Hunter-Sanders, Rethinking-pi-pi} Thus an overlapping geometry is preferred. This view on ``$\pi$-$\pi$'' interactions" has been heavily disputed in recent years on the basis that it overstates the electrostatic quadrupole interactions and underestimates additional factors like dispersion.\cite{Rethinking-pi-pi, Reint-pi-stack} Another important factor, as Wheeler and Houk pointed out, are direct interactions between substituents adjacent to one aromatic ring and the opposing aromatic ring.\cite{wheeler-houk} This interaction can be understood qualitatively in terms of interaction between the quadrupole moment of the aromatic rings present in tolane and the local dipole introduced by the azo-bridge.\cite{wheeler-houk}. \\
To further understand those different conformations and their behavior in solution, we calculated the enthalpy as well al the entropy of both molecules in three different media via the C-PCM solvent model (see table \ref{tab:thermo_D3}).
%
%
%%%%%%%%%%%%%%%%%%%%%%%%%% Some possible explanations %%%%%%%%%%%%%%%%%%%%%%%%
%
%
%%%%%%%%%%%%%%%%%%%%%%%%%%%%%%%%%%%%%%%%%%%%%%%%%%%%%%%%%%%%%%%%%%%%%%%%%%%%%%
%$- write that the aromatic rings that oppose each other are closer in the D3 corrected version than the non D3 %coorected Version => suggests that dispersion plays a minor role in the folded shape of the molecule
%- Hunter Sanders model can explain this staggered stacked conformation of the aromatic rings by quadropol %moments in both rings where d- inside of the ring and d+ at the edges thus resulting in an overlap
%- Wheeler et al. showed that the Hunter Sanders picture too crude and in many cases the direct interaction of %subtituents with the aromatic ring is stronger thus N-ph creates a dipol which can interact in with the %quadropol of the benzene rings 
%
%
% Please add the following required packages to your document preamble:
% \usepackage{booktabs}
%
\begin{table}[H]
\caption{Enthalpy $H$, entropy $S$ and the free enthalpy $G$ in relation to the solvents MeCN, DCM and toluene of tAB-3c-tolane and tAB-3c-tolane-flat. The calculations were performed with the C-PCM\cite{c-pcm-1} solvent model and included a D3 correction.}
\vspace{0.1 cm}
\label{tab:thermo_D3}
\centering
\begin{tabular}{llSSSS}
\toprule
\multicolumn{1}{l}{Molecule (with D3)} & \multicolumn{1}{c}{Solvent} & \multicolumn{1}{c}{$H$ $\left(\frac{\SI{}{\kilo\joule}}{\SI{}{\mole}}\right)$} & \multicolumn{1}{c}{$S$ $\left(\frac{\SI{}{\joule}}{\SI{}{\mole\kelvin}}\right)$} &  \multicolumn{1}{c}{$S$ $\left(\frac{\SI{}{\joule}}{\SI{}{\mole\kelvin}}\right)$}\\ \midrule
                            & MeCN     &  1440.765       & 754.287 				&	1215.987     	 \\
tAB-3c-tolane               & DCM     &  1440.627       & 753.911 	& 1215,961 		 \\
                            & Toluene  &  1440.279       & 750.668       			&  1216,580		\\ \midrule
                            & MeCN     &  1436.606       & 725.054       			& 1220,540		\\
tAB-3c-tolane-flat          & DCM     &  1436.518       & 725.058 & 1220,451		\\
                            & Toluene  &  1438.694       & 754.258       			&  1213,925		\\ \midrule
                            & MeCN     &  4.159          & 29.234          				& -4,553			\\
Difference bet. Mol.        & DCM     &  4.109          & 28.853  	&	-4,489			\\
                            & Toluene  &  1.586          & -3.590          				& 2,656				\\ \bottomrule
\end{tabular}
\end{table}
%
Looking at tAB-3c-tolane one can see that solvents which possess a higher polarity lead to a lower enthalpy as well as a lower entropy, albeit the differences are relatively minor. Quite the opposite trend can be observed for tAB-3c-tolane-flat in which the entropy reaches its maximum in toluene and differs about $\SI{29}{\frac{\SI{}{\joule}}{\SI{}{\mole\kelvin}}}$ from acetronitrile as well as dichloromethane. \\ One explanation for the stark contrast in entropy of tAB-3c-tolane-flat dissolved in toluene and DCM or acetronitrile could be the low polarity of toluene compared with the remaining solvents. Toluene possesses a dielectric constant of 2.38 in contrast to acetonitrile (36.64) and DCM (9.08).\cite{Dieelectric-constants} Due to the nature of the C-PCM solvation model non-electrostatic effects (cavitation, dispersion and repulsion) are being disregarded which leads to the assumption that solutions with low dielectric constants interact with the molecule less than those with high dielectric constants. Thus solvents like DCM and acetonitrile seem to lower the total entropy of a dissolved molecule due to the nature of their stronger electrostatic interaction with it. \\The question why this effect appears only in tAB-3c-tolane-flat and not in tAB-3c-tolane-rot3 could be answered by looking at the higher surface area of the flat molecule in comparison to the stacked variant. This could lead to more interactions between the solution and the dissolved molecule and thus lead to a different entropy.
% This is quite suprising, because one would assume that a molecule that includes four aromatic rings and only two nitrogen atoms would be quite apolar and thus better dissolved in solvents which mirror this apolarity. One explanation could be that the non-electrostatic interactions are much more pronounced in toluene than in the more polar solvents. Due to the electrostatic nature of the C-PCM model those non-electrostatic interactions aren't accounted for and could lead to higher molecular energies than expected. \\
% While the changes in enthalpy are negligable in all 3 solvents, there is a difference in entropy especially between 
%
%- S of flat mol @toluol same entropy as non flat regardless of solvent => pi stacking between toluol and molecule instead of between mol and mol
%
%
\begin{table}[H]
\caption{Enthalpy $H$,  entropy $S$ and the free enthalpy $G$ in relation to the solvents MeCN, DCM and toluene of tAB-3c-tolane and tAB-3c-tolane-flat. The calculations were performed with the C-PCM\cite{c-pcm-1} solvent model without including a D3 correction.}
\vspace{0.1 cm}
\label{tab:thermo_noD3}
\centering
\begin{tabular}{llSSSS}
\hline
\multicolumn{1}{l}{Molecule (no D3)} & \multicolumn{1}{c}{Solvent} &  \multicolumn{1}{c}{$H$ $\left(\frac{\SI{}{\kilo\joule}}{\SI{}{\mole}}\right)$} & \multicolumn{1}{c}{$S$ $\left(\frac{\SI{}{\joule}}{\SI{}{\mole\kelvin}}\right)$} & \multicolumn{1}{c}{$G$ $\left(\frac{\SI{}{\kilo\joule}}{\SI{}{\mole}}\right)$}  \\ \midrule
                               & Acn      & 1440.765       & 753.923        				   		&1216,095	\\
tAB-3c-tolane-noD3             & DCM      & 1440.618       & 753.831 		&1215,976	\\
                               & Toluol   & 1440.296       & 750.564     						&1216,628	\\ 
\midrule
                               & Acn      & 1436.610       & 725.058 								&1220,543	\\ 
tAB-3c-tolane-flat-noD3        & DCM      & 1436.522       & 725.066	&1220,452	\\
                               & Toluol   & 1438.689       & 754.237        						&1213,927	\\ 
\midrule
                               & Acn       & 4.155          & 28.865         							&-4,447			\\
Difference bet. Mol            & DCM       & 4.096          & 28.765         	&-4,476			\\
                               & Toluol    & 1.607          & -3.674         							&2,701			\\
\bottomrule
\end{tabular}
\end{table}
%
To grasp the influence of dispersion on the trends observed in the prior calculations of those two molecules, we performed the same calculations which lead to table \ref{tab:thermo_D3} without an additional dispersion correction. The results of those calculations are presented in table \ref{tab:thermo_noD3}.\\
It appears that the dispersion correction has no major influence on the observed trends. % One could conclude that dispersion does not explain the differences in entropy between the two molecules.
%S (\SI{}{\kilo\joule\per\mole\per\kelvin})}





%%%%%%%%%%%%%%%%%% OLD FORM %%%%%%%%%%%%%%%%%%%%
\begin{comment}
\begin{table}[H]
\caption{The energy $E$, enthalpy $H$ and entropy $S$ in relation to the solvents MeCN, DCM and toluene of tAB-3c-tolane and tAB-3c-tolane-flat. The calculations were performed with the C-PCM\cite{c-pcm-1} solvent model and included a D3 correction.}
\vspace{0.3 cm}
\label{tab:thermo_D3}
\centering
\begin{tabular}{llSSS}
\toprule
\multicolumn{1}{l}{Molecule (with D3)} & \multicolumn{1}{c}{Solvent} & \multicolumn{1}{c}{$E$ (\SI{}{\eV})} & \multicolumn{1}{c}{$H$ $\left(\frac{\SI{}{\kilo\joule}}{\SI{}{\mole}}\right)$} & \multicolumn{1}{c}{$S$ $\left(\frac{\SI{}{\joule}}{\SI{}{\mole\kelvin}}\right)$} \\ \midrule
                            & MeCN     & -36607.670 & 1440.765       & 754.287         \\
tAB-3c-tolane               & DCM     & -36607.614 & 1440.627       & 753.911         \\
                            & Toluene  & -36607.431 & 1440.279       & 750.668         \\ \midrule
                            & MeCN     & -36607.456 & 1436.606       & 725.054         \\
tAB-3c-tolane-flat          & DCM     & -36607.398 & 1436.518       & 725.058         \\
                            & Toluene  & -36607.222 & 1438.694       & 754.258         \\ \midrule
                            & MeCN     & -0.214     & 4.159          & 29.234          \\
Difference bet. Mol.        & DCM     & -0.216     & 4.109          & 28.853          \\
                            & Toluene  & -0.209     & 1.586          & -3.590          \\ \bottomrule
\end{tabular}
\end{table}
%
Looking at tAB-3c-tolane one can see that solvents which possess a higher polarity lead to a lower total energy as well as lower enthalpy and entropy. This is quite suprising, because one would assume that a molecule that includes four aromatic rings and only two nitrogen atoms would be quite apolar and thus better dissolved in solvents which mirror this apolarity. One explanation could be that the non-electrostatic interactions are much more pronounced in toluene than in the more polar solvents. Due to the electrostatic nature of the C-PCM model those non-electrostatic interactions aren't accounted for and could lead to higher molecular energies than expected. \\
While the changes in enthalpy are negligable in all 3 solvents, there is a difference in entropy especially between 


- S of flat mol @toluol same entropy as non flat regardless of solvent => pi stacking between toluol and molecule instead of between mol and mol


\begin{table}[H]
\caption{The energy $E$ (eV), enthalpy $H$ and entropy $S$ in relation to the solvents MeCN, DCM and toluene of tAB-3c-tolane and tAB-3c-tolane-flat. The calculations were performed with the C-PCM\cite{c-pcm-1} solvent model without including a D3 correction.}
\vspace{0.3 cm}
\label{tab:thermo_noD3}
\centering
\begin{tabular}{llSSS}
\hline
\multicolumn{1}{l}{Molecule (no D3)} & \multicolumn{1}{c}{Solvent} & \multicolumn{1}{c}{$E$ (\SI{}{\eV})} & \multicolumn{1}{c}{$H$ $\left(\frac{\SI{}{\kilo\joule}}{\SI{}{\mole}}\right)$} & \multicolumn{1}{c}{$S$ $\left(\frac{\SI{}{\joule}}{\SI{}{\mole\kelvin}}\right)$}  \\ \midrule
                               & Acn     & -36607.671 & 1440.765       & 753.923        \\
tAB-3c-tolane-noD3             & DCM     & -36607.614 & 1440.618       & 753.831        \\
                               & Toluol  & -36607.431 & 1440.296       & 750.564        \\ 
\midrule
                               & Acn     & -36607.456 & 1436.610       & 725.058        \\ 
tAB-3c-tolane-flat-noD3        & DCM     & -36607.402 & 1436.522       & 725.066        \\
                               & Toluol  & -36607.222 & 1438.689       & 754.237        \\ 
\midrule
                               & Acn     & -0.215     & 4.155          & 28.865         \\
Difference bet. Mol            & DCM     & -0.212     & 4.096          & 28.765         \\
                               & Toluol  & -0.209     & 1.607          & -3.674         \\
\bottomrule
\end{tabular}
\end{table}

%S (\SI{}{\kilo\joule\per\mole\per\kelvin})}
\end{comment}
%
\chapter{Conclusion}
In this work, we investigated molecules consisting of a tolane and azobenzene backbone
linked by either a methylene (figure \ref{fig:summary_total} \textbf{(a)}) or a propylene (figure \ref{fig:summary_total} \textbf{(b)}) bridge.
%
\begin{figure}[H]
\centering
\begin{subfigure}{.33\textwidth}
  \centering
  \includegraphics[width=.8\linewidth]{Figures/Conformer/tAB-1c-tolane-crest-rot9.png}
  \caption{tAB-1c-tolane-rot9}
  \label{fig:summary_a}
\end{subfigure}
\begin{subfigure}{.33\textwidth}
  \centering
  \includegraphics[width=.9\linewidth]{Figures/Conformer/tAB-3c-tolane-crest-rot3.png}
  \caption{tAB-3c-tolane-rot3}
  \label{fig:summary_b}
\end{subfigure}%
\caption{Two of the investigated molecules.}
\label{fig:summary_total}
\end{figure}
%benchmakr1
The benchmark test yielded good results for the CAM-B3LYP functional, especially for \textit{cis}- and \textit{trans}-azobenzene. In the case of tolane B97M-rv seems to perform better at calculating the excited states of tolane, so accepting a slightly worse performance for cAB and tAB should be considered for future investigations of molecules with a tolane substructure. 
%seems to yield better results. %for the excited states of tolane
%
% benchmark
\begin{comment}
- For cis and trans AB CAM-B3LYP performed quite well, but for tolane B97 seems to be the better functional

An alternative functional one could use would be the B97M-rV, since it performed espe-
cially well for tolane with some deficits concerning tAB.
\end{comment}
%
\\
The CREST software was successfully used to sample through a wide range of conformers. Their GS geometries were then optimized with the CAM-B3LYP functional and for each \textit{trans}-azobenzene molecule a \textit{cis}-azobenzene molecule was generated, which served as the likely endpoint of a \textit{trans}-\textit{cis}-isomerisation. 
%
%
% GS calc
\begin{comment}
- we could successfully use CREST to sample through a wide range of conformers and could select those with the lowest energies
- those were used to construct the thermodynamical isomerisation of trans cis AB
\end{comment}

% Excited state
Analyzing their respective excited state revealed that the first excited state describes the same n$\pi$* transition centered around the AB substructure of the molecules. Differences appear in the following excited states: For the molecules possessing \textit{trans}-geometry the $\pi\pi$* centered around AB are lower in energy than the following tolane centered $\pi\pi$* transition. An exception is the S2 of tAB-3c-tolane-rot9 describing a charge-transfer transition between the HOMO centered at the tolane backbone and the tAB centered LUMO which likely arises due to the staggered stacked conformation of tAB-3c-tolane-rot9. \\
The opposite case can be observed from the \textit{cis}-molecules, cAB-1c-tolane-tor and cAB-3c-tolane-tor. Their excited states describing the $\pi\pi$* transitition centered around tolane seem to be lower in energy than the AB $\pi\pi$* transitions. The reason for this is probably the deformed $\pi$* orbitals which increase the energy of the transition. 
%
%
\begin{comment}
- tAB-1c-tolane:
-quite similar to the tAB and tolane transitions
-intensity of pi pi AB stark reduced
-S1 n pi AB
-S2 pi pi AB
-S6 pi pi tol

-tAB-3c-tolane:
-quite similar to tAB-tolane transitions overall
-intensity of pi pi AB stark reduced
-S1 n pi AB
-S2 pi(tol) pi(AB) charge transfer
-S3 pi pi AB
-S6 pi pi tol

-cAB-1c-tolane
-S1 n pi AB
-S2 pi pi tol
-S3 pi pi AB

-cAB-3c-tolane
-S1 n pi AB
-s2 pi pi tol
-s3 pi pi AB
-no CT transition

- tAB 1c + 3c: pi(AB) pi*(AB) mainly lower excited states; pi(tol) pi*(tol) higher exc. states 
- cAB 1c + 3c: pi  
\end{comment}
%
Plotting the isomerisation pathway reveals no direct crossing between GS and S1 which is likely due to the SCF not converging around a CNNC-angle between \SI{80}{\degree} and \SI{100}{\degree}. The slope of the GS and S1 indicate that a possible CI is likely at \SI{90}{\degree}, similar to azobenzene.
%
%isomerisation
\begin{comment}
-no direct crossing observed but that is mainly due to the SCF not converging around a CNNC angle between 80 and 100 deg
-slope of the GS and S1 indicate that the possible CI is likely around 90 deg, similar to AB
\end{comment}
%
The energies of the tolane centered orbitals of the methylene- and propylene-linked molecules seem rather unaffected by the changing CNNC-angle compared to those centered on the AB. This seems quite plausible since the biggest structural change is found at the AB substructure during isomerisation while the tolane backbone only deforms slightly. On the contrary both $\pi$(AB) and $\pi$*(AB) orbitals are strongly affected resulting in a HOMO-LUMO gaps of \SI{3.95}{\eV} (methylene-linked molecule) and \SI{4.19}{\eV} (propylene-linked molecule). It appears that the tolane centered $\pi^*$ of tAB-3c-tolane-rot9 is higher in energy than the AB-centered one, but this energetic order reverses for the \textit{cis}-isomer cAB-3c-tolane-tor, which could have implications for reactions targeting those molecular orbitals.\\
% In the case of the propylene-linked molecule one can observe that for the \textit{cis}-isomer the tolane centered $\pi^*$ is higher in energy than the AB centered one, but the reverse case is true for the \textit{trans}-isomer.
%MO torsion
\begin{comment}
-MO of tolane rather unaffected for both 1c and 3c
- pi AB energy increases while pi* AB energy decreases until the gap is as low as 3.95 eV (1c) and $INSERT_HOMO_LUMO_GAP$ (3c)
-For 3c: pi*(cAB) > pi*(cis-tol) but pi*(tAB) < pi*(trans-tol)
=> could have implications for reactions targeting the as an electrophile or nucleophile
\end{comment}
%
The thermodynamic calculations, including the C-PCM solvent model, showed that the different solutions had only a negligible influence on the entropy and enthalpy values of tAB-3c-tolane-rot3. On the other hand one could observe a lower entropy value regarding the acetonitrile solvent for the tAB-3c-tolane-flat molecule. It appears that polar solvents seems to lower the molecular entropy for the flat molecule, but not for tAB-3c-tolane-rot3, which could be due to tAB-3c-tolane-rot3 possessing less surface area than tAB-3c-tolane-flat and thus having a smaller area which can interact with the solvent. \\

%
%thermodynamics
%
\begin{comment}
-tAB-3c-tolane-rot3 quite similar Enthalpy and entropy in all solutions
-tAB-3-tolane-flat no big change in enthalpy
-tAB-3c-tolane-flat lower entropy in entropy in MeCN and DCM but not in toluene
=> polar solvents seem restrict the movement of the flat molecule more than non polar solvents resulting in a lower molecular entropy
=> effect not visible for tAB-3c-tolane-rot9 could be due to less area to interact with
\end{comment}


%%%%%%%%%%%%%%%%%%%%%%%%%%%%%%%%%%%%%%%%%%%%%%
\newpage
\setcounter{tocdepth}{0}

% Dinge eben ...
\makeatletter
\newcommand*{\topchapter}[2][0]{%
  \global\csdef{top@label@\the\numexpr\value{chapter}+1\relax}{#1}%
  \global\csdef{top@title@\the\numexpr\value{chapter}+1\relax}{#2}%
  \immediate\write\@auxout{%
    \unexpanded{\global\@namedef{toptitle@#1}{#2}}%
  }
  \chapter{#2}
  \label{TOP#1}
}
\newcommand*{\reftop}[1]{\hyperref[TOP#1]{TOP~#1} (\csuse{toptitle@#1})}
\renewcommand*{\thechapter}{\csuse{top@label@\arabic{chapter}}}
\makeatother
%%%%%%%%%%%%%%%%%%%%%%%%%%%%%%%%%%%%%%%%%%%%%%




\cleardoublepage
% Bibliography
%\nocite{*}
\topchapter[A]{\iflanguage{ngerman}{Literaturverzeichnis}{}\iflanguage{british}{Bibliography}{}}
\vspace{-12pt}
\printbibliography[heading=secbib]

% Acknowledgements
\cleardoublepage
\thispagestyle{empty}
% { \centering \section*{Acknowledgements}}
% We thank Adrian Dempwolff for the opportunity to work with him, his expertise and his patience. We thank the entire AK Dreuw for their moral and scientific support. We thank Lukas Schubotz for providing the \LaTeX \ template. 
% \vspace{6pt}









%First and foremost, I would like to thank \textcolor{supervisor} for the chance to work with them and for their office door being always open for me. Their immense expertise, openness, patience, experience and guidance have proven invaluable to me uncountably many times.

%My gratitude goes \textcolor{red}{\dots}. Special thanks to \textcolor{red}{\dots}

%With a special mention (alphabetical order) to \textcolor{red}{\dots} and many others, friends and family.

% \vfill

% { \centering \section*{Dedication}}

% \vspace{6pt}
 
% \textcolor{red}{\dots}
\topchapter[B]{Appendix}
%%%%%%%%%%%%%%%%%%%%%%%%%%%%%%%%%%%%%%%%%%%%%%
%\addtocontents{toc}{\begingroup\string\c@tocdepth 0\relax}
%%%%%%%%%%%%%%%%%%%%%%%%%%%%%%%%%%%%%%%%%%%%%%
%%%%%%%%%%%%%%%%%%%%%%%%%%%%%%%%%%%%%%%%%%%%%%%%%%%%%%%%%%%%%%%%%%%%%%%
\section*{Attachment and detachment densities}




% ############################ tAB ############################
\begin{figure}[h]
\centering
\begin{subfigure}{.23\textwidth}
  \centering
  \includegraphics[width=.9\linewidth]{Figures/Appendix/tAB/tAB-s1-att.png}
  %\caption{S1}
  \label{fig:}
\end{subfigure}%
% \medskip
\begin{subfigure}{.25\textwidth}
  \centering
  \includegraphics[width=.9\linewidth]{Figures/Appendix/tAB/tAB-s2-att.png}
  %\caption{S2}
  \label{fig:}
\end{subfigure}%
%
\begin{subfigure}{.25\textwidth}
  \centering
  \includegraphics[width=.9\linewidth]{Figures/Appendix/tAB/tAB-s3-att.png}
  %\caption{S3}
  \label{fig:}
\end{subfigure}
%
\begin{subfigure}{.25\textwidth}
  \centering
  \includegraphics[width=.9\linewidth]{Figures/Appendix/tAB/tAB-s4-att.png}
  %\caption{S4}
  \label{fig:}
\end{subfigure}
%
\vskip\baselineskip
%
\begin{subfigure}{.23\textwidth}
  \centering
  \includegraphics[width=.9\linewidth]{Figures/Appendix/tAB/tAB-s1-det.png}
  \caption{S1}
  \label{fig:}
\end{subfigure}%
% \medskip
\begin{subfigure}{.25\textwidth}
  \centering
  \includegraphics[width=.9\linewidth]{Figures/Appendix/tAB/tAB-s2-det.png}
  \caption{S2}
  \label{fig:}
\end{subfigure}%
%
\begin{subfigure}{.25\textwidth}
  \centering
  \includegraphics[width=.9\linewidth]{Figures/Appendix/tAB/tAB-s3-det.png}
  \caption{S3}
  \label{fig:}
\end{subfigure}
%
\begin{subfigure}{.25\textwidth}
  \centering
  \includegraphics[width=.9\linewidth]{Figures/Appendix/tAB/tAB-s4-det.png}
  \caption{S4}
  \label{fig:}
\end{subfigure}
%
\caption{Detachment (red) and attachment (blue) densities for the energetically lowest four excited singlet states S1 to S4 of tAB at the CAM-B3LYP/\\6-311G* level of theory.}
\label{fig:NTOS-tAB}
\end{figure}
\vspace{1 cm}




% ############################ tAB ############################
\begin{figure}[h]
\centering
\begin{subfigure}{.23\textwidth}
  \centering
  \includegraphics[width=.9\linewidth]{Figures/Appendix/cAB/cAB-s1-att.png}
  %\caption{S1}
  \label{fig:}
\end{subfigure}%
% \medskip
\begin{subfigure}{.25\textwidth}
  \centering
  \includegraphics[width=.9\linewidth]{Figures/Appendix/cAB/cAB-s2-att.png}
  %\caption{S2}
  \label{fig:}
\end{subfigure}%
%
\begin{subfigure}{.25\textwidth}
  \centering
  \includegraphics[width=.9\linewidth]{Figures/Appendix/cAB/cAB-s3-att.png}
  %\caption{S3}
  \label{fig:}
\end{subfigure}
%
\begin{subfigure}{.25\textwidth}
  \centering
  \includegraphics[width=.9\linewidth]{Figures/Appendix/cAB/cAB-s4-att.png}
  %\caption{S4}
  \label{fig:}
\end{subfigure}
%
\vskip\baselineskip
%
\begin{subfigure}{.23\textwidth}
  \centering
  \includegraphics[width=.9\linewidth]{Figures/Appendix/cAB/cAB-s1-det.png}
  \caption{S1}
  \label{fig:}
\end{subfigure}%
% \medskip
\begin{subfigure}{.25\textwidth}
  \centering
  \includegraphics[width=.9\linewidth]{Figures/Appendix/cAB/cAB-s2-det.png}
  \caption{S2}
  \label{fig:}
\end{subfigure}%
%
\begin{subfigure}{.25\textwidth}
  \centering
  \includegraphics[width=.9\linewidth]{Figures/Appendix/cAB/cAB-s3-det.png}
  \caption{S3}
  \label{fig:}
\end{subfigure}
%
\begin{subfigure}{.25\textwidth}
  \centering
  \includegraphics[width=.9\linewidth]{Figures/Appendix/cAB/cAB-s4-det.png}
  \caption{S4}
  \label{fig:}
\end{subfigure}
%
\caption{Detachment (red) and attachment (blue) densities for the energetically lowest four excited singlet states S1 to S4 of cAB at the CAM-B3LYP/\\6-311G* level of theory.}
\label{fig:NTOS-cAB}
\end{figure}
\vspace{1 cm}



% ############################ Tolane ############################
\begin{figure}[h]
\centering
\begin{subfigure}{.23\textwidth}
  \centering
  \includegraphics[width=.9\linewidth]{Figures/Appendix/tolane/tolane-s1-att.png}
  %\caption{S1}
  \label{fig:}
\end{subfigure}%
% \medskip
\begin{subfigure}{.25\textwidth}
  \centering
  \includegraphics[width=.9\linewidth]{Figures/Appendix/tolane/tolane-s2-att.png}
  %\caption{S2}
  \label{fig:}
\end{subfigure}%
%
\begin{subfigure}{.25\textwidth}
  \centering
  \includegraphics[width=.9\linewidth]{Figures/Appendix/tolane/tolane-s3-att.png}
  %\caption{S3}
  \label{fig:}
\end{subfigure}
%
\begin{subfigure}{.25\textwidth}
  \centering
  \includegraphics[width=.9\linewidth]{Figures/Appendix/tolane/tolane-s4-att.png}
  %\caption{S4}
  \label{fig:}
\end{subfigure}
%
\vskip\baselineskip
%
\begin{subfigure}{.23\textwidth}
  \centering
  \includegraphics[width=.9\linewidth]{Figures/Appendix/tolane/tolane-s1-det.png}
  \caption{S1}
  \label{fig:}
\end{subfigure}%
% \medskip
\begin{subfigure}{.25\textwidth}
  \centering
  \includegraphics[width=.9\linewidth]{Figures/Appendix/tolane/tolane-s2-det.png}
  \caption{S2}
  \label{fig:}
\end{subfigure}%
%
\begin{subfigure}{.25\textwidth}
  \centering
  \includegraphics[width=.9\linewidth]{Figures/Appendix/tolane/tolane-s3-det.png}
  \caption{S3}
  \label{fig:}
\end{subfigure}
%
\begin{subfigure}{.25\textwidth}
  \centering
  \includegraphics[width=.9\linewidth]{Figures/Appendix/tolane/tolane-s4-det.png}
  \caption{S4}
  \label{fig:}
\end{subfigure}
%
\caption{Detachment (red) and attachment (blue) densities for the energetically lowest four excited singlet states S1 to S4 of tolane at the CAM-B3LYP/\\6-311G* level of theory.}
\label{fig:NTOS-tolane}
\end{figure}
\vspace{1 cm}

%%%%%%%%%%%%%%%%%%%%%%%%%%%%%%%%%%%%%%%%%%%%%%%%%%%%%%%%%%%%%%%%%%%%%%%%%%%%%%%%
%                               tAB-1c-tolane-crest-rot9                                  
%%%%%%%%%%%%%%%%%%%%%%%%%%%%%%%%%%%%%%%%%%%%%%%%%%%%%%%%%%%%%%%%%%%%%%%%%%%%%%%%
\begin{figure}[h]
\centering
\begin{subfigure}{.25\textwidth}
  \centering
  \includegraphics[width=.9\linewidth]{Figures/Appendix/tAB-1c-tolane-crest-rot9/tAB-1c-tolane-s1-att.png}
  %\caption{S1}
  \label{fig:}
\end{subfigure}%
% \medskip
\begin{subfigure}{.25\textwidth}
  \centering
  \includegraphics[width=.9\linewidth]{Figures/Appendix/tAB-1c-tolane-crest-rot9/tAB-1c-tolane-s2-att.png}
  %\caption{S2}
  \label{fig:}
\end{subfigure}%
%
\begin{subfigure}{.25\textwidth}
  \centering
  \includegraphics[width=.9\linewidth]{Figures/Appendix/tAB-1c-tolane-crest-rot9/tAB-1c-tolane-s3-att.png}
  %\caption{S3}
  \label{fig:}
\end{subfigure}
%
\vskip\baselineskip
%
\begin{subfigure}{.25\textwidth}
  \centering
  \includegraphics[width=.9\linewidth]{Figures/Appendix/tAB-1c-tolane-crest-rot9/tAB-1c-tolane-s1-det.png}
  \caption{S1}
  \label{fig:}
\end{subfigure}%
% \medskip
\begin{subfigure}{.25\textwidth}
  \centering
  \includegraphics[width=.9\linewidth]{Figures/Appendix/tAB-1c-tolane-crest-rot9/tAB-1c-tolane-s2-det.png}
  \caption{S2}
  \label{fig:}
\end{subfigure}%
%
\begin{subfigure}{.25\textwidth}
  \centering
  \includegraphics[width=.9\linewidth]{Figures/Appendix/tAB-1c-tolane-crest-rot9/tAB-1c-tolane-s3-det.png}
  \caption{S3}
  \label{fig:}
\end{subfigure}
%
%
\vskip\baselineskip
\vspace{1 cm}
%
%
\begin{subfigure}{.25\textwidth}
  \centering
  \includegraphics[width=.9\linewidth]{Figures/Appendix/tAB-1c-tolane-crest-rot9/tAB-1c-tolane-s4-att.png}
  %\caption{S1}
  \label{fig:}
\end{subfigure}%
% \medskip
\begin{subfigure}{.25\textwidth}
  \centering
  \includegraphics[width=.9\linewidth]{Figures/Appendix/tAB-1c-tolane-crest-rot9/tAB-1c-tolane-s5-att.png}
  %\caption{S2}
  \label{fig:}
\end{subfigure}%
%
\begin{subfigure}{.25\textwidth}
  \centering
  \includegraphics[width=.9\linewidth]{Figures/Appendix/tAB-1c-tolane-crest-rot9/tAB-1c-tolane-s6-att.png}
  %\caption{S3}
  \label{fig:}
\end{subfigure}
%
\vskip\baselineskip
%
\begin{subfigure}{.25\textwidth}
  \centering
  \includegraphics[width=.9\linewidth]{Figures/Appendix/tAB-1c-tolane-crest-rot9/tAB-1c-tolane-s4-det.png}
  \caption{S4}
  \label{fig:}
\end{subfigure}%
% \medskip
\begin{subfigure}{.25\textwidth}
  \centering
  \includegraphics[width=.9\linewidth]{Figures/Appendix/tAB-1c-tolane-crest-rot9/tAB-1c-tolane-s5-det.png}
  \caption{S5}
  \label{fig:}
\end{subfigure}%
%
\begin{subfigure}{.25\textwidth}
  \centering
  \includegraphics[width=.9\linewidth]{Figures/Appendix/tAB-1c-tolane-crest-rot9/tAB-1c-tolane-s6-det.png}
  \caption{S6}
  \label{fig:}
\end{subfigure}
%
\caption{Detachment (red) and attachment (blue) densities for the energetically lowest four excited singlet states S1 to S4 of tAB-1c-tolane-crest-rot9 at the CAM-B3LYP/6-311G* level of theory.}
\label{fig:att-det:tAB-1c-tolane-crest-rot9}
\end{figure}



%%%%%%%%%%%%%%%%%%%%%%%%%%%%%%%%%%%%%%%%%%%%%%%%%%%%%%%%%%%%%%%%%%%%%%%%%%%%%%%%
%                               tAB-1c-tolane-crest-rot9                                  
%%%%%%%%%%%%%%%%%%%%%%%%%%%%%%%%%%%%%%%%%%%%%%%%%%%%%%%%%%%%%%%%%%%%%%%%%%%%%%%%

\begin{figure}[h]
\centering
\begin{subfigure}{.25\textwidth}
  \centering
  \includegraphics[width=.9\linewidth]{Figures/Appendix/cAB-1c-tolane/cAB-1c-tolane-s1-att.png}
  %\caption{S1}
  \label{fig:}
\end{subfigure}%
% \medskip
\begin{subfigure}{.25\textwidth}
  \centering
  \includegraphics[width=.9\linewidth]{Figures/Appendix/cAB-1c-tolane/cAB-1c-tolane-s2-att.png}
  %\caption{S2}
  \label{fig:}
\end{subfigure}%
%
\begin{subfigure}{.25\textwidth}
  \centering
  \includegraphics[width=.9\linewidth]{Figures/Appendix/cAB-1c-tolane/cAB-1c-tolane-s3-att.png}
  %\caption{S3}
  \label{fig:}
\end{subfigure}
%
\vskip\baselineskip
%
\begin{subfigure}{.25\textwidth}
  \centering
  \includegraphics[width=.9\linewidth]{Figures/Appendix/cAB-1c-tolane/cAB-1c-tolane-s1-det.png}
  \caption{S1}
  \label{fig:}
\end{subfigure}%
% \medskip
\begin{subfigure}{.25\textwidth}
  \centering
  \includegraphics[width=.9\linewidth]{Figures/Appendix/cAB-1c-tolane/cAB-1c-tolane-s2-det.png}
  \caption{S2}
  \label{fig:}
\end{subfigure}%
%
\begin{subfigure}{.25\textwidth}
  \centering
  \includegraphics[width=.9\linewidth]{Figures/Appendix/cAB-1c-tolane/cAB-1c-tolane-s3-det.png}
  \caption{S3}
  \label{fig:}
\end{subfigure}
%
%
\vskip\baselineskip
\vspace{1 cm}
%
%
\begin{subfigure}{.25\textwidth}
  \centering
  \includegraphics[width=.9\linewidth]{Figures/Appendix/cAB-1c-tolane/cAB-1c-tolane-s4-att.png}
  %\caption{S1}
  \label{fig:}
\end{subfigure}%
% \medskip
\begin{subfigure}{.25\textwidth}
  \centering
  \includegraphics[width=.9\linewidth]{Figures/Appendix/cAB-1c-tolane/cAB-1c-tolane-s5-att.png}
  %\caption{S2}
  \label{fig:}
\end{subfigure}%
%
\begin{subfigure}{.25\textwidth}
  \centering
  \includegraphics[width=.9\linewidth]{Figures/Appendix/cAB-1c-tolane/cAB-1c-tolane-s6-att.png}
  %\caption{S3}
  \label{fig:}
\end{subfigure}
%
\vskip\baselineskip
%
\begin{subfigure}{.25\textwidth}
  \centering
  \includegraphics[width=.9\linewidth]{Figures/Appendix/cAB-1c-tolane/cAB-1c-tolane-s4-det.png}
  \caption{S4}
  \label{fig:}
\end{subfigure}%
% \medskip
\begin{subfigure}{.25\textwidth}
  \centering
  \includegraphics[width=.9\linewidth]{Figures/Appendix/cAB-1c-tolane/cAB-1c-tolane-s5-det.png}
  \caption{S5}
  \label{fig:}
\end{subfigure}%
%
\begin{subfigure}{.25\textwidth}
  \centering
  \includegraphics[width=.9\linewidth]{Figures/Appendix/cAB-1c-tolane/cAB-1c-tolane-s6-det.png}
  \caption{S6}
  \label{fig:}
\end{subfigure}
%
\caption{Detachment (red) and attachment (blue) densities for the energetically lowest four excited singlet states S1 to S4 of cAB-1c-tolane at the CAM-B3LYP/6-311G* level of theory.}
\label{fig:att-det:cAB-1c-tolane-crest-rot9}
\end{figure}



%%%%%%%%%%%%%%%%%%%%%%%%%%%%%%%%%%%%%%%%%%%%%%%%%%%%%%%%%%%%%%%%%%%%%%%%%%%%%%%%
%                               cAB-3c-tolane-crest-rot9                                  
%%%%%%%%%%%%%%%%%%%%%%%%%%%%%%%%%%%%%%%%%%%%%%%%%%%%%%%%%%%%%%%%%%%%%%%%%%%%%%%%

\begin{figure}[h]
\centering
\begin{subfigure}{.33\textwidth}
  \centering
  \includegraphics[width=.9\linewidth]{Figures/Appendix/cAB-3c-tolane/cAB-3c-tolane-s1-att.png}
  %\caption{S1}
  \label{fig:}
\end{subfigure}%
% \medskip
\begin{subfigure}{.33\textwidth}
  \centering
  \includegraphics[width=.9\linewidth]{Figures/Appendix/cAB-3c-tolane/cAB-3c-tolane-s2-att.png}
  %\caption{S2}
  \label{fig:}
\end{subfigure}%
%
\begin{subfigure}{.33\textwidth}
  \centering
  \includegraphics[width=.9\linewidth]{Figures/Appendix/cAB-3c-tolane/cAB-3c-tolane-s3-att.png}
  %\caption{S3}
  \label{fig:}
\end{subfigure}
%
\vskip\baselineskip
%
\begin{subfigure}{.33\textwidth}
  \centering
  \includegraphics[width=.9\linewidth]{Figures/Appendix/cAB-3c-tolane/cAB-3c-tolane-s1-det.png}
  \caption{S1}
  \label{fig:}
\end{subfigure}%
% \medskip
\begin{subfigure}{.33\textwidth}
  \centering
  \includegraphics[width=.9\linewidth]{Figures/Appendix/cAB-3c-tolane/cAB-3c-tolane-s2-det.png}
  \caption{S2}
  \label{fig:}
\end{subfigure}%
%
\begin{subfigure}{.33\textwidth}
  \centering
  \includegraphics[width=.9\linewidth]{Figures/Appendix/cAB-3c-tolane/cAB-3c-tolane-s3-det.png}
  \caption{S3}
  \label{fig:}
\end{subfigure}
%
%
\vskip\baselineskip
\vspace{1 cm}
%
%
\begin{subfigure}{.33\textwidth}
  \centering
  \includegraphics[width=.9\linewidth]{Figures/Appendix/cAB-3c-tolane/cAB-3c-tolane-s4-att.png}
  %\caption{S1}
  \label{fig:}
\end{subfigure}%
% \medskip
\begin{subfigure}{.33\textwidth}
  \centering
  \includegraphics[width=.9\linewidth]{Figures/Appendix/cAB-3c-tolane/cAB-3c-tolane-s5-att.png}
  %\caption{S2}
  \label{fig:}
\end{subfigure}%
%
\begin{subfigure}{.33\textwidth}
  \centering
  \includegraphics[width=.9\linewidth]{Figures/Appendix/cAB-3c-tolane/cAB-3c-tolane-s6-att.png}
  %\caption{S3}
  \label{fig:}
\end{subfigure}
%
\vskip\baselineskip
%
\begin{subfigure}{.33\textwidth}
  \centering
  \includegraphics[width=.9\linewidth]{Figures/Appendix/cAB-3c-tolane/cAB-3c-tolane-s4-det.png}
  \caption{S4}
  \label{fig:}
\end{subfigure}%
% \medskip
\begin{subfigure}{.33\textwidth}
  \centering
  \includegraphics[width=.9\linewidth]{Figures/Appendix/cAB-3c-tolane/cAB-3c-tolane-s5-det.png}
  \caption{S5}
  \label{fig:}
\end{subfigure}%
%
\begin{subfigure}{.33\textwidth}
  \centering
  \includegraphics[width=.9\linewidth]{Figures/Appendix/cAB-3c-tolane/cAB-3c-tolane-s6-det.png}
  \caption{S6}
  \label{fig:}
\end{subfigure}
%
\caption{Detachment (red) and attachment (blue) densities for the energetically lowest four excited singlet states S1 to S6 of cAB-3c-tolane at the CAM-B3LYP/6-311G* level of theory.}
\label{fig:att-det:cAB-3c-tolane-crest-rot9}
\end{figure} 

%%%%%%%%%%%%%%%%%%%%%%%%%

\end{document}
